%% $RCSfile: proj_report_outline.tex,v $
%% $Revision: 1.3 $
%% $Date: 2016/06/10 03:41:54 $
%% $Author: kevin $

\documentclass[11pt
              , a4paper
              , twoside
              , openright
              ]{report}


\usepackage{float} % lets you have non-floating floats
\usepackage{color}
\usepackage[dvipsnames]{xcolor}
\usepackage{url} % for typesetting urls
\usepackage{pdfpages}
\usepackage{subcaption}
\usepackage{graphicx}
\usepackage{balance}
\usepackage{cite}
\usepackage{booktabs}
\usepackage{fancyhdr}
\usepackage{amsmath}
\usepackage{amssymb}
\usepackage{algorithmic}
\usepackage{algorithm}
\usepackage{graphicx}
\usepackage{float}
\usepackage{multicol}
\usepackage{color, colortbl}
\usepackage{enumitem}

\usepackage[normalem]{ulem}
\usepackage[font=footnotesize,labelfont=bf]{caption}

\newcommand{\bx}[1]{\textcolor{black}{#1}} 
\newcommand{\comm}[1]{\textcolor{blue}{#1}} 
\newcommand{\tran}[1]{\textcolor{black}{#1}} 
\definecolor{LightCyan}{rgb}{0.88,1,1}


% \newcommand{\qy}[1]{\textcolor{magenta}{#1}}
\newcommand{\qy}[1]{\textcolor{black}{#1}}
\newcommand{\howto}[1]{\textcolor{black}{#1}}

% \newcommand\tab[1][0.5cm]{\hspace*{#1}}

%
%  We don't want figures to float so we define
%b
\newfloat{fig}{thp}{lof}[chapter]
\floatname{fig}{Figure}

%% These are standard LaTeX definitions for the document
%%                            
\title{Energy-efficient Server Consolidation in Container-based Clouds with Evolutionary Computation Approaches}
\author{Boxiong Tan}

%% This file can be used for creating a wide range of reports
%%  across various Schools
%%
%% Set up some things, mostly for the front page, for your specific document
%
% Current options are:
% [ecs|msor|sms]          Which school you are in.
%                         (msor option retained for reproducing old data)
% [bschonscomp|mcompsci]  Which degree you are doing
%                          You can also specify any other degree by name
%                          (see below)
% [font|image]            Use a font or an image for the VUW logo
%                          The font option will only work on ECS systems
%
\usepackage[font,ecs,mcompsci]{vuwproject}

% You should specifiy your supervisor here with
\supervisors{Dr.Hui Ma, Dr.Yi Mei, and Prof. Mengjie Zhang }
% use \supervisors if there is more than one supervisor
\otherdegree{PhD}
% Unless you've used the bschonscomp or mcompsci
%  options above use
%   \otherdegree{OTHER DEGREE OR DIPLOMA NAME}
% here to specify degree

% Comment this out if you want the date printed.
\date{}

\begin{document}

% Make the page numbering roman, until after the contents, etc.
% \frontmatter

%%%%%%%%%%%%%%%%%%%%%%%%%%%%%%%%%%%%%%%%%%%%%%%%%%%%%%%

%%%%%%%%%%%%%%%%%%%%%%%%%%%%%%%%%%%%%%%%%%%%%%%%%%%%%%%

\begin{abstract}
\small
A container-based cloud is a new trend in Cloud computing that introduces more granular management of applications and reduces overheads of virtual machine (VM).  Compared with VM-based clouds, container-based clouds can further improve the energy efficiency with a finer granularity server consolidation in data centers. Current VM-based single level of server consolidation cannot be used in container-based clouds because container-based clouds have two levels of placement: container to VM and VM to Physical machine. Existing research lacks energy model and optimization algorithms that consider the joint allocation of container and VM. 
This work aims to improve energy efficiency in container-based cloud by proposing two bilevel energy models and three Evolutionary Computation (EC)-based optimization algorithms for three placement decision scenarios: initial placement of applications, periodic placement of applications, and dynamic placement of applications. The two novel bilevel energy models and three EC algorithms will contribute to better management of resources for energy efficiency in container-based clouds. 


\end{abstract}

%%%%%%%%%%%%%%%%%%%%%%%%%%%%%%%%%%%%%%%%%%%%%%%%%%%%%%%

\maketitle

% \include{acknowledge}

\tableofcontents

% we want a list of the figures we defined
\listof{fig}{Figures}

%%%%%%%%%%%%%%%%%%%%%%%%%%%%%%%%%%%%%%%%%%%%%%%%%%%%%%%

\mainmatter

%%%%%%%%%%%%%%%%%%%%%%%%%%%%%%%%%%%%%%%%%%%%%%%%%%%%%%%

% individual chapters included here
\chapter{Introduction}\label{C:intro}
This chapter introduces this research proposal. It starts with the problem statement, then outlines the motivations, research goals and the organization of this proposal.

\section{Problem Statement}
\label{sec:statement}

\bx{Cloud computing has made a huge impact on the modern software industry by offering on-demand computing capacity 
(e.g storage and computing) \cite{2010arxiv1006.0308b}.} Compared with the traditional software industry, where applications run
on individual hardware, on-demand cloud computing provides a cluster of servers for the software industry. For example, 
web service providers, such as Google and Neflix, deploy their applications on clouds. These web service providers 
do not need to purchase and maintain hardware resources. They rent hardwares from clouds.
In addition, web service providers do not need to worry 
about scalability and availability issues when demands of their applications increase. Cloud dynamically increases the capacity of applications and cloud computing services can be accessible 99.99\% of the time~\cite{adhikari:2012uq}.
\qy{Moreover, application users} can enjoy applications without experiencing breakdown and access the applications from anywhere in the world.

\bx{A major issue in cloud computing is the huge energy consumption generated by data centers.}
A typical data center consumes as much energy as 25,000 households \cite{dayarathna:2016ua}. 
This huge energy consumption has become the major expense of cloud providers. It is necessary to find ways to reduce the energy bills. 
The reduction of energy bills would benefit \qy{cloud providers, web service providers and environment.} 
Furthermore, people would pay less to use the applications on clouds. 

\bx{Generally, cloud providers can reduce the energy consumption in clouds by
improving the resource utilization of live physical machines (PMs) such as servers.} 
Studies \cite{Barroso:2007jt, Shen:2015hm} show that \qy{PMs account for} more than 40\% of energy consumption \qy{in clouds among other components such as cooling systems and network devices}. However, studies show that PMs are not used efficiently -- the average utilization of PMs is ranging from 10\% to 50\% \cite{Hameed:2016cmb}.
Therefore, we can reduce energy by improving the average utilization of PMs \qy{and turning off the ideal PMs in clouds}.

\bx{The common way to improve the utilization of PMs in clouds 
is through resource management of PMs \cite{Manvi:2014hm} (see Figure \ref{fig:workflow}).} 
\qy{ A centralized resource management system in clouds}
has two \qy{main} functionalities. \qy{First, the management} system 
allocates resources, such as CPUs and memories of PMs, \qy{for} cloud users \qy{to run} applications. \qy{Second, the management} system 
handles the fluctuation of workloads \qy{to reduce the number of potential migrations}. \qy{These two main functionalities that directly determine the utilization of PMs in clouds.} 

\bx{The two main functionalities of resource management of PMs in clouds involve four steps.}
First, the resource management system collects and analyzes the utilization information of PMs and the resources needed by applications. Next, triggered by resource analysis, the management system determines the 
placement of applications. The placement of applications includes three scenarios: initial placement of new applications; periodic placement of existing applications; 
and dynamic placement of applications for adjusting applications in a fast manner when emergency such as overloading happen. Finally, 
the management system executes the placement decision of applications to PMs. Hence, better management of resources 
in clouds contributes towards a fewer number of PMs and thus a reduction of energy consumption.



\begin{figure}
	\centering
	\includegraphics[width=0.8\textwidth]{pics/workflow_management.png}
	\caption{A workflow of resource management \cite{Mishra:2012kx}}
	\label{fig:workflow}
\end{figure}



\bx{The core strategy of resource management is server consolidation \cite{Varasteh:2015fu}.} Two different types of
server consolidation are used in clouds: static and dynamic. Static server consolidation manages resources in an off-line fashion. It is mainly 
used in the initial placement of applications and periodic placement of applications. Dynamic server consolidation manages resources in 
an on-line fashion. It is used in the dynamic placement of applications. Both static and dynamic server consolidation aim to place
applications in fewer PMs. This leads to fewer number of PMs with higher utilization and lower energy consumption.



\bx{Currently, server consolidation in clouds is based on \emph{virtualization} technology\cite{Uhlig:2005do} and the mainstream is virtual machine (VM)-based virtualization.}
Such virtualization separates the resources (e.g. CPUs and RAMs) of a PM into several parts called VMs. Hence, a PM can run multiple VMs and each VM runs an isolated operating system for applications. 
% This VM-based technology is very different from 
% traditional clouds that place each application to a single PM and lead to the low reserved utilization of PMs. 
% Compared with traditional clouds, current 
Therefore, VM-based clouds significantly improve the utilization of PMs and reduce the number of running PMs.


\bx{However, in recent years, VM-based virtualization cannot catch up with a new trend in the software industry -- Service Oriented Architecture 
(SOA) \cite{Sprott:2004wt}.} This SOA is widely used in the modern software industry because of its agility and re-usability \cite{Sprott:2004wt}.
SOA separates a centralized application into multiple distributed components called web services. 
As most web services only require a small amount of resources (e.g. 15\% of a typical CPU), 
using a VM for a web service causes resource wastage inside a VM. Consequently, the low utilization of PMs decreases
the energy efficiency in clouds.


\bx{To support SOA and further reduce energy consumption, a new container-based virtualization~\cite{Felter:2015ki, Soltesz:2007cu} has been proposed.} Containers running on top of 
VMs are called an operating system (OS) level of virtualization~\cite{Soltesz:2007cu}. Similar to VMs, 
containers provide performance and resource isolation for applications. 
Different to VMs, multiple containers can run in the same VM without interfering with each other. 
In addition, containers naturally support \emph{vertical scaling} (change capacity during runtime)~\cite{Vaquero:2011gb}. 
The vertical scaling provides resilient resources to the fluctuation of workloads. The container technology provides a new architecture for allocating applications and a finer granularity of resource management. Hence, containers have the potential to further
improve the utilization of PMs that leads to a high energy efficiency.



\bx{Although the use of containers provides the opportunity of improving the utilization of VMs and PMs, 
containers bring new challenges and difficulties to server consolidation~\cite{:2017ff}.} First, we cannot directly apply current VM-based server consolidation approaches to container-based server consolidation. It is because the VM-based consolidation is a one-level allocation problem: VMs-PMs while the container-based consolidation is a bilevel allocation problem: containers-VMs and VMs-PMs. These two levels of allocation interact with each other. Second, with the increasing capacity of containers, the vertical scaling in clouds requires the VMs to reserve sufficient resources. The interaction between VMs and containers changes 
the server consolidation into a bilevel allocation problem: VMs-PMs, and VMs-containers. Bilevel allocation problems are NP-hard~\cite{Sinha:2013tn} and therefore it is impossible to find the optimal solution within a reasonable amount of time. Hence, we need to use heuristics to find near optimal solutions for the bilevel allocation problem.

The aim of this research is to improve the energy efficiency in container-based clouds by proposing new bilevel energy models and server consolidation algorithms for three placement decision scenarios: 
initial placement of applications, periodic placement of applications, and dynamic placement of applications.


\vspace{5mm}


\section{Motivation}

The motivation for this thesis mainly includes two parts, in the first part, we illustrate three roots of container-based server consolidation problem. In the second parts, we explain the motivations for solving the sub-problems of the problem.
\begin{itemize}

\item Container is a new virtualization technology which provides an operating level of virtualization.
Figure \ref{fig:root} illustrates the root of container technology from an energy efficient point of view. Most Clouds provide a set fixed types of VM for service providers to choose. Each type of VM represents a certain amount of resources (e.g. CPU, RAM, and Storage). This service model leads to a great waste of resources for two reasons. Firstly, service providers tend to over estimate the resources for ensuring the QoS at the peak hours, hence, they often reserve more resource than they need\cite{Chaisiri:2012wg}. 
Secondly, specific types of application may use a type of resources a lot more than another \cite{Tomas:2013iv}, for example, computation intensive tasks consume CPU much more than RAM; a fixed type of VM may provides much more RAM than it needs. In order to solve this problem, overbooking strategy tends to place more VMs than the server's maximum capacity. However, this technique is highly relied on the workload prediction running in a VM. Otherwise, servers are easily overloaded. Container technique can improve the utilization by further partitioning VM into resource isolated chunks. Therefore, multiple applications can share the same VM. This technique avoids the prediction of workload as well as improving the utilization. However, it also need consolidation technique to ensure the high utilization.

\begin{figure}
	\centering
	\includegraphics[width=0.3\textwidth]{pics/problem_flow.jpeg}
	\caption{The root of container technology}
	\label{fig:root}
\end{figure}

\item This container-based Cloud certainly brings many advantages to current Cloud \cite{Felter:2015wl}. However, it also brings difficulties for server consolidation. Server consolidation problems are typically modeled as vector bin-packing problem which are NP-hard. Container-based server consolidation add another level of abstraction which makes it a two-level vector bin-packing problem. Current server consolidation methods are mostly VM-based which can not be directly applied on this problem, because two-level of bin-packing problems are interact with each other. Piraghaj \cite{Piraghaj:2016tl} proposes a two-step procedure, mapping tasks to VMs and allocation of VMs. As Mann illustrated in \cite{Mann:2016hx},  these two steps should be conducted simultaneously, otherwise it leads to local optimal. Other research \cite{} propose greedy-based heuristics on this problem. They are relatively fast in execution, but they can be easily stuck at local optimal. 

\end{itemize}

Traditional Cloud computing offers three services models: Infrastructure as a Service (IaaS), Platform as a Service (PaaS) and Software as a Service (SaaS). Both IaaS and PaaS describe how does a service provider use the cloud resources. The main difference of these two models are  
IaaS allows service providers to manage the low-level details including the operating system and libraries. While, PaaS provides a higher level of abstraction where users only focus on the application development without caring the underlying operating system and system-level of resources such as CPU cores and memories. However, one drawback of PaaS is that cloud users must make sure their applications are complete compatible with the platform. And in many of the cases, it is not the situation. In order to solve this problem, a container-based virtualization technology starts to reform the Cloud industry. Container as a Service (CaaS) 
is a new concept but it has been used in industry for many years. Containers provide an operating system-level of isolation environment for applications. It does not need a hypervisor but complete rely on the operating system. 


This exciting new technology has bring so many advantages for both Cloud users and Cloud providers. From the providers' perspective, In a large system, running VMs means there are probably many same operating systems occupying memories and storages. Lightweight containers share operating system and therefore, there are more rooms for softwares. It increases the capability of Cloud data centers. Furthermore, in terms of resource utilization, it provides much finer granularity operation than a VM-based Cloud model. Containers partition
a VM into smaller chunks so that with appropriate management, better energy efficiency can be achieved. From the cloud users' perspective, each container provides separated libraries for specific application.  Therefore, it does not contrained by the underlying platform. Like PaaS, Cloud users do not need to concern the scalability of applications. 
Therefore, CaaS can potentially become one of the main stream in the future Cloud computing industry. 

Secondly, energy-efficent computing has been the major concern since the begining of computers. Specifially, Cloud computing has become a popular form. Large-scale data centers have been built around the world. A data center can consume huge amount of energies and it needs to improve its energy-efficiency from multiple perspectives. As we discussed in the Introduction, computing servers are one of the major contribution to the energy consumption. 
And according to observation by \cite{}, the average utilization resource are still very low which causes huge energy wastage. As we mentioned above, the container technolgy provides a better way of managing resources, it has the potential to largly improve the utilization than current VM-based Cloud model because it avoids some of the major drawbacks of VM-based model. 

Thirdly, because the container technology is relatively new, previous research are mostly focus on IaaS model and so that the server consolidation has based on the VM-level. However,   

Frist is this new technology of container that can potentially change the landscape of
Cloud computing. It has so many benefits but also it brings difficulty in managing resources.

Second, from green computing point of view, we still need to manage resource so that, the 
data centers consume less energy. And container technolgy actually bring a better chance to
be more energy-efficient than previous VM based technology.
Third, it is very difficult to manage this container-based resources because of the problem-nature is too complicated. And existed algorithms can not be directly applied on it.
Fourth, the evolutioanry computation provides a good framework to handle such difficult problem.

% \textcolor{Blue}{Motivation is what is now lack from the literature.} \\

% The advantage of Platform as a Service (PaaS) has been discovered in the recent years. 
The disadvantage of tranditional IaaS model has been discovered in the recent years \cite{Mann:2016hx}.
In IaaS, on one hand, cloud customers need to manage the low-level details ranging from application capacity estimation,
resource planning and selection and deployment. 
On the other hand, Cloud providers manage resource provisioning and allocation. 
Although these two tasks are seemingly different, 


The container as a Service (CaaS) cloud model has gain increasing attention in the recent years.
However, the energy efficiency in CaaS cloud environment has not been investigate. 
Particularly, the virtual machine and container joint consolidation is the core problem.
Therefore, in this thesis, we will focus on the end-to-end energy-aware server consolidation on container-based
Cloud. In the meanwhile,  a major research direction of large scale server consolidation is also considered. 
The end-to-end server consolidation refers to the server consolidation techniques used
in the different stages throughout the routine Cloud resource management including  initial VM provisioning and placement, dynamic VM placement, and static VM placement:

\section{Research Goals}

\begin{figure}
	\centering
	\includegraphics[width=\textwidth]{pics/thesisPlan.png}
	\caption{Relationship between objectives}
	\label{fig:objectives}
\end{figure}
\bx{The overall goal of this research is to optimize energy consumption of a container-based Cloud data center using EC-based approaches for three placement decision scenarios:} application initial placement, periodic optimization, and dynamic placement. The specific research objectives of this work can be itemized as follows.
% In this thesis, we aims at providing a series of approaches to continuously optimize the a joint allocation of VMs and containers that considers three consolidation scenarios: Initialization, global consolidation, Dynamic consolidation. In addition, the static allocation normally involves with large amount of variables which is particular difficult to optimize. We are also going to propose a method to solve this problem.  These approaches combine element of AI planning, to ensure the objectives and constraint fulfillment, and of Evolutionary Computation, to evolve a population of near-optimal solutions. The research aims at determining a flexible way in creation of solutions to solve server consolidation problems. As discussed in the previous section, the research goal can be achieved in the following objectives and sub-objectives.

\subsection{Objective One: Develop EC-based approaches for the single objective joint placement of containers and VMs for application initial placement}
\label{sec:obj1}

\bx{The goal is to reduce the energy consumption in application initial placement considering container-based cloud data center.} To achieve this goal, the first step is to propose a new bilevel model for the joint placement of containers and VMs problem which considers the problem as a bilevel optimization problem. We will explore evolutionary computation based approaches to solve the bievel problem (For the sake of simplicity, we will use ``the bilevel model/problem'' to replace ``the joint placement of containers and VMs model/problem'' in the following content.). The research goal leads to three objectives as follows. 
% \textcolor{Maroon}{Currently, most research on container consolidation do not consider the two-level of allocation problem.} Unlike previous VM-based service consolidation, 
% most research focus on VM-based server consolidation technique. They often modeled the VM allocation problem as a vector bin-packing problem \cite{Zhang:2016cx}. 
% Container adds an extra layer of abstraction on top of VM. The placement problem has become a two-step procedure, in the first step, containers are packed into VMs and then VMs are consolidated into physical machines. These two steps are inter-related to each other. Previous research \cite{Piraghaj:2015uf} solve this problem in separated steps where the first step allocates containers to VMs and the second step allocates VMs to PMs with simple bin-packing heuristics. According to Mann's \cite{Mann:2016hx} observation, these two allocations should be conducted simultaneously to reach a near-optima solution, which essentially minimizes the energy consumption.

\begin{enumerate}
	\item Develop a new bilevel model to capture the relationship between containers allocation and energy consumption in container-based cloud data center.
	% \textcolor{Maroon}{This problem can be considered as a bilevel problem \cite{} the lower-level optimization: allocate containers to VMs and the upper-level: allocate VMs to PMs.} 
	% \textcolor{Maroon}{Since the existing models for container-based consolidation are based on VM-based model which incurs two problems.}
	% First problem is that they did not consider the interaction between two levels of allocation.
	% Second problem is that they did not consider balancing the residual resources (e.g between CPU and memory). 
	\bx{The goal of the first sub objective is to propose a bilevel model for the joint placement of container and VM.} 

	\bx{The major challenge is that no previous research considers the joint placement container and VM as a bilevel problem while the relationship between container, VM and energy consumption is unclear.} Specifically, three issues remain unsolved. 
	The first issue is that it is still unclear that which energy function is the best to capture the relationship between container and VM so that the overall energy is low. Specifically, the objective for the upper level - placing container to VM, is still unclear. This is because the minimum number of VM does not necessary lead to the minimum number PMs; the types of VM also play an important role.
	The second issue is that previous VM-based research do not consider the overhead of VM. However, the overhead of VMs is a major source of resource wastage (addressed in Section \ref{sec:comparison_container_vm}). Therefore, how to represent the impact of VMs remains unsolved.
	The third issue is related to a VM-based research,  Mishra \cite{Mishra:2011bz} discovers that when multiple resources are considered in the model, the balance between resources has a heavy impact on the optimization results. Therefore, in the bilevel model, the balance of resources should also be considered.

	In order to establish a bilevel model, variables, constraints and objective functions need to be clarified before applying any optimization algorithm. Each level of the problem will be formulated to a multi-dimensional vector bin packing problem. 
	We will start from the simplest case - single dimension of resource - to more general multi-dimensional resources model by reviewing a number of VM-based approaches. Specifically, we focus on their variables, constraints and objective function. Objective function is mainly related to energy consumption. Hence, energy model is another major issue to study. In addition, in the multi-dimensional resource model, we will address the balance of CPU and memory problem by investigating several resource wastage models \cite{Ferdaus:2014ep, Xu:2010vh, Gao:2013gg}. In this objective, we consider the static workload of applications, this is because the initial resource demand is often provided by the Cloud users.

	\item Propose a new EC based bilevel optimization approach to solve the application initial placement.\\
	\bx{Based on the proposed bilevel model, the goal of this sub-objective is to develop an approach for the bilevel optimization problem using nested Evolutionary algorithms \cite{Sinha:2017et}.}

	\bx{Three challenges need to be solved. First one is to understand the interaction between bilevel's placement.} In the bilevel problem, placing containers into a minimum number of VMs does not necessary lead to the minimum energy consumption. Therefore, it is still unclear that relation among the selection type of VM, placement of container and placement of VM will affect the energy consumption. Second challenge is how to design the search operators and representation. Currently two types of representation: direct and indirect representation can be considered. However, it is unclear that which one is more suitable for the nature of the bilevel problem. Third, bilevel optimization is strongly NP-hard \cite{Mathieu:2011dw}, the solution space can be non-linearity, discreteness, no-differentiability, and non-convexity. Therefore, it is extremely difficult to design a proper search mechanism to find near optimal solutions.

	In order to discover the relation among the selection type of VM, placement of container and placement of VM, we will first use one type of VMs and one type of container. By controlling these variables, the effect of different types of VMs and containers will be eliminated. Therefore, the relationship between bilevel placement would be clear. We will gradually add up variables and constraints. For the representation of bilevel problem, we will develop direct binary representation \cite{Xu:2010vh}, and indirect continuous probability representation \cite{Xiong:2014jq}. Genetic operators are also designed along with the proposed representation.
	Current nested methods have been used in solving bilevel problem, however, there is no research focus on bilevel bin-packing problem. We will investigate several approaches such as Nested Particle Swarm Optimization \cite{Li:2006br}, Differential evolution (DE) based approach \cite{Angelo:2013ee, Zhu:2006in} and Co-evolutionary approach \cite{Legillon:2012dd}.

	\item Investigate methods to improve the scalability of the EC-based bilevel optimization approach.

	\bx{Based on proposed EC-based approach, the goal of this sub-objective is to improve scalability of the approach.} Although nested approaches have been reported effective, they are very time consuming \cite{Sinha:2017et}. Therefore, this sub objective intends to explore other directions to improve the execution time. 

	\bx{Three approaches can be potentially used in improving the scalability.} The first one is single-level reduction \cite{Sinha:2017et}, which reduces the bilevel problem into a single dimensional problem. Containers can be categorized into VMs which is then placed into PMs. The combination of container must be based on the knowledge of two-level placement interaction which we discover in the previous objective. Clustering approaches such as K-means \cite{Xie:2011fj} or decision tree can be useful in categorizing containers. Then, complementary containers can be grouped to reduce the variables of placement. The challenge is to identify the features of static workload so that different workloads can be combined to fill a VM. Another way is use reinforcement learning to learn the pattern of energy-efficient combination of containers. 
	The second approach is using a divide and conquer method to split the large number of containers into smaller chunks. The main challenge is that how to split the problem is unknown. Randomly dividing is very likely lead to a sub-optimal solution.
	The third approach is combine heuristics into the EC algorithm, for example, develop a representation which is embedded with a simple heuristic (e.g First Fit). The heuristic is expected to reduce the search space so that the EC algorithm can find solution more efficiently. However, design a heuristic which embedded inside an EC algorithm is extreme difficult since evaluation of heuristic is indirect. 

	% \item Third, although nested approaches have been reported effective, they are often very time consuming. Therefore, our third sub-objective will focus on developing more efficient algorithms. There are several possible directions to be explored such as metamodeling-based methods \cite{Wang:2007em} and single-level reduction. 
		% \emph{New operators and searching mechanisms}\\
		% In order to utilize Evolutionary Computation (EC) to solve this problem, we are going to develop searching mechanisms according to the nature of problem as well as the selected representation. In order to achieve this goal, we will design several new operators. In order to evaluate the quality of these components, we will perform analytical analysis on the result.
\end{enumerate}
\subsection{Objective Two: Develop EC-based approaches for the multi-objective joint allocation problem for periodic optimization}
The goal is to develop multi-objective EC-base approaches for container-based cloud in periodic optimization with considering various types of workload to reduce the overall energy consumption.

% As previously (see Section  \ref{sec:motivation}) mentioned, the task is multi-objective: minimizing the number of migration and minimizing the overall energy consumption. This two objectives are conflicting since intensive optimization may incur a large number migration. The first challenge is how to solve the multi-objective bi-level optimization problem. In addition, we consider propose a robust periodic optimization which means the placement of applications does not affect much from the variant workloads. Therefore, we divide workloads into five categories according to Fehling \cite{Fehling:2014tl}: static, periodic, once-in-a-life-time, continuously changing, and unpredictable. Among five types of workloads, two of them:  once-in-a-life-time and unpredictable workloads are unsuitable for static placement, since their behavior are hard to foresee and plan, hence, they are normally solved by dynamic approaches which will be addressed in our third objective.  For static, periodic, and continuously changing workload, we are going to design specific solutions. We also use three questions to guide our objective.
% The robustness of a data center is particularly important. 
% The robustness measures the stableness of result of consolidation.
% Furthermore, we will investigate proactive approaches - considering future allocation.
% In order to measure the degree of robustness, we need to design a robustness measure. The second sub-objective is to design static consolidation algorithm with considering its previous immediate result. The third objective extends the second objective to a more general case, considering both previous immediate and next allocation. The evaluation of algorithm is based on analytical analysis of fitness functions and robustness measure. 

\begin{enumerate}
	\item Modify the proposed model to adapt to the multi-objective problem with various types of workload. \\
	\bx{The goal of this sub-objective is to modify previous proposed bilevel model so that it adapts to the multi-objective problem.}

	\bx{There are mainly two challenges, the first one is to add an migration model to the existing model, and the second is to adapt the model to various types of workload.} The migration model is distinct with VM-based model. Because both container and VM can be migrated, it is unclear that migration model should be added to both layer or just one. One possible solution is to represent all VM migration with container migration. It may reduce the complexity. In addition, majority traditional migration models only consider the migration number without including the size of the applications which is unrealistic. Because the size will affect the overhead on the networking. 
	The second challenge is to adapt the model to various types of workload. Previously, we simply the problem as applications can be represented as static workloads. In this sub objective, we consider three types of predictable types of workload: static, linear continuous changing and periodic workload. These workloads may be represented as a function of time. Other models may also be changed accordingly.

	We will further develop an EC-based multi-objective algorithm with aggregation approach to test the model. The aggregation approach turns a multi-objective problem into a single-objective problem by combining objectives into a single one. Therefore, we may use previous developed algorithm to solve periodic optimization problem. We will use static workload to test the model. 

	\item Propose an EC-based multi-objective algorithm for periodic optimization with Pareto front approach.\\
	\bx{The goal of this sub-objective is to develop an EC-based approach to solve the multi-objective joint allocation problem with Pareto front approach.} 

	The major challenge in this sub-objective is to design genetic operators so that the proposed algorithm can steer the search close to the correct Pareto front. The aggregation approach proposed in previous sub objective has some defects such as it cannot find the non-convex solution. A Pareto front approach is able to find a set of trade-off between objectives, therefore, we decide to explore this direction. Currently only a few research \cite{Yin:2000bt, Deb:2009jh,Deb:2010in} focus on bilevel optimization problem. This will be the first time that bilevel optimization with Pareto front approach is applied on a bilevel bin packing problem.
	% However, most of them are designed for continuous problems. Therefore, new representations and operators need to be considered for discrete problem. 

	% The assumptions for this objective, we will start from one dimensional of resource: CPU utilization. We will consider the static workload in this sub objective because they are common and easy to start with.  We can utilize the representation and problem models from previous objective. However, we need to propose new genetic operators to adapt the multi-objective problem.

	% like the case in single objective problem, we need to develop new representations, genetic operators.
	\item Propose an EC-based multi-objective algorithm for periodic optimization considering various types of predictable workload. \\
	\bx{The goal of this sub objective is to propose an approach for three predictable workloads \cite{Fehling:2014tl}: static, linear continuously changing, and periodic.} The change of problem model from static to changing workload may affect the solution space. Therefore, two unsolved the issues are new representation must be proposed so that the function of a workload can be represented in a solution. Accordingly, the second issue is new search mechanisms must be proposed.
	% Factor analysis such Principle Component Analysis \cite{Wold:1987wx} can be employed in developing new measure. Meanwhile, the representation used in static workload might not work, therefore, new representation, genetic operators need to be developed. 

	% Proactive consolidation \cite{Farahnakian:2015vj, Tan:2011jd} has driven a lot of attention in recent years. They mainly focus on making prediction of the workloads using a regression approach such as linear regression, multi-linear regression, and K-means regression. However, most of their consolidation methods are simple heuristics. In our approach, we seek to propose a combined technique.

	% \item Second, we will design a robustness measure. Previous studies only use simple measurement which counts the migration number between two static consolidation. This measurement aims at minimizing the number of migration between two  static placement processes. It may cause more migration in the next consolidation. Therefore, it needs a time-aware measure of the robustness of system. Therefore, in this objective, the first sub-problem we are going to solve is to propose a robustness measure. Currently, only a few research propose robustness aware server consolidation techniques \cite{Takouna:2014fa, Grimes:2016ia} have been proposed. They are either static threshold or probability-based threshold to measure the robustness of PMs. We will investigate an adaptive measure based on the historical data and current status.
	

	
	% \item \emph{Design a }\\

		% We will generalize the previous sub-objective to a more general one: design a time-aware allocation method which takes previous and next allocation into consider.
	\end{enumerate}

\subsection{Objective Three: Develop a hyper-heuristic single-objective Cooperative Genetic Programming (GP) approach for automatically generating dispatching rules for dynamic placement.}

\bx{The goal for this objective is to develop a cooperative GP-based hyper-heuristic algorithm so that the generated dispatching rules can achieve both fast placement and global optimization with various workloads.}

% Previously, dynamic consolidation methods,including both VM-based and container-based, are mostly based on bin-packing algorithm such as First Fit Descending and human designed heuristics. As Mann's research \cite{Mann:2015ua} shown, server consolidation is more harder than bin-packing problem because of multi-dimensional of resources and many constraints. Therefore, general bin-packing algorithms do not perform well with many constraints and specific designed heuristics only perform well in very narrow scope. Genetic programming has been used in automatically generating dispatching rules in many areas such as job shop scheduling \cite{Nguyen:2014eu}. GP also has been successfully applied in bin-packing problems \cite{Burke:2006ei}. Therefore, we will investigate GP approaches for solving the dynamic consolidation problem. We will start from considering one-level of problem: migrate one VM each time to a PM. 

% Therefore, in this objective, we will use GP to automatically generate heuristics or dispatching rules.

\begin{enumerate}

	\item Develop a GP-based hyper-heuristic (GP-HH) algorithm for the placement of container to VM. \\
	\bx{In order to develop a cooperative GP-based hyper-heuristic to the bilevel problem, it is necessary to develop a GP-HH for the single level of the problem. Therefore, the goal of this sub-objective is to develop a GP-HH algorithm for placing containers to VMs.} This task is none trivial since no GP-HH has been dynamic placement problem. Therefore, we may start from considering the features such as the status of VMs (e.g resource utilization), features of workloads (e.g resource requirement) that will affect the placement decision. We will construct primitive set with the selected features. Other unsolved issues are the functional set and search mechanism. We will use the functional set by using the general operators. The original genetic programming will be used as the search mechanism. 

	To train the GP-based hyper-heuristic, we will use the solutions in the first objective as the model solution. 

	In order to evaluate the automatically generated heuristics. We will use a widely used simulator called CloudSim \cite{}. Since our proposed algorithm is focus on one level of placement, it is equivalent to the VM-based placement problem. We will compare our heuristic to a highly cited work \cite{Beloglazov:2012ji} from Beloglazov who propose a Best Fit Decreasing heuristic for the energy consumption problem.

	\item Conduct feature extraction on the predictable workloads and unpredictable workloads. \\
	\bx{The goal of this sub-objective is to construct a GP primitive set by applying feature extraction on various types of application workload.} In previous sub-objective, we develop a baseline GP-HH on static workload. In order to develop a general GP-HH that can handle all kinds of workloads, we will extract features from predictable workloads such as linear continuous changing workloads, periodic workloads, and from unpredictable workloads: once-in-a-lifetime workloads. 

	The first challenge is to find suitable representation for workloads. Currently, representation time-series are classified into three categories: temporal, spectral and others. It is still unclear which pattern extraction technique and representation that is best for workload data. The second challenge the high dimensionality of dataset which requires a dimensionality reduction technique to reduce the number of data point. Some possible techniques are sampling, extrema extraction.

	We will test the extracted features by applying classification on the training and test set. The final features will be used in the primitive set.

	% types including static, continuously changing, and periodic workloads, and two other workload types: once-in-a-life-time, unpredictable workloads. Based on these features, 
	% \textcolor{Blue}{More...}
	% \item Investigate the possible functional operators. 

	% \bx{The goal of this sub-objective is to construct a GP functional set.}
	% \textcolor{blue}{More will come.}

	\item Develop a Cooperative GP-HH approach to evolve dispatching rules for placing container and VMs. \\
	\bx{The goal of this sub-objective is to develop a cooperative GP approach to evolve dispatching rules.} In the baseline approach, we develop a GP-HH approach for single-level of placement. However, there is a case that no current VM is suitable for a container to place in; a new VM is needed to place at this moment. This case incurs a second level of placement. 

	Therefore, to construct a complete placement dispatching rule, we will develop a cooperative GP-HH approach to solve the two-level of placement problem. We may reuse the single-level GP-HH in both level or develop a new GP-HH in the VMs to PMs level.  

	% This sub-objective explores suitable representations for GP to construct useful dispatching rules. It also proposes new genetic operators as well as search mechanisms. \textcolor{blue}{More will come.}

	\end{enumerate}

% \subsection{Objective Four (Optional) Large-scale Static Consolidation Problem}
% Propose a preprocessing method to eliminate redundant variables 
% Current static consolidation takes all servers into consider which will lead to a scalability problem. In this objective, we will investigate two branches of methods, first one categorizes a number of containers into fewer groups so that the granularity decreases \cite{Piraghaj:2015uf}. Second method categorizes PMs so that only a small number of PMs are considered. This approach will dramatically reduce the search space. The potential approaches that can be applied in this task are various clustering methods.

	% The 
	% initial placement can be considered as a two-level of multi-dimensional bin-packing problem with multi-objectives. 
	% \item First, from the perspective of \emph{Cloud resource allocation model}, 
	% 	traditional Infrastructure as a Service (IaaS) resource allocation model 
	% 	considers service allocation and VM placement as separated responsibility.
	% 	Cloud users or brokers need to concern about the resource mapping and VM selection and 
	% 	Cloud providers take care of the VM placement. 
	% 	However, as Cloud users tend to over-provision in order to satisfy the QoS, they often book more
	% 	resources than their need. This is the major reason for the low utilization in 
	% 	Cloud computing \cite{Vogels:2008bg}. This problem cannot be 
	% 	resolved solely from the Cloud providers' perspective but to change the current resource allocation model. In the new resource allocation model, the responsibility of service allocation and VM placement are in the same hand of the Cloud provider.
	% 	Thereafter, Cloud providers have the full control of resource management. 
	% 	However, this process makes the VM initial placement a more complicated problem. Currently, only few researchers \cite{} have noticed this problem and propose initial work to address this problem.
	% \item Dynamic server consolidation is the process that the resource management continuously detects the server runtime status and if one of the server is overloaded. Then,
	% one of the VM or container running inside the server will be migrated to other machine 
	% so that the applications do not suffer from a performance degradation. In a container-based
	% environment, there are three questions to be answered. \emph{When to migrate ?} refers to determine the time point that a physical server is overloaded. \emph{Which container to migrate?} refers to determine
	% which container need to be migrated so that it optimize the global energy consumption.
	% \emph{Where to migrate?} refers to determine which VM and host that a container is migrated to.  
	% Specifically, in the second question, the main idea in the literature is still simple heuristics and random selection. Therefore, we are going to investigate using a genetic programming technique to learn to choose the best. In the 
	% third question, literature also rely on simple bin-packing heuristics which do not consider the impact of environment. Therefore, we are going to propose an idea which uses the features of workload, to decide which VM
	% is the best choice.

	% \item Static server consolidation is the process that a batch of VM and container joint is
	% consolidated in order to achieve an low energy consumption status. This stage is often
	% applied when the overall energy consumption is reached a predefine threshold. The static
	% server consolidation can globally optimize the energy consumption of the data center.
	% The process is similar with the initialization stage but with different objectives and constraints.

	% \item 

	% \item Second, container as a service has become an important trend in the 		Cloud computing industry and being support by many Cloud providers 	such as Amazon, Azure and many 
	% 	open-source projects. Both Cloud users and providers 
	% 	are beneficial to its lightweight. It provides a finer granularity resource management
	% 	for Cloud providers. From Ref \cite{}, we observed that VM-level consolidation could further improve the utilization of resource as well as the footprint from traditional hypervisors. 
	% 	However, there is not much container consolidation methods were discussed in the literature. 
	% 	New models and consolidation methods need to be proposed to solve the problem. Moreover, similar to
	% % 	VM placement, container placement is also an multi-objective which need to be addressed.

	% \item Third, the joint VM and container poses another level of consolidation problem. Ref \cite{} states, 
	% 	one of the reasons that container consolidation has high SLA violations is because the 
	% 	higher migration rate of containers. Therefore, designing algorithms that dynamically select between VM and container migration based on application SLA requirement as well as the impact on energy consumption is the major concern of the joint allocation. This can be treated as a dynamic task. 

	% \item Fourth, a common problem that faced by both traditional VM-based and the recent container-based data 	center is the affinity aware resource allocation problem. 
	% 	Modern Cloud-native applications normally have more than one copy of its implementation called replica in order to resolve the stateless as well as load balancing problem. Hence, they must be allocated into different servers to maintain its reliability. Similarly, the backup of databases has the same issue. In the CaaS scenario, more constraints appear such as operating system aware allocation which means, 
	% 	certain container can only be allocated in a specific operating system \cite{}. 
	% 	The affinity-aware allocation has been discussed in the literature, 
	% 	however, they can only be applied in the VM-based data center.   
	% \item Fifth, a resource-utilization aware co-location scheme can be helpful in order to resolve the 
	% 	resource competition problem. The study is about the behavior of the applications deployed
	% 	in the same physical machine. The previous research assumes that the applications' behavior
	% 	is a priori \cite{}, however, applications' behavior can be changed over time. It is important to allocate
	% 	the compatible applications in the same physical machine so that the physical machine reaches a 
	% 	stable status.

	% \item Sixth, large scale of server consolidation has always been a challenge in a Cloud data center. 
	% 	Especially, typical number of servers in a data center is at the million-level. Many approaches 
	% 	have been proposed in the literature to resolve the problem. 
	% 	There are mainly two ways, both rely on distributed methods, 
	% 	hierarchical-based \cite{} and agent-based management systems \cite{}.
	% 	The major problem in agent-based systems is that agents rely on heavy communication to maintain a high-level utilization. Therefore, it causes heavy load in the networking. Hierarchical-based approaches are the predominate methods. Hierarchical-based methods, in essence, are centralized systems where all the states of machines are collected and analyzed. One of way to improving the effectiveness of centralized system is to reduce the size of variables without losing too much of the consolidation performance. The main idea is to eliminating the high-utilized servers so that it reduces the dimensionality. 
% \end{enumerate}
\section{Published Papers}

During the initial stage of this research, some investigation 
was carried out on the model of container-based server consolidation \cite{Tan:2017tz}. 

\begin{enumerate}
	\item Tan, B., Ma, H., Mei, Y. and Zhang, M., ``A NSGA-II-based Approach for Web Service Resource Allocation On Cloud''. \textit{
	Proceedings of 2017 IEEE Congress on Evolutioanry Computation (CEC2017). } Donostia, Spain. 5-8 June, 2017.pp.2574-2581 



\end{enumerate}


\section{Organisation of Proposal}
The remainder of the proposal is organised as follows: Chapter \ref{C:background} provides a fundamental
background of  the resource management in cloud data centers and its the energy consumption problem. It also conducts a literature review covering a range of works in this field; Chapter \ref{C:preliminary} discusses the preliminary work carried out to explore the techniques and EC-based techniques for the joint allocation of container and VMs; Chapter \ref{C:con} presents a plan
detailing this project’s intended contributions, a project timeline, and a thesis outline.





% \subsection*{Limitation of Current CaaS Server Consolidation}
% We identify the limitation of state-of-the-art server consolidation approaches 
% in terms of two phases \cite{Varasteh:2015fu} and computing system design.
% Phase 1 - Problem definition including objective functions and constraints; 
% Phase 2 - The techniques to solve the optimization problems. 
% From the computing system design point of view, we identify a cross-layer collaborative 
% technique and a VM multiplexing technique which can potentially improve the utilization of 
% computation resources.

% \subsubsection*{Limitation in the IaaS model}
% Infrastructure as a Service is one of the three basic 
% service models (others are SaaS and PaaS) in Cloud computing. Different from other two service 
% models, IaaS allows Cloud customers to manage the low-level details of 
% virtual machine sizes, security settings and network regions. 
% From the perspective of resource management, IaaS separates 
% the concern of customers' task allocation and VM placement.
% Cloud customers or brokers estimate the computational resources they need and map them into
% a set of virtual machines. Once the task allocation is done, 
% Cloud resource management system will find a set of servers and provision the required VM
% on them. 

% However, this separated responsibility has become of a major reason for the low utilization in Cloud
% data centers. Cloud customers and brokers tend to over-estimate the resource they need in order
% to maintain its QoS in the peak service time. In fact, the peak time only accounts for a small portion
% of the overall service period. 
% Although modern virtualization technology allows overbooking strategies, Cloud providers also
% need to keep monitoring the overbooked servers to avoid overload. This surely increases the 
% overhead of management.

% One of the approach to improve the resource utilization is to put 
% the responsibility of task allocation and VM consolidation in the same hand of Cloud providers,
% thereafter, Cloud provider can estimate and dynamically adjust the resource assigned to a task.
% In addition, VM multiplexing can also be used in improving the utilization.
% Mann's research \cite{} gives an example which shows a scenario that simultaneously decide the 
% task allocation and VM placement can reduce one third of the energy consumption.

% \begin{figure}
% 	\centering
% 	\includegraphics[width=0.8\textwidth]{example.png}
% 	\caption{An example shows VM multiplexing can reduce 33\% of energy consumption \cite{}}
% 	\label{fig:example}
% \end{figure}

% Another limitation of IaaS model is that it only provides limited types of VM with fixed amount resources (including CPU, memory, etc). When a customer chooses VM type, they are forced to
% choose a VM type with the enough resource of a critical resource (e.g. CPU), therefore, other resources are over-provisioned \cite{Gmach:2012uu}. In order to tackle the fixed size of resources,
% previous researchers employ an overbooking technique \cite{Tomas:2013us}, where extra
% VM can be placed in a PM dependent on the actual resource utilization in that PM instead of the VM sizes.

% As shown above, service allocation has strong influence on VM placement.
% Therefore, instead of considering of complicated algorithms to achieve server consolidation.
% We can think from system design point of view to build a cross-layer (SaaS and IaaS) 
% collaborative technique which leads to high utilization as well as energy saving. 

% Another benefit brought from this technique is that, it will further releases the burden from 
% server providers of estimating the resource requirement and resource selection. 
% The price model will follow the pay-as-you-go policy so that service providers do not need to 
% rent a certain amount of reserved resources to cover the minimum requirement.


% % \subsubsection*{Affinity aware VM Placement}
% \subsubsection*{VM Multiplexing}

% Another direction that could potentially improve the utilization of resource is VM multiplexing.
% VM multiplexing is a technique that several services are deployed into a same VM so that 
% it improve the utilization of saving a little amount of resource from running a separated VM.

% VM multiplexing is a technique that usually be conducted by a private cloud owner, because 
% private cloud owner has complete control of cloud resources and tasks, multiple tasks 
% can be consolidated in one VM. Previously, VM multiplexing is not suitable for a public cloud
% environment because the operating system cannot offer a performance isolation among the 
% applications running internally.

% With the prevalence of container technology (e.g Docker \cite{}), public cloud providers can
% offer isolated virtual environments with shared operating system 
% which is in essence VM multiplexing. 
% Containers can increase the efficiency of cloud resource 
% utilization because it is much light weighted than a VM which means it requires 
% much less resource to than a VM manager.  Additionally, containers share a single operating system,
% which offers much fast inter-container communication via system standard calls than inter-VM
% communication techniques.

% Current research mostly concentrate on VM consolidation problem. However, when the 



% % \subsubsection*{EC-based Dynamic VM Placement}
% % \subsubsection*{Large Scale of Server Consolidation Problem with Container}

% \section*{Research Goal}
% The overall goal of this thesis is to propose a server consolidation approach that
% considers the challenges from all six points mentioned 
% above when generating solutions.
% Specifically, this approach combines element of AI planning to ensure the correctness
% and constraint, and of Evolutionary Computation, 
% to evolve a population of near-optimal solution. 
% The research aims to determine a flexible way in which planning and EC can be combined to allow the creation of solution to solve consolidation problems.
% The research goal described above can be achieved by completing the following set of 
% objectives, which are intended to be used as research guides throughout this project.

% \begin{enumerate}
% 	\item First, develop an Evolutionary computation approach that to solve the 
% 	combination of service allocation and VM placement problem. 
% 	This objective has three sub-objectives:
% 	\begin{enumerate}
% 		\item \textit{Develop a formulation to capture the relationship between service allocation and VM placement.}
% 		\item \textit{Propose a representation which address this problem} 
% 		\item \itextit{Develop a multi-objective EC-based algorithm to resolve this problem.}
% 	\end{enumerate}

% 	\item Second, we would consider the multi-objective container-based consolidation problem with predefine VM types.
% 	\begin{enumerate}
% 		\item bla
% 		\item bla bla 
% 	\end{enumerate}

% 	\item Third, we would develop an EC based approach that considers the container-based consolidation problem
% 		with customized the VM types.
% 	\begin{enumerate}
% 		\item bla
% 		\item bla
% 	\end{enumerate}

% 	\item Fourth, we would develop an approach for the joint allocation container and VM. 
% 	\begin{enumerate}
% 		\item bla
% 		\item bla
% 	\end{enumerate}

% 	\item Fifth, we would consider the affinity-aware constraint in the consolidation
% 	problem.
% 	\begin{enumerate}
% 		\item bla
% 		\item bla
% 	\end{enumerate}

% 	\item Sixth, we would consider the co-location problem with aware the applications resource usage. 
% 	\begin{enumerate}
% 		\item bla
% 		\item bla
% 	\end{enumerate}

% 	\item Seventh, we would consider the large scale consolidation problem with
% 	centralized approach with a dimensionality reduction method.
% 	\begin{enumerate}
% 		\item bla
% 		\item bla
% 	\end{enumerate}

% \end{enumerate}

% \section*{Organization of Proposal}
% The remainder of the proposal is organized as follows: Chapter 2 provides a fundamental definition of the Server Consolidation problem and performs a literature review covering a range of works in this field; Chapter 3 discusses the preliminary work carried out to explore using  EC-based techniques for the integration of service allocation and VM placement problem, one of the objectives proposed in this project; Chapter 4 presents a plan detailing this project’s intended contributions, a project timeline, and a thesis outline.

% \emph{The third aspect} The dynamic 
% \textcolor{Maroon}{Brief Literature Review}

% In the previous decade, 
% Cloud providers who were focusing on ensuring quality of service, 
% has quickly expanded their infrastructures into a large scale.
% As a result, the average utilization of servers is as low as 20\%, 
% according to \cite{energy_2007}'s observation of google's datacenter, hence, the energy was largely wasted. 

% Despite the energy consumption by none information and communication technologies (ICT) equipments
% such as cooling and airing systems, 
% the energy can be derived from two aspects, first, the hardwares of servers contribute a static consumption. 
% Second, the usage of computing, storage and network resources cause a dynamic consumption. 
% Therefore, improving the efficiency of resources are also two folds, minimizing the static part and 
% delievering more performance proportional to the dynamic workload.

% These are the inituition behind server consolidation. Server consolidation improves the utilization of resources by 
% concentrates workloads in a few servers so that others can be turned off or put into sleep to save energy. 
% Therefore, it achieves reduction of static energy consumption as well as improving the utilization of resources. 
% The consolidation can be done with the help of virtual machine (VM), 
% which can be easily transport from one physical machine (PM) to another. 
% However, as the scale of reallocation of VMs become large and various quality of services (QoS) requirements have to
% be considered, an efficient automatic approach is an urgent and necessary need.
%% $RCSfile: using.tex,v $
%% $Revision: 1.1 $
%% $Date: 2010/04/23 01:57:05 $
%% $Author: kevin $
%%
\chapter{Literature Review}\label{C:background}
This chapter begins by providing a fundamental background to the field of Cloud computing in Section \ref{sec:background}.
Section \ref{} discusses the resource allocation in PaaS as well as the bi-level optimization including both biologically and non-biologically
inspired approaches; Section \ref{} delves into approaches that optimize the static placement. In addition, time-series aware problems and their approaches will be discussed; Section \ref{} discusses the issue of dynamic server consolidation and genetic programming; Section \ref{} covers approaches of scalability; Finally, Section \ref{} presents a summary of the important points identified in this review, alongside a discussion of the limitations of existing approaches.


\section{Background}

% This chapter begins by providing an overall understanding of Cloud computing and its related research field.
% Then, it narrows down to the Energy-aware resource management problem in Section \ref{C:}.


\subsection{An Overview of Cloud Computing}

Cloud computing is a paradigm that allows Cloud users and End users to acess Cloud services and applications based on their requirements regardless of where the services are. The term ``Cloud'' refers to no matter where the businesses and users are, they are able to access services. The advent of Cloud computing matches the trend of fast growing applications where these applications are deploying in a third-party data center and application users can use them without installing on their local computers.

Cloud computing has five chacteristics that makes it popular:

\begin{enumerate}
 \item On-demand self-service, it means a Cloud user can require computing resources (e.g CPU time, storage, software use) without the interaction with Cloud provider.
 \item Broad network access, Computing resources are connected and delivered over the network.
 \item Resource pool, a Cloud provider has a ``pool'' of resources which are normally virtualized servers. In IaaS, it provides predefined sizes of VMs. In PaaS, the resources are `invisible' to Cloud users who have no knowledge or ability to control. 
 \item Rapid elasticity, from the perspective of Cloud users, computing resources are assigned and released in real time. In addition, the resources assign to their software is ``infinite''. Therefore, Cloud users do not need to worried about the scalability of their applications.
 \item Measured Service provides an accurate measure of the usage of computing resources. It is fundamental to the pay-as-you-go policy.
\end{enumerate}

Cloud computing is derived from a few computing paradigms including Grid computing, service computing and etc. Grid computing allow their users to run a large-scale computational intensive tasks on a geographically distributed computing resources. Cloud computing is similar to grid computing of utilizing a group distributed
comtpuing resources. Unlike grid takes each computer as an individual resource, Cloud computing virtualizes the Physical resources into VMs and provides a centralized resource management. Service computing, including Service Oriented Architecture (SOA) and Web services, focus on constructing large scale Web-based application using the Internet. In SOA, web services are the building blocks of softwares and they can be distributed developed, deployed and composited through the Internet. The technologies, such as SOAP and REST, enables Web serivces.

% Service computing such as Service Oriented Architecture and Web Services, focus on the seamless business processes. 

% The purpose of Cloud computing is to reduce the cost of ``in-house'' 


% Cloud computing has made one critical change in software industry, it separates the role of traditional service provider into service provider and infrastructure provider. As Wei \cite{Wei:2010fn} states, ``one provides the computing of services, and the other provides the services of computing''. Therefore, this separation add one more layer between service provider and users, as: Cloud providers, Cloud users (service providers), and End users. 

% The separate responsibility between Cloud providers and Cloud users has completely reformed the software industry \cite{Buyya:2009ix} by providing three major benefits to Cloud users.
% First, Cloud users do not need upfront investment in hardwares (e.g PMs and networking devices) and pay for hardwares' maintenance. Therefore, it eliminates the risk of initial investment.
% Second, Cloud users do not need to worry about the limited resources which can obstruct the performance of their services when unexpected high demand occurs. Cloud providers off an elastic nature of Cloud which can dynamic allocates and releases resources for a software.  Cloud users only need to pay the resources that they have used under a \emph{pay-as-you-go} policy. Third, Cloud users can publish and update their applications at any location as long as there is an Internet connection. These advantages allow anyone or organization to deploy their softwares on Cloud in a reasonable price. 

% As previous section mentioned, our goal in this thesis is to help Cloud providers to increase their profit from data centers. Specifically, we focus on how to save money by cutting expenses of Cloud data centers. 
% Cloud providers have two ways of achieving this goals, one is to provide better Quality of Service (QoS) service such as high throughput, low latency, and high availability, to attract more Cloud users to use Cloud services. The other way is to save money 
% Each stakeholder has their objectives.  End users consume application. They require a guarantee quality of softwares including functional requirements which are the functionalities defined by Cloud users, and non-functional requirements such as availability, security, and network latency. Cloud users develop and deploy softwares on Cloud. They provide functional correct softwares and they also desire the non-functional requirements of softwares can be satisfied by Cloud resources. Cloud providers offer low-level resources such as computational power, storage and network bandwidth. Cloud providers want to increase their profit by attracting more Cloud users to use Cloud services and reducing the expense caused
% \begin{figure}[H]
% 	\centering
% 	\includegraphics[width=0.5\textwidth]{pics/energyConsumption.png}
% 	\caption{Energy consumption distribution of data centers \cite{Rong:2016js}}
% 	\label{fig:consumption}
% \end{figure} 

% Apart from upfront investment, energy consumption \cite{Kaplan:up01fR-k} is the major expense of data centers. Therefore, it is also the top concern of Cloud providers. Energy consumption is derived from several parts as illustrated in Figure \ref{fig:consumption}. Cooling system and servers or PMs account for a majority of the consumption. A recent survey \cite{Cho:2016kz} shows that the recent development of cooling techniques have reduced its energy consumption and now 
% server consumption has become the dominate energy consumption component. 

% \begin{figure}
% 	\centering
% 	\includegraphics[width=0.5\textwidth]{pics/util.png}
% 	\caption{Disproportionate between utilization and energy consumption \cite{Barroso:2007jt}}
% 	\label{fig:unproportional}
% \end{figure} 
% According to Hameed et al \cite{Hameed:2016cma}, PMs are far from energy-efficient. 
% The main reason for the wastage is that the energy consumption of PMs remains high even when the utilization are low (see Figure \ref{fig:unproportional}). Therefore, a concept of \emph{energy proportional computing} \cite{Barroso:2007jt} raised to address the disproportionate between utilization and energy consumption. 


% Virtualization \cite{Uhlig:2005do} is the fundamental technology that enables Cloud computing. It partitions a physical machine's resources (e.g. CPU, memory and disk) into several isolated units called virtual machines (VMs) where each VM allows an operating system running on it. This technology rooted back in the 1960s' and was originally invented to enable isolated software testing, because VMs can provide good isolation which means applications running in co-located VMs within the same PM without interfering each other \cite{Somani:2009ho}. Soon, people realized that it can be a way to improve the utilization of hardware resources: With each application deployed in a VM, a PM can run multiple applications. Later, a dynamic migration of VM was invented, which compresses and transfers a VM from one PM to another. This technique allows resource management in real time which inspires the strategy of server consolidation. 


% \subsection{Resource Management in IaaS}
% The resource management in IaaS can be roughly separated into three \cite{Svard:2015ic, Mishra:2012kx} which are applied in different scenarios: Application initialization, Prediction and Global consolidation, and Dynamic resource management (see Figure \ref{fig:management}). 

% \begin{figure}
% 	\centering
% 	\begin{subfigure}[b]{0.9\textwidth}
% 		\includegraphics[width=\textwidth]{pics/initialization.png}
% 		\caption{Initialization}
% 	\end{subfigure}
% 	\begin{subfigure}[b]{0.9\textwidth}
% 		\includegraphics[width=\textwidth]{pics/dynamic_resource.png}
% 	\caption{Dynamic resource management}
% 	\end{subfigure}
% 	\begin{subfigure}[b]{0.9\textwidth}
% 		\includegraphics[width=\textwidth]{pics/predict_consolidate.png}
% 	\caption{Prediction and Consolidation}
% 	\end{subfigure}
% 	\caption{Three stages of resource management in IaaS}
% 	\label{fig:management}
% \end{figure}


% Server consolidation is the core functionality involving in all Cloud resource management operations. These operations

% Data center constantly receives new requests for applications initialization. Once the new applications have been allocated, the utilization begins to drop. This is because, initially, applications are compactly allocated on PMs. As old applications instance are released because of canceling, the compact structure become loose. Dynamic resource management is a process which can slow the utilization from decreasing. It consolidates by re-allocating one application at a time. Finally, global consolidation is conducted periodically to dramatically improve the resource utilization.
% \begin{enumerate}
% 	\item \emph{Application initialization} is applied when new applications or new VMs arrive and the problem is to allocate them into a minimum number of PMs.

% 	In this problem, a set of applications or VMs are waiting in a queue. The resource capacity of the PM and usage by applications are characterized by a vector of resource utilizations including CPU, memory and etc. Then, the allocation system must select a minimum number of PMs to accommodate them so that after the allocation, the resource utilizations remain high. The problem is to consider the different combinations of applications so that the overall resource utilization is high. This problem is naturally modeled as a static bin-packing problem \cite{CoffmanJr:1996ui} which is a NP-hard problem meaning it is unlikely to find an optimal solution of a large problem. 

% 	\item \emph{Prediction and Global consolidation} is conducted periodically to adjust the current allocation of applications so that the overall utilization is improved.

% 	In this problem, time is discrete and it can be split into basic time frames, for example: ten seconds. A periodical operation is conducted in every $N$ time frames.
% 	A cloud data center has a highly dynamic environment with continuous arriving and releasing of applications. Releasing applications cause hollow in PMs; new arrivals cannot change the structure of current allocation. Therefore, after the initial allocation, the overall energy efficiency is likely to drop along with time elapsing. 

% 	In prediction, an optimization system takes  the current applications' utilization records as the input. Make a prediction of their utilization in the next period of time. 
% 	In Global consolidation, based on the predict utilization and the current allocation - including a list of applications/VMs and a list of PMs, the system adjusts the allocation so that the global resource utilization is improved.

% 	In comparison with initialization, instead of new arrivals, the global consolidation considers the previous allocation. Another major difference is that global consolidation needs to minimize the differences of allocation before and after the optimization. This is because the adjustment of allocation relies on a technique called live migration \cite{Clark:2005uda}, and it is a very expensive operation because it occupies the resources in both the host and the target. Therefore, global optimization must be considered as a time-dependent activity which makes the optimization even difficult.

% In comparison with dynamic consolidation, global consolidation takes a set of VMs as input instead of one. Therefore, it is time consuming and often treated as a static problem.
% 	\item \emph{Dynamic resource management} 
%  	Dynamic resource management is applied in three scenarios. \textbf{First},  it is applied when a PM is overloading. In order to prevent the QoS from dropping, an application is migrated to another PM. This is called hot-spot mitigation \cite{Mishra:2012kx}. \textbf{Second}, it is applied when a PM is under-loading. Under-loading is when a PM is in a low utilization state normally defined by a threshold. At this moment, all the applications in the under-loading PM are migrated to other active PMs, so the PM becomes empty and can be turned off. This is called dynamic consolidation. \textbf{Third}, it is applied when a PM having very high level of utilization while others having low. An adjustment is to migrate one or more application from high utilized PMs to low ones. This is called load balancing.

% 	No matter which scenario it is, a dynamic resource management always involves three steps . 
% 	\begin{itemize}
% 		\item \emph{When to migrate?} refers to determine the time point that a PM is overloaded or underloaded. It is often decide by a threshold of utilization.
% 		\item \emph{Which application to migrate?} refers to determine which application need to be migrated so that it optimize the global energy consumption.
% 		\item \emph{Where to migrate?} refers to determine which host that an application is migrated to. This step is called dynamic placement which is directly related to the consolidation, therefore, it is decisive in improving energy-efficiency. 
% 	\end{itemize}

% 	Among three operations, dynamic placement is a dynamic and on-line problem.
% 	The term ``dynamic'' means the request comes at an arbitrary time point. An on-line problem is a problem which has on-line input and requires on-line output \cite{Borodin:uQcy_H6C}. It is applied when a control system does not have the complete knowledge of future events.

% 	There are two difficulties in this operation, firstly, dynamic placement requires a fast decision while the search space is very large (e.g hundreds of thousands of PMs). Secondly, migrate one application at a time is hard to reach a global optimized state.

% \end{enumerate}



% Finally, a consolidation plan includes four major items:
% 			\begin{enumerate}
% 				\item A list of existing PMs after consolidation
% 				\item A list of new virtual machines created after consolidation
% 				\item A list of old PMs to be turned off after consolidation
% 				\item The exact placement of applications and services
% 			\end{enumerate}

% By the nature of Cloud resource management, server consolidation techniques can also be categories into static and dynamic methods \cite{Xiao:2015ik, Verma:2009wi}. Static method is a time consuming process which is often conducted off-line in a periodical fashion; initialization and global consolidation belong to this category. It provides a global optimization to the data center. Dynamic method adjusts PMs in real time. It often allocates one application at a time. Therefore, it can be executed quickly and often provides a local optimization to the data center.


















% They are bonded by the Quality of service and computing resource. Quality of service and the computing resources are the two sides of a coin. 



% Cloud users deploy their software on Clouds. They want to increase the profit by increasing income and decreasing expense. In order to accomplish this goal, they can attract more End suers by improving the functionality of softwares and the non-functionality features by guaranteeing Quality of Service (QoS). To improve the non-functionality features, Cloud users need to reserve enough resources as well as minimizing the resource so that the cost is low.

% service capacity planning is the core process. The capacity planning has two conflicting objectives, on one hand, it must meet End users' QoS requirement by using enough resources.  On the other hand, the cost must be minimized. In pre-Cloud era, the capacity planning determines the upfront investment in infrastructure, therefore, capacity, reliability, and scalability are all need to be carefully considered and balanced. In Cloud environment, the burden of capacity planning is largely released by elastic resource management and the pay-as-you-go policy.

% Cloud users identify a list of critical QoS parameters called Service Level Agreement (SLA) which specifies the non-functional requirements such as throughput, latency, and availability. These QoS parameters are mapped to resources (e.g. CPU, memory, network bandwidth) which can satisfy these requirements. Violation of SLA will lead to penalty and decreasing in number of users. Therefore, in essence, the key to attract more users is an effective resource management system which can rapidly react to the fluctuating resource demand. 

% Beside increase the income, reduce the expense is another way to improve profit. As previous section mentioned, energy consumption is the main source of expense. In energy consumption, server energy consumption is the core that needs to be improved. 

% \subsection{Energy-aware Resource Management}
% \subsection{An Overview of Evolutionary Computation}


% In order to understand Cloud computing, firstly we will illustrate the five essential elements of Cloud computing and their advantages.

% Cloud computing has five essential elements:


% \textcolor{Blue}{What is your purpose to describe the following content?}
% \textcolor{Red}{
% 	I would like to discuss the differences, advantages of disadvantage of the resource management in different service models. Therefore, after illustrate how they are work. The point is to compare the resource management. And then, lead to a new service model. And the advantage of new service model should be obvious. 
% }\\
% Traditional Cloud computing has three service model as illustrated in Figure \ref{}.
% Infrastructure as a Service (IaaS), Platform as a Service (PaaS), and Software as a Service. 
% \subsection{Resource Management}
% Scope of Cloud computing resource management.
% \begin{enumerate}
%  \item Actors
%  \item Management Objectives
%  \item Resource Types
%  \item Enabling Technologies
% \end{enumerate}







% \section{VM-based Static Consolidation Methods}

So far in the industry, most Cloud data center is based on virtual machine technology. Therefore, VM-based resource management is the mainstream in both industry and academia. 

As previous mentioned, server consolidation is one of the technique to reduce the power consumption. Various techniques have been proposed in this field, these techniques can be roughly grouped into static and dynamic approaches.

Static approaches adjust the allocation of VMs in a periodical fashion (e.g. weekly or monthly). 

Dynamic appproaches are conducted by a runtime placement manager to migrate VMs 
automatically in response to workload variations.




% Static initialization, is also frequently referred to initial placement problem \cite{Jennings:2015ht}. Whenever a request for provisioining of applications by one or more Cloud users. The resource management system schedules the applications into a set of PMs. Currently, most state-of-the-art research focus on VM-based placement, in this case, applications are installed in VMs. Therefore, ``application placement'' and ``VM placement'' are used interchangable in the literature. 

% In energy-aware resource management, the initialization has the objective of minimizing the used PMs. In literature, the static initialization problem is often modeled as the vector bin packing problem. Each application represents an item and PMs represents bins.

% A d-dimensional Vector Bin Packing Problem ($VBP_d$), give a set of items $I^1, I^2, \dots, I^n$ where each item has $d$ dimension of resources represented in real number $I^i \in R^d$. A valid solution is packing $I$ into bins $B^1, B^2, \dots, B^k$. For each bin and each dimension, the sum of resources can not exceed the capacity of bin. The goal of Vector Bin Packing problem is to find a valid solution with minimum number of bins. $VBP_d$ is an NP-hard problem.

% Because of its NP-hard nature, several researchers propose well-known bin-packing heuristics
% First Fit (FF), First Fit Decreasing (FFD), Best Fit (BF) and etc. These algorithms have constant-factor approximation to bin-packing problems. However, VM placement is more complex than bin-packing problem \cite{Mann:2015ua}, 

% Panigraph et al \cite{Panigrahy:2011wk} study variants of First Fit Decreasing (FFD) algorithms and inspired by bad isntances for FFD-type algorithms, they propose a geometric heuristics which outperform FFD-based heuristics in most of cases.

\section{Container initilization}
Initialization is one of the major step in resource management. It can be considered as a static problem \cite{Jennings:2015ht} or dynamic problem \cite{Beloglazov:2012bw}. 
As we discussed in the previous section, a dynamic allocation normally cannot gives a global optimized solution of a batch of tasks. Therefore, in the context of maximizing the energy efficiency of a data center, we category initialization into a static optimization approach.

% Previous research mainly use heuristic to solve this problem. 
Mesos \cite{Hindman:2011ux} is a platform for sharing commodity clusters between cluster computing frameworks such as Hadoop and MPI. It has a two-level of resource allocation architecture where a master node and several slave node. A master node only decides how many resources to offer to each framework (slave) based on fair sharing policy \cite{Ghodsi:2011vm}. Each slave node belongs to a cluster framework and it makes the decision of which resources to use. Each framework has to define its allocation policy. Mesos focus on sharing resources across multiple frameworks and has a better scalability since it deligates the application placement to decentralised slave nodes. The main problem for Mesos is that it does not consider energy consumption for Cloud providers with user-defined allocation policy.

Piraghaj et al \cite{Piraghaj:2016bw} propose an architecture for container-based resource management. Their allocation approach is policy-based, where it allocates each VM with containers until its estimated utilization above 90\%. This heuristic is similar to First Fit, it is fast but does not guarantee global optimal.

% \input{dynamic_consolidation}
% \section{Evolutionary Computation Approaches on Server Consolidation Problem}






% \subsection{An Overview of server consolidation}







% The reasons for energy wastage can be derived from several components of a data center, including 
% cooling systems, network equipments, and server consumption. 
% A well-accepted measurement: PUE (Power Usage Effectiveness) \cite{Belady:IMLoaM62}
% a standard measurement for data center energy efficiency which compares the 
% total power with the power used to power IT equipment (e.g. server, network equipments). 
% A recent survey \cite{Cho:2016kz} shows that the recent development of cooling techniques 
% have reduced its energy consumption and now 
% server consumption has become the dominate energy consumption component.
% Despite improvements in hardwares, various software techniques have been proposed 
% to reduce the energy consumption of servers 
% such as: Server Consolidation and Dynamic Voltage and Frequency Scaling (DVFS) \cite{}.

% Virtualization \cite{Uhlig:2005ub} is the core technology that not only enables 
% the elastic management of Cloud resource but also can be used to improve the utilization and reduce 
% energy consumption.
% It maps a physical machine's system resource - including processors, memory, and 
% other devices - into isolated units called \emph{Virtual Machines (VMs)} which allows 
% multiple operating system to run on. 
% In essence, virtualization add an extra layer of software called 
% \emph{Virtual Machine Monitor (VMM)} or \emph{hypervisors} that can deploy, 
% release and migrate VMs at runtime. 
% Numerous VMMs have been designed for x86 commodity machines such as 
% Xen \cite{Barham:2003vu}, KVM \cite{Kivity:2007wu}, and VMware ESX server \cite{Barham:2003vu}.
 
% \textcolor{Maroon}{A brief introduction of server consolidation}

% It aims at improving the income by guaranteeing \emph{Quality of Service (QoS)}
% \cite{Calheiros:2011ul} of the maximum number of applications that a datacenter can accommodate.
% Server consolidation \cite{Zhang:2010vo} is one of the widely used strategies 
% for resource management \cite{marinescu2013cloud}.
% It reduces the server energy consumption by gathering virtual machines (VMs) into a fewer 
% number of physical servers so that idle servers can be turned off. 
% The server consolidation techniques on the server-level
% have been extensively studied in the past decade \cite{}. 
% However, the recent development of container technology enables a VM-level of consolidation, which 
% has not driven much attention. 
% Container is a lightweight virtualization
% technology which allows an application running in a single container. 
% Multiple containers can be packed in a single virtual machine. 
% Two main advantages make the container popular. 
% First, containers do not need a Virtual Machine Monitor (VMM) but relies on the operating system; 
% it reduces the overhead used on managing the virtual system. 
% Second, the communication \cite{} between containers are much 
% easier (e.g. inter-process communication) than an inter-VM communication. This feature is 
% particularly useful for micro-service-based Web applications where their processes are packed
% into separated containers.
% This new technology has brought new challenges to server consolidation. 
% Traditional algorithms can not be directly applied since there is an extra level
% of virtualization. Affinity and communication aware allocation play an much important role 
% in container-based environment. Therefore, new techniques and algorithms are need to be proposed. 

% Currently, few literatures address the 

% Therefore, this thesis will focus on providing solutions to 
% container-based server consolidation.

% Mainly, there are two types of method: static and dynamic.
% Static methods are often treated as off-line approaches and applied in a periodical manner 
% where a batch of VMs are allocated to a set of servers. 
% They are conducted at a given point of time when
% the overall utilization in a data-center is degraded into a certain level: 
% e.g, a predefined CPU utilization threshold. Because static methods often consider partial or all VMs
% in a datacenter, it is often treated as a global optimization task \cite{}.
% The static method often models the problem as a off-line bin-packing problem and 
% solved with deterministic or heuristic algorithms. The goal is often to find a global optimal solution
% in terms of server utilization and other criteria.
% Dynamic method is an on-line approach. It assumes a scenario when a single server is 
% overloading with multiple VMs, migrate one of the internal VMs out from 
% the host will release the overloading. Dynamic method is used in between 
% two static consolidation processes to ease the overloaded server as well as consolidation.
% As it only moves one VM at a time, it often applies greedy-based heuristic, therefore, hard to 
% reach a global optimization.

% \textcolor{Maroon}{Difficulty of server consolidation}
% Server consolidation is often considered as 
% a global optimization problem where its goal is to minimize the energy consumption.
% Challenges are posed at different stages of consolidation process. 
% Static problem is often modeled as a bin-packing problem  \cite{Mann:2015ua} 
% which is known as NP-hard meaning it is unlikely to find an optimal solution 
% of a large problem. 
% Furthermore, server consolidation often has 
% much complicated assumptions and constraints - including multi-dimension resources, 
% migration cost, and heterogeneous bins \cite{Mann:2015ua}.
% Because of its NP-hard nature, deterministic methods such as 
% Integer Linear Programming \cite{Speitkamp:2010vp} and Mixed
% Integer Programming \cite{} are unsuitable for a large scale problem 
% because of the long computation time. 
% Heuristic methods such as First Fit Decreasing (FFD) \cite{Panigrahy:2011wk}, 
% Best Fit Decreasing (BFD) \cite{Xu:2010vh}, 
% and other bin packing algorithms are often applied to approximate the optimal solution.
% Moreover, manually designed heuristics are designed to tackle the special requirements such 
% as \cite{}. 
% Although these greedy-based heuristics can quickly solve the consolidation problem, 
% As \cite{Mann:2015ua} shown, server consolidation is a lot more harder than bin-packing problem,
% therefore, these greedy-based heuristics can not reach a good approximation but easy to 
% be stuck at a local optima.

% In addition to traditional VM-based server consolidation, container-based server consolidation
% has an extra level of virtualization which leads to an even difficult problem.
% Traditional server consolidation algorithms cannot be directly applied to 
% the problem because of the different structure and complexity. 

% This thesis, therefore, aims at
% providing an end-to-end solution to the container-based server consolidation problem.

% First, aggressive consolidation causes overloading physical resources. 
% It leads to performance degradation since the application cannot obtain enough resources
% the VM promised. It is hard to determine the maximum level of utilization of a physical machine.


% Resource allocation and scheduling is the core of resource management in Cloud computing.
% The main purpose is to satisfy both Cloud users' and Cloud providers' requirements by 
% allocating sufficient resources to incoming tasks as well as keep a high utilization of the resources.
% In order to accomplish this goal, 
% resource allocation and scheduling tasks are often treated as optimization tasks.

% An abstract model of resource allocation is shown in Figure \ref{}.





% \subsection*{An Overview of Server Consolidation}

% A Service consolidation is the process of packing virtual machines in a number of physical 
% machines in order to reach a high utilization of resource as well as using a minimum number of 
% physical machines. The key aspect of server consolidation is that, in order to achieve the 
% desired result, the permutation of virtual machines must be considered. It is important to list the 
% difference between a static and dynamic server consolidation approaches \cite{}. 
% Static 
% In static approaches, . In dynamic approaches, bla.... A typical system model for a data center
% resource management system can be seen in Figure \ref{}. 
% \begin{figure}
% 	\centering
% 	\includegraphics[width=1.0\textwidth]{pics/dataCenter-1.png}
% 	\caption{A datacenter management model \cite{Varasteh:2015fu}}
% 	\label{fig:arch}
% \end{figure}

% The taxonomy of server consolidation has not reach an agreement. In Verma's work \cite{Verma:2009wi}, they categorize it into three groups: static, semi-static and dynamic consolidation. In static consolidation, applications are placed on PM without any further movement. Semi-static refers to periodical adjustment. Dynamic consolidation is still applied on a set of VMs in response to their workload variations.









% A dynamic server consolidation approach
% can usually be decomposed into a series of steps, 
% reflecting the process required to produce a solution \cite{}. 
% These steps are shown in Figure \ref{} and discussed below:
% \begin{enumerate}
% 	\item When to migrate. Dynamic migration occurs on two scenarios: migrating VMs from overloaded server; and item migrating VMs from underloaded server.
% 	\item Which VMs to migrate.  After deciding to migrate a VM from a server, the next step is 
% 	to make a decision of which VM to migrate.
% 	\item Where to migrate the VMs. The key step is to determine where to allocate a VM which leads to global optimization.
% \end{enumerate}


% \subsection*{A Comparison between CaaS and IaaS based Cloud model}
% From a computing system design point of view, we believe service allocation and VM Placement
% are closely related and should be considered as a single allocation task.
% \subsection*{Service Allocation}
% Service allocation refers to the process of mapping a Web service on a certain type of VM.
% It is conducted by Cloud customers (e.g Web service providers) or Cloud brokers deligated by a 
% Cloud customer.
% The resource mapping involve with two steps, 
% resource demand profiling \cite{} and VM selection \cite{}. 
% Resource demand profiling is an estimation of the workload of a service. 
% Because the web application has dynamic workload over time \cite{}, 
% service providers or cloud brokers normally would like to estimate the future workload so that they 
% can choose how much resources to rent in order to guarantee the 
% Service Level Agreements (SLAs)  \cite{} to end customers. In this step, historical statistics
% are often used and based on its peak workload estimation, service providers often
% rent resources more than they need. Since the peak workload only accounts for a small portion
% of its total operation time, intelligent strategies are applied to tackle the over-provisioning and 
% under-provisioning problems.

% Public Cloud providers often provide various configurations of VM, 
% often refered to as VM types or instance types \cite{Li:2011ti}. 
% An instance type is defined as its resources such as memory size, number of processors and 
% CPU frequency. 

% Previous research focus on how to rent an appropriate amount of resources so that it minimize the 
% service providers' costs.

% Ref \cite{Candeia:2010wt} considers e-Science applications with bag-of-tasks (BoT) model. It
% aims at executing a bag of independent tasks with the least amount of Cloud resources before
% a deadline.
% Unlike a service allocation problem, where service is permanantly deployed in a reserved VM, 
% they consider an on-demand VM allocation. 
% That means, when a bag of tasks comes, the system
% dynamically assign a set of VM to execute these tasks. 
% It evaluates four heuristic algorithms and 
% concludes that a greedy-based approach achieves the best result.

% Li et al \cite{Li:2011ti} consider a dynamic cloud scheduling problem 
% from Cloud brokers' perspective. It considers scenarios such as Cloud provider changing its offer 
% (e.g changing of pricing schemes or VM types) and service performance changing, 
% a cloud broker needs to adjust the VMs allocation across multiple Cloud providers. 
% This work proposes a model which maximize a Cloud consumers' profit by adjusting VMs across
% multiple Clouds. Their model does not consider a Cloud provider's profit.

% Wang and Xia \cite{Wang:2016ui} propose a MIP formulation for energy-aware VM placement in Cloud.
% The major difference between their work and previous work is that they use a non-linear energy model \cite{Gandhi:2009wp}.  Based on this model, they consider two resources CPU and memory. In order to solve the non-linear problem, they propose a linearization method which uses piecewise linear function to approximate the non-linear objective function. In the end, they uses a relaxation method to relax the integer linear programming problem into continuous and apply a rounding function to obtain a near-optimal solution. In summary, the energy model is the key objective in consolidation problem. With different models, the applied algorithm can be very different. However, EC approaches can deal with both linear and non-linear problem without any changes. That is an obvious advantage. 

% Virtualization technology was first developed in IBM System/360 in 1960s 
% targeting a finer granularity resource management. 
% It partitions a physical machine into separated resources 
% called virtual machine (VM) which can be allocated 
% and moved from one server to another. 
% This flexibility not only allows resources to be managed in a dynamic manner, 
% but also enables server consolidation.

% In a Cloud datacenter, server consolidation is used as technique to combat \emph{server sprawl}.
% Server sprawl refers to the low utilization of physical servers. The main cause for server sprawl is
% the requirement of running applications in isolation \cite{Vogels:2008bg}. 
% That is, an application is deployed in one or more servers which 
% offer much more resources than it needs. 
% With full virtualization \cite{}, a server's physical resources including CPUs, memory, and I/O devices
% are divided into finer granularity level of resources.
% Virtual machines offer different sizes of resources that 
% can be choosed to satisfy different demands from applications. 

% \subsection*{An Overview of Evolutionary Computation}


% \subsection*{Initial Placement}

% \section*{Container-based VM Multiplexing}
% Container has been introduced back in the 1980s' \cite{}. The recent development of container
% allows only one process running in a container; this is revolutionary invention is called application
% container. It plays an important role in Cloud computing since it is lightweighted, easier to configure
% and enable finer adjustment than the VM-based resource management.


% \begin{figure}
% 	\centering
% 	\includegraphics[width=0.6\textwidth]{pics/container.png}
% 	\caption{A Container as a Service Deployment Model}
% 	\label{fig:container}
% \end{figure}

% % % % % \section*{Dynamic VM Placement}
% % % % The traditional dynamic VM placement approaches can be categorized into three groups: Heuristic
% % % based approaches, deterministic approaches and dyanmic techniques with load prediction.

% % However, we believe the dynamic VM placement problem is highly dependent on an overall workload

% \section*{Large scale VM Placement}
% If you are writing an MSc or PhD thesis you should \emph{not} be using this style. Instead use \verb=vuwthesis=, which is based on the book style, and conforms to the VUW thesis rules. The thesis style is rather different from the project report style. 

% This document is formatted using a local (to ECS and MSOR at VUW) style file. When you write your project report you should be very careful when changing the beginning. The document class settings should read:

% \begin{verbatim}
% \documentclass[11pt
%               , a4paper
%               , twoside
%               , openright
%               ]{report}
% \end{verbatim}
% The options to the document class specify that:
% \begin{itemize}
% \item 11pt font is to be used for the main body text,
% \item  we will print on A4 paper, 
% \item we will use duplex (two-sided) printing,
% \item we want chapters to start on a right-hand page. 
% \end{itemize}

% The opitons you supply to the  \texttt{vuwproject} style will depend upon
% what you are using the style for.

% \subsection{Specifying the details}
% The \texttt{vuwproject} style sets up the front page properly, and provides various commands allowing you to specify the author, title, supervisor or supervisors, the school from which the report is being submitted and the degree that the report is being submitted for. The style has deliberately been designed to do as little as possible. This means that your document can easily be re-formatted as a technical report, or for submission to a conference or journal by using the appropriate style.

% It is also possible to use the style to easily produce documents on a
% stand-alone computer where your \LaTeX installtion might not have all
% of the  files and fonts available to machines within ECS or MSOR.

% Most of the options to the \texttt{vuwproject} style are currently a simple
% choice and there's a default that will make it obvious if you do not make
% a choice.

% Use one of the following options to use fonts available on ECS/MSOR machines
% or to use images that imitate them (assumes you have copies of the images)
% \begin{itemize}
% \item \verb+font+
% \item \verb+image+
% \end{itemize}

% Use one of the following options to set the school,
% \begin{itemize}
% \item \verb+ecs+
% \item \verb+msor+
% \end{itemize}

% Use one of the following options to choose a pre-defined degree,
% \begin{itemize}
% \item \verb+bschonscomp+
% \item \verb+mcompsci+
% \end{itemize}

% or use this command to use an explicit degree or diploma name
% \begin{itemize}
% \item \verb+\otherdegree{DEGREE OR DIPLOMA NAME}+
% \end{itemize}

% So, for example, to submit a report for the Master of Comp Sci degree, which
% the style knows about, from within ECS, using the images, you'ld ensure the
%  \texttt{vuwproject} line options looked like:

% \begin{verbatim}
% \usepackage[image,ecs,mcompsci]{vuwproject}
% \end{verbatim}

% whereas for a degree from within MSOR, when creating the final version on
% an ECS or MSOR machine where you have access to the fonts, you would use
% these options

% \begin{verbatim}
% \usepackage[font,msor]{vuwproject}
% \end{verbatim}


% and add the other degree's name using this command 

% \begin{verbatim}
% \otherdegree{DEGREE OR DIPLOMA NAME}
% \end{verbatim}

% To specify the supervisor or supervisors use either of the following commands in the preamble.
% \begin{itemize}
% \item \verb+\supervisor{The Supervisor}+
% \item \verb+\supervisors{Super 1 and Super 2}+
% \end{itemize}

% If you fail to set any degree or supervisor, or the school, then the front page will report this.

% The \texttt{vuwproject} style also sets the default font to be Palatino, using the \texttt{mathpazo} package. Palatino is one of VUW's `offical' fonts, and is the font used for the heading on the front page. The \texttt{mathpazo} package also typesets maths in a style which suits Palatino. 

% \section{Copying the style}
% If you want to write your project report away from VUW you will need to make your own copy of the \texttt{vuwproject} style.

% You can find out where the original lives by reading the messages that \LaTeX\ prints when it is run.

% Alternatively, you can down load a copy of the  \texttt{vuwproject} style from
% the ECS webpages.

% Any changes made to your own copy of the \texttt{vuwproject} style will not be reflected in the original, and \textit{vice versa}. Hence it makes sense to leave this as it is, and use a local style file for your own definitions.   

\chapter{Preliminary Work}\label{C:preliminary}
This chapter presents the initial work for investigating the first sub objective in objective one -- a bilevel model including power model, variables, and constraints. Furthermore, we also consider a multi-objective bilevel mode with two optimization objectives: minimizing the energy consumption and minimizing the total price of the used virtual machines (VMs). In addition, this work investigates an NSGA-II algorithm for solving the initial placement. We consider the web services are deployed in containers. Therefore, ``web service'' is used in the content instead of ``container''.
The result covers the evaluation of the proposed algorithm along with analysis, and concluding remarks and discussion of the future work (Section \ref{sec:con}). 

We first introduce the related model, then follow with the evaluation and analysis of results. In the end, we summarize the findings and the plan for future work.
% \LaTeX\ is a very good tool for producing well-structured documents 
% carefully. It is very bad tool for banging things together in a rush 
% and panic. 

% \section{Floats}
% One perennial problem with \LaTeX\ is its treatment of 
% \emph{floats}.  Suppose you have a figure or table which you want to 
% include in your document. Where should it go? Traditional typesetting 
% practice is to put these in some convenient place, such as the top or 
% bottom of the current or next page, or at the end of the section or 
% chapter.  \LaTeX\ adopts a similar strategy, and allows floats to 
% ``float'' away from where they were defined. You can give a hint 
% about where you want the figure, but \LaTeX\ may move it. Sometimes 
% this is fine but sometimes you may want to have more control and 
% insist that a float goes \emph{here}. Anselm Lingau's 
% \textsf{float} package gives you this flexibility. For example, the following figure is an example of a non-floating float:

% \begin{fig}[H]
% \begin{center}
% \begin{tabular}{l|lll}
% $\delta$ & $\mathit{a}$ & $\mathit{b}$ & $\Lambda$ \\ \hline 
% $S_{1}$  & $\{\}$       & $\{\}$      & $\{S_{2}, S_{5}, S_{10}\}$\\
% $S_{2}$  & $\{S_{3}\}$  & $\{\}$      & $\{\}$\\
% $S_{3}$  & $\{S_{4}\}$  & $\{\}$      & $\{\}$\\
% $S_{4}$  & $\{S_{3}\}$  & $\{\}$      & $\{\}$\\
% $S_{5}$  & $\{\}$       & $\{S_{6}\}$ & $\{\}$\\
% $S_{6}$  & $\{\}$       & $\{S_{7}\}$ & $\{S_{8}\}$\\
% $S_{7}$  & $\{S_{6}\}$  & $\{\}$      & $\{\}$\\
% $S_{8}$  & $\{S_{9}\}$  & $\{\}$      & $\{\}$\\
% $S_{9}$  & $\{\}$       & $\{S_{8}\}$ & $\{\}$\\
% $S_{10}$ & $\{S_{11}\}$ & $\{\}$      & $\{\}$\\
% $S_{11}$ & $\{\}$       & $\{S_{10}\}$& $\{\}$\\ 
% \end{tabular}
% \caption{The transition function of an NFA with $\Lambda$  transitions}

% \end{center}
% \end{fig}

% On the other hand, Figure \ref{Fig:two} is a floating float. 



% \begin{fig}[tbh]
% \begin{center}
% \begin{tabular}{l|ll}
% $\delta''$ & $\mathit{a}$ & $\mathit{b}$ \\ \hline 
% $T_{1}$  & $T_{2}$ & $T_{3}$\\ 
% $T_{2}$  & $T_{4}$ & $T_{5}$\\ 
% $T_{3}$  & $T_{6}$ & $T_{7}$\\ 
% $T_{4}$  & $T_{8}$ & \\
% $T_{5}$  & $T_{10}$ & \\
% $T_{6}$  &  & $T_{11}$\\ 
% $T_{7}$  & $T_{3}$ & \\
% $T_{8}$  & $T_{4}$ & \\
% $T_{10}$  &  & $T_{5}$\\ 
% $T_{11}$  & $T_{6}$ & 
% \end{tabular}
% \caption{The transition function of an FA to accept 
% the same language.}
% \label{Fig:two}
% \end{center}
% \end{fig}

% You can define different types of new floats, and you can have tables 
% of them in the contents pages.


% \section{URL's}
% Use \verb=\url= from the \textsf{url} package to typeset URL's. Just 
% using \verb+\texttt+ or \verb+\tt+ does not work:

% \begin{itemize}
% \item \verb+\texttt{http://www.mcs.vuw.ac.nz/~neil/}+
% \item \verb+\url{http://www.mcs.vuw.ac.nz/~neil/}+
% \end{itemize}

% Give:
% \begin{itemize}
% \item \texttt{http://www.mcs.vuw.ac.nz/~neil/}
% \item \url{http://www.mcs.vuw.ac.nz/~neil/}
% \end{itemize}
% If you use the \textsf{hyperref} package then you can produce PDF 
% files with clickable hyperlinks using \verb=\url=.

% \section{Graphics and \LaTeX}
% \LaTeX\ offers rather poor support for the inclusion of graphics. 
% There are lots of ways to include pictorial material in \LaTeX, all 
% of which are deficient in some way or other. Look at \cite{GRM97GC} for a 
% description of them. If your document does need to have pictures in it 
% it is worth thinking about what is needed \emph{before} you generate 
% the pictures.

% \section{The bibliography}

% You should build up your bibliography as you go along.  Trying to get 
% the details of the bibliography correct at the end of the project is 
% hard work. Make sure that you record all the relevant details. Beware 
% that material on the internet is likely to change very rapidly. If you 
% are going to include material which is only available on the internet, 
% then you should probably include in the reference the date on which 
% you obtained the document.

% \section{Run \LaTeX, run}

% \LaTeX\ builds up information about your document for the table of 
% contents, references and so on at each run. This means that, for 
% example, the table 
% of contents is really the table of contents of the previous 
% compilation. You may need to run \LaTeX\ two or three times to let it 
% catch up with itself. If you have cross references within your 
% bibliography (for example two papers from the same collection, such 
% as \cite{Dum93a,Dum93b}) you may need to run 
% BibTeX more than once. 

% It is also possible that the table of contents file has garbage in 
% it, and will prevent the document from being compiled. This may 
% happen if you have had to abort compilation, due to a bug in the 
% source file. If this is the case then removing the \texttt{.toc} file 
% will usually solve the problem. You will have to fix the original 
% bug, of course.


% \section{Find out more by\ldots}
% You can find out more by:
% \begin{itemize}
% \item reading any one of a number of books, such as \cite{GMS94,Lam94}. The 
% VUW library has copies of these;
% \item visiting  the Comprehensive \TeX\ Archive Network (CTAN) at 
% \url{www.ctan.org};
% \item typing \texttt{latex} into Google.
% \end{itemize}

% It is \emph{highly unlikely} that you are the first person who ever 
% wanted to do what you want to do with \LaTeX. Therefore it is likely 
% that someone has already solved your problem: the real key to using  
% \LaTeX\ well is to make effective use of what other people have done.

% \section{Summary}
% In this chapter we explained some things about \LaTeX.



% Service Oriented Architecture (SOA) and Cloud computing have significantly reformed the software industry. SOA provides a decentralized application architecture which allows software composition and reuse in a large, global scale. Meanwhile, Cloud computing provides  a scalable, reliable, and flexible infrastructure to web services. 


% These advantages come from virtualization technology, which provides different level of isolation \cite{vm_technology}: workload-level and full virtualization. Container \cite{container} is one of the workload level virtualization which allows single-application workloads sharing operating system and physical resources such as CPU time and memory. Therefore, this technology improves the utilization of a single VM. The full virtualization is used to partition physical machine into a number VMs (VMs) which enables VM migration. 

% As the dramatic increase of web services and cloud facilities, the management of resources has become a critical issue. 
% In recent years, as the power bill has become the largest fraction of the operating cost of Cloud facility \cite{Energy_6}, to reduce power consumption has become a paramount concern for Cloud service providers.
% In order to achieve that, a common approach is to re-allocate web services to a minimum number of physical machines (PMs) \cite{Energy_7}. 
% Therefore, idle computing servers are turned down or put into save mode. This optimization process, often called \textit{consolidation} involves with two levels of delivery mode, Software as a service (SaaS) and Infrastructure as a service (IaaS). Because of the complexity, consolidation tasks for IaaS and SaaS are often considered as separated tasks with different objectives. 
% For SaaS, the challenges concentrate on satisfying the Service Level Agreements (SLAs) with unpredictable requests using a minimum amount of resources. Whereas, for IaaS, the challenges are the VM migrations and 
% energy conservation.  

% There are extensive algorithms proposed for SaaS and IaaS levels of resource allocation \cite{Mazumdar20174, dubois2015autonomic}.
% Ref. \cite{Service_2} proposes a heuristic algorithm for service consolidation in a set of servers with minimizing costs while avoiding the overload of server and satisfying end-to-end response time constraints. 

% Ref. \cite{Energy_8} proposes two algorithms for energy efficient scheduling of VMs in Cloud, including an exact VM allocation algorithm which is an extended Bin-Packing approach, and a migration algorithm based on integer linear programming. 

% However, as the two levels of resource allocation are interact with each other, we believe they cannot be separated. They should be considered as one global optimization with multi-objectives from the perspectives of both service providers and cloud providers. Therefore, in this paper, we first propose a model for solving \textit{service resource allocation in Cloud (initial placement)}. Secondly, we propose a NSGA-II-based multi-objective algorithm with specifically designed operators to solve the problem. The two objectives are:

% \begin{enumerate}
%   \item propose a model for solving IaaS and SaaS resource allocation together
%   \item propose a NSGA-II-based algorithm to solve initial placement.
% \end{enumerate}


% The rest of the paper is organized as follows. Section \ref{sec:back} discusses the traditional 
% approaches for IaaS and SaaS and the power model for VM allocation. It will also introduce
% related works of evolutionary multi-objective optimization techniques. Section \ref{sec:problem} describes the 
% definition of the initial placement problem. Section \ref{sec:method} introduces the representation and genetic operators for initial placement problem. Section \ref{sec:exp} illustrates the experiment design, results and discussions. Section \ref{sec:con} draws a conclusion and discusses the future work.

% Copyright information



\section{Related models}
% Ref. \cite{ga_saas} proposes a single-objective genetic algorithm to solve placement of service (SaaS) on physical machines. Their major contributions are three-fold. Firstly, they consider web services as a workflow and optimize the makespan of a workflow. Secondly, they design a representation to the problem. Thirdly, they do not only consider computing 
% nodes, but storage nodes as well.
We develop a bilevel model for allocating web services to VMs and VMs to physical machines (PMs). We consider two models: the workload model for describing the resource demand of web services and the power model for describing the relationship between resource utilization of web services and energy consumption of PMs.


\subsection{Workload model}
A workload model of web services defines the relationship between the resource demand and requests of a web service.
Xavier el al \cite{Service_1} develop a \textit{Resource-Allocation-Throughput (RAT)} model for web service allocation. The \textit{RAT model} mainly defines several important variables for an atomic service which represents a software component. Based on this model, firstly, an atomic service's throughput equals its coming rate if the resources of the allocated VM are not exhausted. Secondly, increasing the coming rate will also increase an atomic service's throughput until the allocated resource is exhausted. Thirdly, when the resource is exhausted, the throughput will not increase as request increasing. At this time, the VM reaches its capacity. 
In this work, we adopt the RAT model for web service and model the resource requirement of web services as the number of request times resources per request. 
% Beloglazov et al.  \cite{Energy_Service_2} propose two algorithms for VM allocation. The first one is a bin-packing algorithm, called Modified Best Fit decreasing (MBFD) which is used when a new VM allocation request arrives. The second algorithm, named Minimization of Migration, is used to adjust the current VMs’ allocation according to the CPU utilization of a physical machine. Their experiments have shown that these methods lead to a substantial reduction of energy consumption in Cloud data centers.

\subsection{Power Model}
A power model of PM defines the relationship between the resource utilization and the energy consumption of a PM.
Shekhar's research \cite{Energy_1} is one of the earliest in energy aware consolidation for cloud computing. They conduct experiments of independent applications running in physical machines. They explain that CPU utilization and disk utilization are the key factors affecting the energy consumption. They also find that only consolidating services into the minimum number of physical machines does not necessarily achieve energy saving, because the service performance degradation leads to a longer execution time, which increases the energy consumption. 

Bohra \cite{Energy_3} develops an energy model to profile the power of a VM. They monitor the sub-components of a VM which includes: CPU, cache, disk, and DRAM and propose a linear model (Eq~\ref{eq:vmeter}). 
Total power consumption is a linear combination of the power consumption of CPU, cache, DRAM and disk. The parameters
$\alpha$ and $\beta$ are determined based on the observations of machine running CPU and IO intensive jobs. 
\begin{equation}
\label{eq:vmeter}
  P_{(total)} = \alpha P_{\{CPU, cache\}} + \beta P_{\{DRAM, disk\}}
\end{equation}
Although this model can achieve an average of 93\% of accuracy, it is hard to be employed in solving initial placement problem, for the lack of data.

Beloglazov et al. \cite{Beloglazov:2012ji} propose a comprehensive energy model for energy-aware resource allocation problem (Eq~\ref{eq:anton}). $P_{max}$ is the maximum power consumption when a VM is fully utilized;
$k$ is the fraction of power consumed by the idle server (i.e. 70\%); and $u$ is the CPU utilization. This linear relationship between power consumption and CPU utilization is also observed by \cite{Energy_4, Energy_5}. 
\begin{equation}
\label{eq:anton}
  P(u) = k \cdot P_{max} + (1 - k) \cdot P_{max} \cdot u
\end{equation}

In this work, we adopt the power model proposed by Beloglazov.


% \subsection{Multi-objective Evolutionary Optimization}
% A multi-objective optimization problem consists of multiple objective functions to be optimized.
% A multi-objective optimization problem can be stated as follows:
% \begin{align}
% \min \ \ & \vec{f}(\vec{x}) = (f_1(\vec{x}), \dots, f_m(\vec{x})), \\
% s.t. \ \ & \vec{x} \in \Omega.
% \end{align}
% where $\Omega$ stands for the feasible region of $\vec{x}$.

% Multi-objective Evolutionary Optimization Algorithm (MOEA) are ideal for solving multi-objective optimization problems \cite{MOEA}, because MOEAs work with a population of solutions. With an emphasis on moving towards the true Pareto-optimal region, a MOEA algorithm can be used to find multiple Pareto-optimal solutions in one single simulation run \cite{ope}. Therefore, this project would employ MOEA approaches. This is also the first time to employ MOEAs technique for initial placement problem.

\section{Problem Description}
\label{sec:problem}
We consider the initial placement problem as a multi-objective problem with two potentially conflicting objectives: 
minimizing the overall cost of web services and minimizing the overall energy consumption of the used physical machines. 

To solve the initial placement problem, we model an atomic service as its request and requests' coming rate, also known as frequency. 


The request of an atomic service is modeled as two critical resources: 
CPU time $A = \{A_1, A_i, \dots, A_t \}$ and 
memory consumption $M = \{M_1, M_i, \dots, M_t \}$, 
for each request consumes a $A_i$ amount of CPU time 
and $M_i$ amount of memory. 
The coming rate is denoted as $R = \{R_1, R_i, \dots, R_t \}$. 
In real world scenario, the size and the number of a request are both variant 
which are unpredictable, therefore, this is one of the major challenges in Cloud resource allocation. 
In this paper, we use fixed coming rate extracted from a real world dataset to represent real world service requests. 
 
The cloud data center has a number of available physical machines which are modeled as
CPU time $PA = \{PA_1, PA_j, \dots, PA_p\}$ and memory
$PM = \{PM_1, PM_j, \dots, PM_p\}$. $PA_j$ denotes the CPU capacity of a physical machine 
and $PM_j$ denotes the size of memory. A physical machine can be partitioned or
 virtualized into a set of VMs; 
 each VM has its 
 CPU time $VA = \{VA_1, VA_n, \dots, VA_v\}$ and 
 memory $VM = \{VM_1, VM_n, \dots, VM_v\}$. 


The decision variable of service allocation is defined as $X^i_n$. $X^i_n$
is a binary value (e.g. 0 and 1) denoting whether a
service $i$ is allocated on a VM $n$.
The decision variable of VM allocation is defined as $Y^n_j$. $Y^n_j$
is also binary denoting whether a
VM $n$ is allocated on a physical machine $j$.


 In this work, we consider homogeneous physical machine which means physical machines have the same size of CPU time and memory. 
 The utilization of a CPU of a VM is denoted as $U = \{U_1, U_n, \dots, U_v\}$. 
 The utilization can be calculated by Eq.\ref{eq:util}.

\begin{equation}
\label{eq:util}
  U_n =
  \begin{cases} 
    \frac{\sum_{i = 1}^t R_i \cdot A_i \cdot X^i_n}{VA_n}, \text{If }  \frac{\sum_{i = 1}^t R_i \cdot A_i \cdot X^i_n}{VA_n} < 1 \\
    1   \quad\quad\quad\quad\quad\quad\quad ,\text{otherwise}
  \end{cases}
\end{equation}

The cost of a type of VM is denoted as $C = \{C_1, C_n \dots, C_v\}$. 



In order to satisfy the performance requirement, Service providers often define Service Level Agreements (SLAs) to ensure the service quality. In this work, we define throughput as a SLA measurement \cite{SLA_metric}. 
Throughput denotes the number of requests that a service could successfully process in a period of time. According to \textit{RAT} model, the throughput is equal to the number of requests when the allocated resource is sufficient. 
Therefore, if a VM reaches its utilization limitation, it means that the services have been allocated exceedingly.
Therefore, all services in that VM suffer from 
performance degradation.


Then we define two objective functions as the total energy consumption and the total cost of VMs:

\begin{equation}
\label{eq:energy}
\begin{aligned}
& {\text{minimize}}\\
& Energy = \sum\limits_{j=1}^p (k \cdot V_{max} + (1 - k) \cdot V_{max} \cdot \sum^v\limits_{n=1} U_n \cdot Y^n_j)\\
\end{aligned}
\end{equation}

\begin{equation}
\label{eq:cost}
\begin{aligned}
& & & & & & & Cost = \sum\limits_{j=1}^p\sum\limits_{n=1}^v C_n \cdot Y^n_j
\end{aligned}
\end{equation}

\subsubsection{Hard constraints}
We define hard constraints as the mandatory constraints.

A VM can be allocated on a physical machine if and 
only if the physical machine has enough available capacity on every resource.

\begin{equation} 
\label{eq:constraint}
\begin{aligned}
\sum\limits_{n=1}^v VM_n \cdot Y^n_j \leq PM_j\\
\sum\limits_{n=1}^v VA_n \cdot Y^n_j \leq PA_j
\end{aligned}
\end{equation}

\subsubsection{Soft constraint}
We define the soft constraint as the constraint that can be relaxed. 

A service can be allocated on a VM even if the 
VM does not have enough available capacity on every resource, but the allocated services will suffer from a quality degradation.
% For the soft constraint, we define that exceeding numbers of atomic services can still
%  be packed into a VM, 
%  but if the resource of a VM has been used up, all the applications 
%  in that VM will suffer from a quality degradation.
\begin{equation}
\sum\limits_{i=1}^t M_i \cdot R_i \cdot X_i^n  \leq VM_n
\end{equation}

\section{Methods}
\label{sec:method}
This section discusses the proposed algorithm including the framework of NSGA-II, the 
detailed problem representation, genetic operators including initialization, mutation and selection.
Lastly, we give the pseudo code in Section \ref{sec:pseudocode}.

Multi-objective Evolutionary Algorithms (MOEAs) are good at solving 
multi-objective problems. 
NSGA-II \cite{nsgaii} is a well-known MOEA that has been widely used in many real-world optimization problems. 
We also adopt NSGA-II to solve the initial placement problem. 
We first propose a representation and then present a NSGA-II based algorithm with novel genetic operators.

\subsection{Chromosome Representation}
\label{sec:representation}
Initial placement is a bilevel bin-packing problem. In the first level, 
bins represent physical machines and items represent VMs. Whereas, in the second level, a VM acts like a bin and web services are items. 
Therefore, we design the representation in two hierarchies, VM level and physical machine level. 

\begin{figure*}
\centering
  \includegraphics[width=0.8\textwidth]{pics/preliminary/cec.png}
  \caption{An example of chromosome representation}
  \label{fig:rep}
\end{figure*}

Figure \ref{fig:rep} shows an example individual which contains seven service allocations. Each allocation of a service is represented as a pair where the index of each pair represents the number of web service. The first number indicates the type of VM that the service is allocated in. The second number denotes the number of VM.  For example, in Figure \ref{fig:rep}, service \#1 and service \#2 are both allocated in the VM \#1 while service \#1 and service \#5 are allocated to different VMs sharing the same type.
The first hierarchy shows the VM in which a service is allocated by defining VM type and number. 
Note that, the VM type and number are correlated once they are initialized. 
With this feature, the search procedure is narrowed down in the range of existing VMs which largely shrinks the search space.
The second hierarchy shows the relationship between a physical machine and its VMs, which are implicit. The physical machine is dynamically determined according to the VMs allocated on it. 
For example, in Figure ~\ref{fig:rep}, the VMs are 
sequentially packed into physical machines. The boundaries of PMs are calculated by adding up the resources of VMs until one of the resources researches the capacity of a PM. At the moment, no more VMs can be packed into the PM, then the boundary is determined.
% That is, we employ a simple heuristic method, 
% where the boundary of a physical machine is dynamically calculated. by adding up the VMs resources sequentially until a physical machine is full. 
The reason we designed this heuristic is because a physical machine is always fully used before launching another. Therefore, VM consolidation is inherently achieved.

Clearly, specifically designed operators are needed to manipulate chromosomes. Therefore, based on this representation, we further developed initialization, mutation, constraint handling and selection method.

\subsection{Initialization}
\begin{algorithm}[!htb]
 \caption{Initialization}
 \footnotesize
 \textbf{Inputs:} \\
  VM CPU Time $VA$ and memory $VM$, \\
  Service CPU Time $A$ and memory $M$ \\
  consolidation factor c \\
 \textbf{Outputs:}
  A population of allocation of services

 \begin{algorithmic}[1]
  \FOR{ Each service $t$ }
    \STATE{ Find its most suitable VM Type}
    \STATE{ Randomly generate a VM type $vmType$ which is equal
            or better than its most suitable type}
    \IF { There are existing VMs with $vmType$}
      \STATE randomly generate a number $u$
      \IF{ $u <$ consolidation factor }
      \STATE randomly choose one existing VM with $vmType$ to allocate
      \ELSE
      \STATE launch a new VM with $vmType$
      \ENDIF
    \ELSE
      \STATE Create a new VM with its most suitable VM type
    \ENDIF

  \ENDFOR
 \end{algorithmic}
 \label{alg:init}
\end{algorithm}


The initialization (see Alg~\ref{alg:init}) is designed to generate a diverse population.
In the first step, for each service, it is able to find the most suitable VM type which is just capable of running the service based on its resource requirements. 
In the second step, based on the suitable VM type, a stronger type is randomly generated. If there exists a VM with that type, the service is either deployed in the 
existing VM or launch a new VM. We design a consolidation factor $c$ which is a real number manually selected from 0 to 1 to control this selection. If a
random number $u$ is smaller than $c$, the service is consolidated in an existing VM.

This design could adjust the consolidation, therefore, controls the utilization of VM.

\subsection{Mutation}
\begin{figure}
\centering
  \includegraphics[width=0.7\textwidth]{pics/preliminary/hollow.png}
  \caption{An example mutation without insertion that causes a lower resource utilization}
  \label{fig:hollow}
\end{figure}

The design principle for mutation operator is to enable individuals to explore the entire feasible search space.
Therefore, a good mutation operator has two significant features, the exploration ability and the its ability to keep an individual within the feasible regions. In order to achieve these two goals, firstly, we generate a random VM type which has a greater capacity than the service needs. It ensures the feasible of solutions as well as exploration capability. Then, we consider whether a service is consolidated with the consolidation factor $c$. 

The consolidation is conducted with a roulette wheel method which assigns fitness value to each VM according to the reciprocal of its current utilization. 
The higher the utilization, the lower the fitness value
it is assigned. 
Therefore, a lower utilization VM has a greater probability to be chosen. 
At last, if a new VM is launched, it will not be placed at the end of VM lists. Instead, it will be placed at a random position among the VMs. The reason is illustrated in Figure \ref{fig:hollow}. In the example, VM \#2 is mutated into a new type and be placed at the end of the VM list. However, because of the size of VM \#3 is too large for PM \#0, the hollow in PM \#0 will never be filled. This problem can be solved with the random insertion method.
\begin{algorithm}[!htb]
 \caption{Mutation}
 \footnotesize
 \textbf{Inputs:} \\
  An individual
  VM CPU Time $VA$ and memory $VM$, \\
  Service CPU Time $A$ and memory $M$ \\
  consolidation factor c \\
 \textbf{Outputs:}
  A mutated individual

 \begin{algorithmic}[1]
  \FOR{ Each service}
    \STATE Randomly generate a number $u$
    \IF { $u <$ mutation rate}
      \STATE find the most suitable VM Type for this service
      \STATE Randomly generate a number $k$
      \IF{ $k <$ consolidation factor }
        \STATE calculate the utilization of used VMs
        \STATE assign each VM with a fitness value of 1 / utilization and generate a roulette wheel
            according to their fitness values
        \STATE Randomly generate a number $p$, select the VM according to $p$
        \STATE Allocate the service
      \ELSE
        \STATE launch a new VM with the most suitable VM Type
        \STATE insert the new VM in a randomly choose position
      \ENDIF
    \ENDIF
  \ENDFOR
 \end{algorithmic}
 \label{alg:mutation}
\end{algorithm}

\subsection{Violation control method}
A modified violation ranking is proposed to deal with the soft constraint, for the hard constraint is automatically eliminated by the chromosome representation.
We define a violation number as the number of services which are allocated in the degraded VMs. 
That is, if there are excessive services allocated in a VM, then all the services are suffered from a degraded in performance. 
The violation number is used in the selection procedure, 
where the individuals with less violations are always preferred.

\subsection{Selection}
Our design uses the binary tournament selection with a constrained-domination principle. A constrained-domination principle is defined as following. A solution $I$ is considered constraint-dominate a solution $J$, if any of the following condition is true:
\begin{enumerate}
  \item The solution $I$ is feasible, solution $J$ is not,
  \item Both solutions  $I$ and $J$ are infeasible, $I$ has smaller overall
violations,
  \item Both solutions $I$ and $J$ are feasible, solution $I$ dominates solution $J$.
\end{enumerate}

An individual with no or less violation is always selected. The ranking method has been proved effectiveness of producing feasible solutions in the original NSGA-II paper \cite{nsgaii}.

\subsection{Fitness Function}
The cost fitness (Eq.\ref{eq:cost}) is determined by the type of VMs at which web services are allocated. 
The energy fitness is shown in Eq.\ref{eq:energy}, the utilizations (Eq.\ref{eq:util}) of VM are firstly converted into the utilizations of PM according to the proportion of VMs’ and PM’s CPU capacity.

\subsection{Algorithm}
\label{sec:pseudocode}
The main difference between our approach and the original NSGA-II is that our approach has no crossover operator.

That is, a random switch of chromosome would completely destroy the order of VMs, 
hence, no useful information will be preserved. 
Therefore, we only apply mutation as the exploration method. Then, the algorithm becomes a parallel optimization without much interaction between its offspring, which is often addressed as Evolutionary Strategy \cite{evo_str}.
\begin{algorithm}[!htb]
 \caption{NSGA-II for initial placement}
 \footnotesize
 \textbf{Inputs:} \\
  VM CPU Time $VA$ and memory $VM$, \\
  PM CPU Time $PA$ and memory $PM$, \\
  Service CPU Time $A$ and memory $M$ \\
  consolidation factor c \\
 \textbf{Outputs:}
  A Non-dominated Set of solutions

 \begin{algorithmic}[1]
  \STATE Initialize a population $P$
  \WHILE{ Termination Condition is not meet}
    \FOR{ Each individual }
      \STATE Evaluate the fitness values
      \STATE Calculate the violation
    \ENDFOR

    \STATE non-Dominated Sorting of $P$
    \STATE calculate crowding distance
    \WHILE{ child number is less than population size }
      \STATE Selection
      \STATE Mutation
      \STATE add the child in a new population U
    \ENDWHILE
    \STATE Combine $P$ and $U$ \COMMENT{ for elitism}
    \STATE Evaluate the combined $P$ and $U$
    \STATE Non-dominated sorting and crowding distance for combined population
    \STATE Include the top popSize ranking individuals to the next generation

  \ENDWHILE
 \end{algorithmic}
 \label{alg:NSGAII}
\end{algorithm}

Next, we will validate the proposed algorithm through experiment.

\section{Experiment}
\label{sec:exp}
\subsection{Dataset and Instance Design}
This project is based on both real-world datasets \textit{WS-Dream} \cite{Service_dataset} and simulated datasets \cite{Energy_9}. 
The \textit{WS-Dream} contains web service related datasets including network latency and service frequency (request coming rate). In this project, we mainly use the service frequency matrix. For the cost model, we only consider the rental of VMs with fixed fees (monthly rent). The configurations of VMs are shown in Table~\ref{tab:vm}, the CPU time and memory were selected manually and cost were selected proportional to their CPU capacity. The maximum PM's CPU and memory are set to 3000 and 8000 respectively. The energy consumption is set to 220W according to \cite{Energy_9}.

We designed six instances shown in Table \ref{tab:problem}, listed with increasing size and difficulty, which are used as representative samples of initial placement problem.

\begin{table}[]
\centering
\caption{Instance Settings}
\label{tab:problem}
\begin{tabular}{ccccccc}
\hline 

Problem           & 1  & 2  & 3  & 4  & 5   & 6    \\
Number of services & 20 & 40 & 60 & 80 & 100 & 200 \\ \hline
\end{tabular}
\end{table}

\begin{table}[]
\centering
\caption{VM configurations}
\label{tab:vm}
\begin{tabular}{@{}cccc@{}}
\toprule
VM Type & CPU Time & Memory & Cost \\ \midrule
1       & 250      & 500    & 25 \\
2       & 500      & 1000   & 50 \\
3       & 1500     & 2500   & 150\\
4       & 3000     & 4000   & 300\\ \bottomrule
\end{tabular}
\end{table}
\begin{flushleft}Selection Method with violation Control vs. without violation control\end{flushleft}
\begin{figure}
   \centering
   \begin{subfigure}[b]{0.45\textwidth} \includegraphics[width=\textwidth]{pics/preliminary/without/testCase1_.png}
   \caption{Instance 1}
   \label{fig:a}
   \end{subfigure}
   \begin{subfigure}[b]{0.45\textwidth} \includegraphics[width=\textwidth]{pics/preliminary/without/testCase2_.png}
   \caption{Instance 2}
   \label{fig:b}
   \end{subfigure}
   \begin{subfigure}[b]{0.45\textwidth}\includegraphics[width=\textwidth]{pics/preliminary/without/testCase3_.png}
   \caption{Instance 3}
   \label{fig:c}
   \end{subfigure}
   \begin{subfigure}[b]{0.45\textwidth}\includegraphics[width=\textwidth]{pics/preliminary/without/testCase4_.png}
   \caption{Instance 4}
   \label{fig:d}
   \end{subfigure}
   \begin{subfigure}[b]{0.45\textwidth}\includegraphics[width=\textwidth]{pics/preliminary/without/testCase5_.png}
   \caption{Instance 5}
   \label{fig:e}
   \end{subfigure}
   \begin{subfigure}[b]{0.45\textwidth}\includegraphics[width=\textwidth]{pics/preliminary/without/testCase6_.png}
   \caption{Instance 6}
   \label{fig:f}
   \end{subfigure}
   \caption{Non-dominated solutions evolve along with the generation}
   \label{fig:evolve}
\end{figure}

We conducted two comparison experiments. For the first experiment, we make a comparison between NSGA-II with violation control and NSGA-II without violation control. 
In second experiment, two mutation operators are compared. The first is the roulette wheel mutation, the second is the mutation with greedy algorithm. The mutation with greedy algorithm is a variant of roulette wheel mutation. The only difference is that instead of selecting a VM to consolidate with fitness values, it
always selects the VM with the lowest CPU utilization. 
Therefore, it is a greedy method embedded in the mutation.


The experiments were conducted on a personal laptop with 2.3GHz CPU and 8.0 GB
RAM. For each approach, 30 independent runs are performed for each problem with
constant population size 100. The maximum number of iteration is 200. $k$ equals 0.7. 
We set mutation rate and consolidation factor to 0.9 and 0.01.


\subsection{Results}
\begin{figure}
   \centering
   \begin{subfigure}[b]{0.45\textwidth}\includegraphics[width=\textwidth]{pics/preliminary/1/evolve.png}
   \caption{Instance 1}
   \label{fig:a}
   \end{subfigure}
   \begin{subfigure}[b]{0.45\textwidth}\includegraphics[width=\textwidth]{pics/preliminary/2/evolve.png}
   \caption{Instance 2}
   \label{fig:b}
   \end{subfigure}
   \begin{subfigure}[b]{0.45\textwidth}\includegraphics[width=\textwidth]{pics/preliminary/3/evolve.png}
   \caption{Instance 3}
   \label{fig:c}
   \end{subfigure}
   \begin{subfigure}[b]{0.45\textwidth}\includegraphics[width=\textwidth]{pics/preliminary/4/evolve.png}
   \caption{Instance 4}
   \label{fig:d}
   \end{subfigure}
   \begin{subfigure}[b]{0.45\textwidth}\includegraphics[width=\textwidth]{pics/preliminary/5/evolve.png}
   \caption{Instance 5}
   \label{fig:e}
   \end{subfigure}
     \begin{subfigure}[b]{0.45\textwidth}\includegraphics[width=\textwidth]{pics/preliminary/6/evolve.png}
   \caption{Instance 6}
   \label{fig:f}
   \end{subfigure}
   \caption{non-dominated solutions comparison between selection with violation control and without violation control}
   \label{fig:dynamicFunctions}
\end{figure}


\begin{table*}[!ht]
\centering
\caption{Comparison between two Mutation methods}
\label{tab:mutations}
\begin{tabular}{@{}cllll@{}}
\toprule
Instance & \multicolumn{2}{c}{roulette wheel mutation} & \multicolumn{2}{c}{Greedy mutation}   \\ \midrule
        & cost fitness         & energy fitness       & cost fitness      & energy fitness    \\
1       & 2664.6 $\pm$ 66.4      & 1652.42 $\pm$ 18.2     & 2661.7 $\pm$ 56.9   & 1653.2 $\pm$ 18.2   \\
2       & 6501.1 $\pm$ 130.2     & 4614.0 $\pm$ 110.7     & 6495.37 $\pm$ 110.7 & 4132.5 $\pm$ 80.4   \\
3       & 8939.2 $\pm$ 118.5     & 6140.7 $\pm$ 204.0     & 9020.5 $\pm$ 204.0  & 5739.6 $\pm$ 148.6  \\
4       & 11633.7 $\pm$ 301.1    & 9301.9 $\pm$ 254.0     & 12900.6 $\pm$ 243.0 & 9376.3 $\pm$ 120.9  \\
5       & 14102.0 $\pm$ 231.7    & 10164.8 $\pm$ 238.9    & 14789.2 $\pm$ 238.8 & 9876.3 $\pm$ 120.9  \\
6       & 27194.3 $\pm$ 243.0    & 19914.4 $\pm$ 307.5    & 27654.2 $\pm$ 307.5 & 19187.1 $\pm$ 176.6 \\ \bottomrule
\end{tabular}
\end{table*}

We conducted the experiment for 30 runs. We first obtained an average non-dominated set over 30 runs by collecting the results from a specific generation from all 30 runs. We then applied a non-dominated sorting over them.

Firstly, we showed the non-dominated solutions evolve along with the evolution process in Figure \ref{fig:evolve}.
These results came from selection method without violation control. 
As it illustrated, different colors represent different generations from 0th to 200th. 
For instance 1, because the problem size is small, the algorithm converged before 100 generations. Therefore, the non-dominated set from the 100th and 150th generations are overlapping with results from the 200th generation. For instance 2 and instance 3, they clearly show the improvement of fitness values. For instance 4 onwards, the algorithm can only obtain a few solutions as the problem size is large, thus, it is difficult to find solutions.

Then, the non-dominated sets of the last generation from two selection methods are compared in Figure \ref{fig:dynamicFunctions}. There are much fewer results are obtained from the violation control method throughout all cases. For the first three instances, the non-dominated set from the violation control method has similar quality
as the no violation control method. From instance 4 onwards, the results from selection with violation control are much worse in terms of fitness values. However, most of the results from non-violation control selection have a high violation rate. That is, the method without violation control is stuck in the infeasible regions and provide high-violation rate solutions. 

From figure \ref{fig:violations}, we can observe the violation rate between two methods: with and without violation control.
It proves violation control has a great ability to prevent the individual from searching the infeasible region. On the other hand, without violation control, although, the algorithm can provide more solutions with better fitness values, most of them have a high violation rate over 10\% which are not very useful in reality.

As we mentioned in previous section, the mutation rate and consolidation factor are set differently for the two methods. For the method with violation control, the mutation rate is set to 0.9 and the consolidation factor $c$ is set to 0.01. This is because the feasible region is narrow and scattered. In order to avoid stuck in the local optima, a large mutation rate can help escaping from local optima. For the factor $c$, a larger percentage would easily lead the algorithm to infeasible regions. Therefore, the factor $c$ is set to a small number.





\begin{figure}
\centering
  \includegraphics[width=.7\textwidth]{pics/preliminary/violations.png}
  \caption{violation rate comparison between selection with violation control and without violation control}
  \label{fig:violations}
\end{figure}



\begin{flushleft}\textbf{Mutation with roulette wheel vs. Mutation with greedy algorithm}\end{flushleft}
Table \ref{tab:mutations} shows the fitness value comparison between mutation methods. According
to statistics significant test, there is little difference between methods. The possible reason
is the consolidation factor is set to 0.01. In each mutation iteration, there is only 1\% probability
that a service will be consolidated in an existed VM, therefore, the influence between
different consolidation strategies is trivial.





\section{Findings and Future work}
\label{sec:con}

This work investigated the bilevel energy model for the initial placement of containers and VMs. We discussed two sub models workload model and power model. We also established a multi-objective formulation of the bilevel problem with two objectives: minimizing the cost of used VMs and minimizing the energy consumption. In order to optimize the problem, we propose a NSGA-II based algorithm with specific designed representation. The representation is embedded with a heuristic to quickly locate feasible solutions. This work designed genetic operators such as population initialization, mutation, selection for generating valid solutions and handling the constraints. We compared the results with different variances of the algorithm. The results provided the evidence that our proposed energy model can be used in container-based server consolidation for the first sub objective in objective one. Furthermore, our NSGA-II based approach and proposed representation can quickly find feasible solutions. This partially addresses the second sub objective in objective one. However, current work does not consider the balance between CPU and memory and the overheads of VM. Therefore, in the next step, we will investigate these two factors and add them into the bilevel energy model. In addition, we will propose a new EC-based approach to solve bilevel optimization problem.

% In recent years, virtualization technology has evolved to allow finer granularity resource management.
% A recent development of Container technique \cite{Soltesz:2007cu} has driven the attention of both industrial and academia.
% Container is an operating system level of virtualization which means multiple containers can be installed in a same operating system (see Figure \ref{fig:comparison} right-hand side). Each container provides an isolated environment for an application. In short, a VM is partitioned into smaller manageable units.}
\chapter{Proposed Contributions and Project Plan}\label{C:con}

This thesis will contribute to the field of Cloud Computing by proposing novel solutions for improving energy efficiency in container-based clouds. It will also contribute to the field of Evolutionary Computation (EC) by proposing new representations and genetic operators for solving bilevel optimization problems. The proposed contributions of this project are listed below:
 
\begin{enumerate}
	\item Two new bilevel models for predicting energy consumption in container-based clouds: \textbf{first}, a new bilevel energy model for initial placement of application; \textbf{second}, a new bilevel energy and migration model for periodic placement of application with the consideration of three types of workload. The above two models will address the relationship between energy consumption and five factors such as locations of container, tyeps of VM, and locations of VM.  
	These two bilevel models can be used in optimizing energy consumption and minimizing migration cost in container-based clouds.

	% A new bilevel model for the joint placement of container and VM problem.  The bilevel model will address the relationship between containers, VMs, PMs, and energy consumption. The bilevel model can be used in optimizing the energy consumption in a container-based cloud data center.
	\item A new EC-based bilevel single-objective optimization algorithm for solving the initial placement of application problem based on previously proposed bilevel energy model. The new algorithm contributes to both Cloud computing and EC. From the perspective of cloud computing, the algorithm produces a resource allocation with better energy efficiency in container-based clouds. From the perspective of EC, the algorithm develops new representations, search mechanisms for bilevel optimization problems. In addition, the work also develops a hybrid approach that combines EC with clustering techniques.

	\item A new EC-based bilevel multi-objective algorithm with Pareto front approach for solving the periodic placement of application. This algorithm will consider three types of predictable workload: static, periodic, and linear continuously changing workloads. From the perspective of cloud computing, the algorithm is expected to achieve better energy efficiency and migration cost than existing VM-based approaches with consideration of three types of workload. From the perspective of EC, the algorithm proposes specific representations and genetic operators for three types of workload. 

	\item A new genetic programming hyper-heuristic (GP-HH) approach for solving single-objective dynamic placement of application with three types of predictable workload and two types of unpredictable workload in \textbf{VM-based clouds}. The proposed GP-HH focuses on solving the single-level of placement: VM to PM. From the perspective of cloud computing,
	the GP-HH can be used in VM-based clouds and it is expected to fast allocate VMs to PMs and achieve a near-optimal solution in energy consumption.
	From perspective of EC,  the GP-HH is the first to solve resource allocation problem in container-based clouds. The algorithm will propose new primitive set, search mechanisms.

	\item A new cooperative GP-HH approach for solving single-objective dynamic placement of application with five types of workload in \textbf{container-based clouds}. This work is based on the previously proposed GP-HH approach.
	From perspective of cloud computing, the cooperative GP-HH can quickly allocate containers and VMs to achieve near optimal solutions in terms of energy consumption.
	From perspective of EC, the new cooperative GP-HH method will solve bilevel dynamic problem.
\end{enumerate}

\section{Overview of Project Plan}
Six overall phases have been defined in the initial research plan for this PhD project, as
shown in Table \ref{tab:plan}. The first phase, which comprises reviewing the relevant literature, investigating both VM-based and container-based server consolidation algorithms, and producing the proposal, has
been completed. The second phase, which corresponds to the first objective of the thesis, is
currently in progress and is expected to be finished on time, thus allowing the remaining
phases to also be carried out as planned.


\begin{table}[]
\centering
\caption{Phases of project plan}
\label{tab:plan}
\scalebox{0.85}{
\begin{tabular}{|c|l|l|}
\hline
\multicolumn{1}{|l|}{Phase} & Task                                                                                                                                                                                 & Duration (Months) \\ \hline
1                           & \begin{tabular}[c]{@{}l@{}}Reviewing literature, overall design, selection of datasets\\ and writing the proposal\end{tabular}                                                       & 12 (Complete)     \\
2                           & \begin{tabular}[c]{@{}l@{}}Develop a single-objective EC-based approach for the joint\\ allocation of containers and VMs\end{tabular}                                                & 7                 \\
3                           & \begin{tabular}[c]{@{}l@{}}Develop multi-objective EC-based approaches for container-based \\ cloud in periodic placement of application with considering various types of workload\end{tabular} & 7                 \\
4                           & \begin{tabular}[c]{@{}l@{}}Develop a cooperative Genetic programming based \\ hyper-heuristic approach for dynamic placement.\end{tabular}                                         & 7                 \\
5                           & Writing the thesis                                                                                                                                                                   & 6                 \\ \hline
\end{tabular}
}
\end{table}



\section{Project Timeline}
The phases included in the plan above are estimated to be completed following the timeline
shown in Table \ref{table-timeline}. The timeline will serve as a guide throughout this project. Note that part of the first phase has already been done. Therefore the timeline only shows the estimated remaining time for full completion.


\begin{table}
\protect\caption{Time Line}
\label{table-timeline}
\footnotesize
\begin{center}
\scalebox{0.8}{
\begin{tabular}{|l||cccc|cccc|cccc|}

\hline
  \textbf{Task}      & \multicolumn{4}{c}{ }&  \multicolumn{4}{c}{Months}&\multicolumn{4}{c|}{ }  \\ \hline
&2&4&6&8&10&12&13&16&18&20&22&24 \\ \hline
Literature Review and Updating &x&x&x&x&x&x&x&x&x&x&x&x \\ [1mm]

\cellcolor{LightCyan}Develop a new bilevel energy model&x&&&&&&&&&&& \\ 

Propose a new EC based bilevel optimization approach to solve the &x&x&&&&&&&&&& \\ 
initial placement of application & &&&& & & & & & & &  \\

\cellcolor{LightCyan} Improve the scalability of the proposed & & x&x & & & & & & & & & \\
\cellcolor{LightCyan}  EC-based approach up to one thousand applications & & & & & & & & & & & &   \\ [1mm]
Propose a multi-objective bilevel model for periodic placement  & & & &x&& & & & & & & \\ 
with consideration of static workload. & & & &&& & & & & & & \\ [1mm]

\cellcolor{LightCyan} Propose a multi-objective bilevel EC-based algorithm for & & & & & x&x&& & & & & \\ 
\cellcolor{LightCyan}  periodic placement of application with Pareto front approach.  & & & & & & & & & & & &   \\ [1mm]

Extend the bilevel multi-objective model for adapting  &&&&&&x&&&&&&\\ [1mm]
three types of predictable workload.  &&&&&&&&&&&&\\ [1mm]

\cellcolor{LightCyan}Extend the previous EC-based multi-objective algorithm &&&&&&&x&&&&&\\ [1mm]
\cellcolor{LightCyan}to adapt to three types of workload. &&&&&&&&&&&&\\ [1mm]

Develop a GP-based hyper-heuristic (GP-HH) algorithm for  &&&&&&&&x&x&&& \\ [1mm]
automatically generating dispatching rules for the single-level placement. & & & & & &&& & & & &  \\[1mm]

\cellcolor{LightCyan} Conduct feature extraction on the three predictable workloads and two  &&&&&&&&&x&&&\\ 
\cellcolor{LightCyan} unpredictable workloads to construct a new primitive set. &&&&&&&&&&&& \\ [1mm]

Develop a cooperative GP-HH approach for automatically   &&&&&&&&&x&x&& \\ [1mm]
generating dispatching rules for placing both containers and VMs. &&&&&&&&&&&&\\ [1mm]
\cellcolor{LightCyan}  Writing the first draft of the thesis &&&&&&&&&&x&x&  \\ \hline
 Editing the final draft   &&&&&&&&&&x&x&x \\ \hline

\end{tabular}
}
\end{center}
\end{table}


\section{Thesis Outline}
The completed thesis will be organized into the following chapters:
\begin{itemize}
	\item \textit{Chapter 1: Introduction} \\
	This chapter will introduce the thesis, providing a problem statement and motivations, defining research goals and objectives, and outlining the structure of the final thesis.
	\item \textit{Chapter 2: Literature Review} \\
	The literature review will illustrate the fundamental background of Cloud computing, resource management, and server consolidation. It will examine the existing work on VM-based and container-based server consolidation and discuss concepts in this field in order to provide readers with the necessary background. Multiple sections will consider issues such as initial placement of application, periodic placement, and dynamic placement of application. The focus of this review is on investigating server consolidation techniques.
	\item \textit{Chapter 3: Develop EC-based approaches for the single objective joint placement of containers and VMs for initial placement of application.} \\
	This chapter will establish a new bilevel energy model for the joint placement of container and VM. Based on this model, this chapter will introduce a new EC-based bilevel algorithm combined with clustering technique and heuristics to solve the initial placement of application. 
	\item \textit{Chapter 4: Develop multi-objective EC-based approaches  for periodic placement of application} \\
	This chapter proposes a new bilevel energy and migration model based on a previously proposed energy model with three types of workload.  This chapter will also propose new EC-based approaches for bilevel multi-objective periodic placement, considering three types of workload. It is then followed by algorithm performance evaluation that contains an experiment design, setting, results and analysis.
	\item \textit{Chpater 5: Develop a  single-objective cooperative Genetic Programming hyper-heuristic (GP-HH) approach for automatically generating dispatching rules for dynamic placement of application} \\
	This chapter focuses on providing a Genetic Programming-based hybrid heuristic approach to automatically generate dispatching rules to a dynamic consolidation problem. This chapter will propose two algorithms -- a GP-HH for single-level of placement: VM-PM and a cooperative GP-HH for bilevel placement: container-VM and VM-PM.
	\item \textit{Chapter 7: Conclusions and Future Work}
	In this chapter, conclusions will be drawn from the analysis and experiments conducted in the different phases of this research, and the main findings for each phase of them will be summarized. Additionally, future research directions will be discussed.

\end{itemize}


\section{Resources Required}
\subsection{Computing Resources}
An experimental approach will be adopted in this research, entailing the execution of experiments that are likely to be computationally expensive. The ECS Grid computing facilities
can be used to complete these experiments within reasonable time frames, thus meeting this requirement.
\subsection{Library Resources}
The majority of the material relevant to this research can be found on-line, using the university electronic resources. Other works may either be acquired at the university library, or
by soliciting assistance from the Subject Librarian for the fields of engineering and computer science.
\subsection{Conference Travel Grants}
Publications to relevant venues in this field are expected throughout this project. Therefore travel grants from the university are required for key conferences.


%%%%%%%%%%%%%%%%%%%%%%%%%%%%%%%%%%%%%%%%%%%%%%%%%%%%%%%

\backmatter

%%%%%%%%%%%%%%%%%%%%%%%%%%%%%%%%%%%%%%%%%%%%%%%%%%%%%%%


%\bibliographystyle{ieeetr}
\bibliographystyle{acm}
\bibliography{sample}


\end{document}
