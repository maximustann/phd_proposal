\section{VM-based Static Consolidation Methods}

So far in the industry, most Cloud data center is based on virtual machine technology. Therefore, VM-based resource management is the mainstream in both industry and academia. 

As previous mentioned, server consolidation is one of the technique to reduce the power consumption. Various techniques have been proposed in this field, these techniques can be roughly grouped into static and dynamic approaches.

Static approaches adjust the allocation of VMs in a periodical fashion (e.g. weekly or monthly). 

Dynamic appproaches are conducted by a runtime placement manager to migrate VMs 
automatically in response to workload variations.




% Static initialization, is also frequently referred to initial placement problem \cite{Jennings:2015ht}. Whenever a request for provisioining of applications by one or more Cloud users. The resource management system schedules the applications into a set of PMs. Currently, most state-of-the-art research focus on VM-based placement, in this case, applications are installed in VMs. Therefore, ``application placement'' and ``VM placement'' are used interchangable in the literature. 

% In energy-aware resource management, the initialization has the objective of minimizing the used PMs. In literature, the static initialization problem is often modeled as the vector bin packing problem. Each application represents an item and PMs represents bins.

% A d-dimensional Vector Bin Packing Problem ($VBP_d$), give a set of items $I^1, I^2, \dots, I^n$ where each item has $d$ dimension of resources represented in real number $I^i \in R^d$. A valid solution is packing $I$ into bins $B^1, B^2, \dots, B^k$. For each bin and each dimension, the sum of resources can not exceed the capacity of bin. The goal of Vector Bin Packing problem is to find a valid solution with minimum number of bins. $VBP_d$ is an NP-hard problem.

% Because of its NP-hard nature, several researchers propose well-known bin-packing heuristics
% First Fit (FF), First Fit Decreasing (FFD), Best Fit (BF) and etc. These algorithms have constant-factor approximation to bin-packing problems. However, VM placement is more complex than bin-packing problem \cite{Mann:2015ua}, 

% Panigraph et al \cite{Panigrahy:2011wk} study variants of First Fit Decreasing (FFD) algorithms and inspired by bad isntances for FFD-type algorithms, they propose a geometric heuristics which outperform FFD-based heuristics in most of cases.

\section{Container initilization}
Initialization is one of the major step in resource management. It can be considered as a static problem \cite{Jennings:2015ht} or dynamic problem \cite{Beloglazov:2012bw}. 
As we discussed in the previous section, a dynamic allocation normally cannot gives a global optimized solution of a batch of tasks. Therefore, in the context of maximizing the energy efficiency of a data center, we category initialization into a static optimization approach.

% Previous research mainly use heuristic to solve this problem. 
Mesos \cite{Hindman:2011ux} is a platform for sharing commodity clusters between cluster computing frameworks such as Hadoop and MPI. It has a two-level of resource allocation architecture where a master node and several slave node. A master node only decides how many resources to offer to each framework (slave) based on fair sharing policy \cite{Ghodsi:2011vm}. Each slave node belongs to a cluster framework and it makes the decision of which resources to use. Each framework has to define its allocation policy. Mesos focus on sharing resources across multiple frameworks and has a better scalability since it deligates the application placement to decentralised slave nodes. The main problem for Mesos is that it does not consider energy consumption for Cloud providers with user-defined allocation policy.

Piraghaj et al \cite{Piraghaj:2016bw} propose an architecture for container-based resource management. Their allocation approach is policy-based, where it allocates each VM with containers until its estimated utilization above 90\%. This heuristic is similar to First Fit, it is fast but does not guarantee global optimal.
