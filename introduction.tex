\chapter{Introduction}\label{C:intro}
\section{Problem Statement}

The advent of Cloud computing has completely reformed the software industry \cite{Buyya:2009wt}. 
In Cloud customers' perspective, applications deployed on a Cloud can 
have unlimited scalability without upfront investment. Beyond that, 
Cloud offers a \emph{pay-as-you-go} policy which allows Cloud customers to 
pay the minimum rental of resources on a basis.
These two advantages make Cloud computing an attractive option.
% particularly for Web-based appliacations which have unpredictable workloads.
Virtualization \cite{Uhlig:2005ub} is the core technology that not only enables 
the elastic management of Cloud resource but also can be used to improve the utilization and reduce 
energy consumption.
It maps a physical machine's system resource - including processors, memory, and 
other devices - into isolated units called \emph{Virtual Machines (VMs)} which allows 
multiple operating system to run on. 
In essence, virtualization add an extra layer of software called 
\emph{Virtual Machine Monitor (VMM)} or \emph{hypervisors} that can deploy, 
release and migrate VMs at runtime. 
Numerous VMMs have been designed for x86 commodity machines such as 
Xen \cite{Barham:2003vu}, Kvm \cite{Kivity:2007wu}, and VMware ESX server \cite{Barham:2003vu}.
 
\emph{Server consolidation} is a strategy for improving utilization of Cloud resources
 \cite{Zhang:2010vo}.
It uses a \emph{live VM migration techinique} \cite{Clark:2005ud} to concentrate VMs into fewer 
physical servers so that a datacenter can accommodate more applications and idle servers 
can be turned off to save energy. 
However, Cloud datacenter is a highly dynamic environment with application demand fluctruation, 
VMs arrival and release.
Maintaining a high level of server utilization is a continuous process with different server consolidation
methods. Mainly, there are two types of server consolidation: static and dynamic. 
Static method is often treated as an offline approach and it is applied in a periodical manner 
where a batch of VMs are allocated to a set of servers. 
Dynamic method is an online approach, 
where a single VM needs to be allocated to a set of servers. 
The overall goal for server consolidation is to maximize the utilization of servers as well as 
minimize the number of migration. 

% The convenience offered by Cloud and the difficulty of managing Cloud resources are 
% two sides of a coin. It is increasingly difficult to manage a large scale virtualized datacenter
% which usually contains hundreds of thousands servers.

Despite the usefulness of server consolidation, 


% In the previous decade, 
% Cloud providers who were focusing on ensuring quality of service, has quickly expanded their infrastructures into a large scale.
% As a result, the average utilization of servers is as low as 20\%, 
% according to \cite{energy_2007}'s observation of google's datacenter, hence, the energy was largely wasted. 

% Despite the energy consumption by none information and communication technologies (ICT) equipments
% such as cooling and airing systems, 
% the energy can be derived from two aspects, first, the hardwares of servers contribute a static consumption. 
% Second, the usage of computing, storage and network resources cause a dynamic consumption. 
% Therefore, improving the efficiency of resources are also two folds, minimizing the static part and 
% delievering more performance proportional to the dynamic workload.

% These are the inituition behind server consolidation. Server consolidation improves the utilization of resources by 
% concentrates workloads in a few servers so that others can be turned off or put into sleep to save energy. 
% Therefore, it achieves reduction of static energy consumption as well as improving the utilization of resources. 
% The consolidation can be done with the help of virtual machine (VM), 
% which can be easily transport from one physical machine (PM) to another. 
% However, as the scale of reallocation of VMs become large and various quality of services (QoS) requirements have to
% be considered, an efficient automatic approach is an urgent and necessary need.