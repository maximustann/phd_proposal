\section{Research Goals}

\begin{figure}
	\centering
	\includegraphics[width=\textwidth]{pics/thesisPlan.png}
	\caption{Relationship between objectives}
	\label{fig:objectives}
\end{figure}
\bx{The overall goal of this research is to optimize energy consumption of a container-based Cloud data center using EC-based approaches for three resource management processes:} application initial placement, periodic optimization, and dynamic placement. The specific research objectives of this work can be itemized as follows.
% In this thesis, we aims at providing a series of approaches to continuously optimize the a joint allocation of VMs and containers that considers three consolidation scenarios: Initialization, global consolidation, Dynamic consolidation. In addition, the static allocation normally involves with large amount of variables which is particular difficult to optimize. We are also going to propose a method to solve this problem.  These approaches combine element of AI planning, to ensure the objectives and constraint fulfillment, and of Evolutionary Computation, to evolve a population of near-optimal solutions. The research aims at determining a flexible way in creation of solutions to solve server consolidation problems. As discussed in the previous section, the research goal can be achieved in the following objectives and sub-objectives.

\subsection{Objective One: Develop EC-based approaches for the single objective joint placement of containers and VMs for application initial placement}
\label{sec:obj1}

\bx{The goal is to reduce the energy consumption in application initial placement considering container-based cloud data center.} To achieve this goal, inevitably, the first step is to propose a new bilevel model for container-based placement problem which considers the problem as a bilevel optimization problem. Then, we will explore evolutionary computation based approaches to solve the problem. The research goal leads to three objectives as follows.
% \textcolor{Maroon}{Currently, most research on container consolidation do not consider the two-level of allocation problem.} Unlike previous VM-based service consolidation, 
% most research focus on VM-based server consolidation technique. They often modeled the VM allocation problem as a vector bin-packing problem \cite{Zhang:2016cx}. 
% Container adds an extra layer of abstraction on top of VM. The placement problem has become a two-step procedure, in the first step, containers are packed into VMs and then VMs are consolidated into physical machines. These two steps are inter-related to each other. Previous research \cite{Piraghaj:2015uf} solve this problem in separated steps where the first step allocates containers to VMs and the second step allocates VMs to PMs with simple bin-packing heuristics. According to Mann's \cite{Mann:2016hx} observation, these two allocations should be conducted simultaneously to reach a near-optima solution, which essentially minimizes the energy consumption.

\begin{enumerate}
	\item Develop a new bilevel model to capture the relationship between containers allocation and energy consumption in container-based cloud data center.
	% \textcolor{Maroon}{This problem can be considered as a bilevel problem \cite{} the lower-level optimization: allocate containers to VMs and the upper-level: allocate VMs to PMs.} 
	% \textcolor{Maroon}{Since the existing models for container-based consolidation are based on VM-based model which incurs two problems.}
	% First problem is that they did not consider the interaction between two levels of allocation.
	% Second problem is that they did not consider balancing the residual resources (e.g between CPU and memory). 

	\bx{The major challenge is that no previous research consider the joint placement container and VM as a bilevel problem while the relationship between container, VM and energy consumption is unclear.} Specifically, two issues remain unsolved. First issue is that previous VM-based research do not consider the overhead of VM. However, the overhead of VMs is a major source of resource wastage (addressed in Section \ref{sec:comparison_container_vm}). Therefore, how to represent the impact of VMs remains unsolved.
	The second issue is related to a VM-based research,  Mishra \cite{Mishra:2011bz} discovers that when multiple resources are considered in the model, the balance between resources has a heavy impact. However, in the bilevel model, the balance of resources is difficult to represent. 

	\bx{The goal of the first sub objective is to propose a bilevel model for the bilevel optimization problem of joint placement of container and VM.} 
	% The reason to establish this model is because current server consolidation models are mostly VM-based, they cannot be directly applied on bilevel problems. 


	In order to establish a bilevel model, variables, constraints and objective functions need to be clarified before applying any optimization algorithm. Each level of the problem will be formulated to a multi-dimensional vector bin packing problem. It is still unclear that which energy function is the best to capture the relationship between container and VM so that the overall energy is low. In addition, we will address the overhead of VMs which is generally missing from previous research.

	We will start from the simplest case - single dimension of resource - to more general multi-dimensional resources model by reviewing a number of VM-based approaches. Specifically, we focus on their variables, constraints and objective function. Objective function is mainly related to energy consumption. Hence, energy model is another major issue to study. In addition, in the multi-dimensional resource model, we will address the balance of CPU and memory problem by investigating several resource wastage models \cite{Ferdaus:2014ep, Xu:2010vh, Gao:2013gg}. In this objective, we consider the static workload of applications, this is because the initial resource demand is often provided by the Cloud users.

	\item Propose a new EC based bilevel optimization approach to solve the application initial placement.

	\bx{Two challenges need to be solved. First one is to understand the interaction between two-level's placement.} In the joint placement of container and VM, placing containers into a minimum number of VMs does not necessary lead to the minimum energy consumption. Therefore, it is still unclear that relation among the selection type of VM, placement of container and placement of VM will affect the energy consumption. Second challenge is how to design the search operators and representation. Currently two types of representation: direct and indirect representation can be considered. However, it is unclear that which one is more suitable for the nature of the joint placement of container and VM problem.

	\bx{Based on the proposed bilevel model, the goal of this sub-objective is to develop an approach for the bilevel optimization problem using nested Evolutionary algorithms \cite{Sinha:2017et}.} Bilevel optimization is strongly NP-hard \cite{Mathieu:2011dw}, the solution space can be non-linearity, discreteness, no-differentiability, and non-convexity. Therefore, it is extremely difficult to design a proper search mechanism to find near optimal solutions.



	We will start from the simplest form: one dimensional bin-packing in each level to more complex multi-dimensional bin-packing. 

	Current nested methods have been used in solving bilevel problem, however, there is no research focus on bilevel bin-packing problem. We will investigate several approaches such as Nested Particle Swarm Optimization \cite{Li:2006br}, Differential evolution (DE) based approach \cite{Angelo:2013ee, Zhu:2006in} and Co-evolutionary approach \cite{Legillon:2012dd}. In order to adapt our problem to these existing approaches, we will develop suitable representations such as direct binary representation \cite{Xu:2010vh}, or indirect continuous probability representation \cite{Xiong:2014jq}. Genetic operators will also be investigated.

	\item Investigate methods to improve the scalability of the EC-based bilevel optimization approach.

	\bx{Based on proposed EC-based approach, the goal of this sub-objective is to improve scalability of the approach.} Although nested approaches have been reported effective, they are very time consuming \cite{Sinha:2017et}. Therefore, this sub objective intends to explore other directions to improve the execution time. One possible direction is single-level reduction \cite{Sinha:2017et}, which reduces the bilevel problem into a single dimensional problem. Clustering approaches such as K-means \cite{Xie:2011fj} can be useful in categorizing containers in predefined groups. Then, complementary containers can be grouped to reduce the variables of placement. Another possible solution is using a representation which embedded with simple heuristic (e.g First Fit); it allows to consider less number of PMs.  

	% \item Third, although nested approaches have been reported effective, they are often very time consuming. Therefore, our third sub-objective will focus on developing more efficient algorithms. There are several possible directions to be explored such as metamodeling-based methods \cite{Wang:2007em} and single-level reduction. 
		% \emph{New operators and searching mechanisms}\\
		% In order to utilize Evolutionary Computation (EC) to solve this problem, we are going to develop searching mechanisms according to the nature of problem as well as the selected representation. In order to achieve this goal, we will design several new operators. In order to evaluate the quality of these components, we will perform analytical analysis on the result.
\end{enumerate}
\subsection{Objective Two: Develop EC-based approaches for the multi-objective joint allocation problem for periodic optimization}
The goal is to develop multi-objective EC-base approaches for container-based cloud in periodic optimization with considering various types of workload to reduce the overall energy consumption.

% As previously (see Section  \ref{sec:motivation}) mentioned, the task is multi-objective: minimizing the number of migration and minimizing the overall energy consumption. This two objectives are conflicting since intensive optimization may incur a large number migration. The first challenge is how to solve the multi-objective bi-level optimization problem. In addition, we consider propose a robust periodic optimization which means the placement of applications does not affect much from the variant workloads. Therefore, we divide workloads into five categories according to Fehling \cite{Fehling:2014tl}: static, periodic, once-in-a-life-time, continuously changing, and unpredictable. Among five types of workloads, two of them:  once-in-a-life-time and unpredictable workloads are unsuitable for static placement, since their behavior are hard to foresee and plan, hence, they are normally solved by dynamic approaches which will be addressed in our third objective.  For static, periodic, and continuously changing workload, we are going to design specific solutions. We also use three questions to guide our objective.
% The robustness of a data center is particularly important. 
% The robustness measures the stableness of result of consolidation.
% Furthermore, we will investigate proactive approaches - considering future allocation.
% In order to measure the degree of robustness, we need to design a robustness measure. The second sub-objective is to design static consolidation algorithm with considering its previous immediate result. The third objective extends the second objective to a more general case, considering both previous immediate and next allocation. The evaluation of algorithm is based on analytical analysis of fitness functions and robustness measure. 

\begin{enumerate}
	\item Propose an EC-based bilevel multi-objective algorithm with aggregation approach for periodic optimization. \\
	\bx{The goal of this sub-objective is to modify previous proposed bilevel model and develop a baseline algorithm for the multi-objective optimization problem.} 
	In previous objective, we develop a single-objective joint allocation of containers and VMs.  For periodic optimization, the two objectives are minimizing the number of migration of containers and VMs and minimizing the energy consumption. We will modify the model (e.g add migration cost model) to adapt to the multi-objective problem. We will further develop an EC-based multi-objective algorithm with aggregation approach as our baseline approach. The aggregation approach turns a multi-objective problem into a single-objective problem by combining objectives into a single one. Therefore, we may use previous developed algorithm to solve periodic optimization problem. In addition, we will consider the workloads as static. 

	\item Propose an EC-based multi-objective algorithm for periodic optimization with Pareto front approach.\\
	\bx{The goal of this sub-objective is to develop an EC-based approach to solve the multi-objective joint allocation problem with Pareto front approach.} 
	In previous sub-objective we develop a baseline approach using aggregation approach. However, aggregation approach has some defects such as it cannot find the non-convex solution. In the sub-objective, we will use a Pareto front approach to solve the bilevel optimization problem. Currently only a few research \cite{Yin:2000bt, Deb:2009jh,Deb:2010in} focus on bilevel optimization problem. We will investigate which approach is the most suitable for this problem.
	The basic assumption for the workload is still static.
	% However, most of them are designed for continuous problems. Therefore, new representations and operators need to be considered for discrete problem. 

	% The assumptions for this objective, we will start from one dimensional of resource: CPU utilization. We will consider the static workload in this sub objective because they are common and easy to start with.  We can utilize the representation and problem models from previous objective. However, we need to propose new genetic operators to adapt the multi-objective problem.

	% like the case in single objective problem, we need to develop new representations, genetic operators.
	\item Propose an EC-based multi-objective algorithm for periodic optimization considering various types of predictable workload. \\
	\bx{The goal of this sub objective is to propose an approach for three predictable workloads: static, linear continuously changing, and periodic.} In previous objectives, we consider the problem using static workloads for simplicity. However, practically, rare workload remains static throughout its life cycle. Therefore, we will consider three types of predictable workloads \cite{Fehling:2014tl}. Based on our previous designed algorithm, we will investigate a general representation for different workloads. Furthermore, the representation may incur the changing of models, genetic operators and search mechanism.
	% Factor analysis such Principle Component Analysis \cite{Wold:1987wx} can be employed in developing new measure. Meanwhile, the representation used in static workload might not work, therefore, new representation, genetic operators need to be developed. 

	% Proactive consolidation \cite{Farahnakian:2015vj, Tan:2011jd} has driven a lot of attention in recent years. They mainly focus on making prediction of the workloads using a regression approach such as linear regression, multi-linear regression, and K-means regression. However, most of their consolidation methods are simple heuristics. In our approach, we seek to propose a combined technique.




	% \item Second, we will design a robustness measure. Previous studies only use simple measurement which counts the migration number between two static consolidation. This measurement aims at minimizing the number of migration between two  static placement processes. It may cause more migration in the next consolidation. Therefore, it needs a time-aware measure of the robustness of system. Therefore, in this objective, the first sub-problem we are going to solve is to propose a robustness measure. Currently, only a few research propose robustness aware server consolidation techniques \cite{Takouna:2014fa, Grimes:2016ia} have been proposed. They are either static threshold or probability-based threshold to measure the robustness of PMs. We will investigate an adaptive measure based on the historical data and current status.
	

	
	% \item \emph{Design a }\\

		% We will generalize the previous sub-objective to a more general one: design a time-aware allocation method which takes previous and next allocation into consider.
	\end{enumerate}

\subsection{Objective Three: Develop a hyper-heuristic single-objective Genetic Programming (GP) approach for automatically generating dispatching rules for dynamic placement.}

\bx{The goal for this objective is to develop a GP-based hyper-heuristic approach so that the generated dispatching rules can achieve both fast placement and global optimization with various workloads.}

% Previously, dynamic consolidation methods,including both VM-based and container-based, are mostly based on bin-packing algorithm such as First Fit Descending and human designed heuristics. As Mann's research \cite{Mann:2015ua} shown, server consolidation is more harder than bin-packing problem because of multi-dimensional of resources and many constraints. Therefore, general bin-packing algorithms do not perform well with many constraints and specific designed heuristics only perform well in very narrow scope. Genetic programming has been used in automatically generating dispatching rules in many areas such as job shop scheduling \cite{Nguyen:2014eu}. GP also has been successfully applied in bin-packing problems \cite{Burke:2006ei}. Therefore, we will investigate GP approaches for solving the dynamic consolidation problem. We will start from considering one-level of problem: migrate one VM each time to a PM. 

% Therefore, in this objective, we will use GP to automatically generate heuristics or dispatching rules.

\begin{enumerate}

	\item Develop a baseline GP-based hyper-heuristic (GP-HH) approach for dynamic placement. \\
	\bx{The goal of this sub-objective is to develop a baseline GP-HH approach for the first level of bilevel optimization problem: placing containers to VMs.} We will start from the basic assumption: static workload. In order to accomplish this task, we will construct primitive set by considering various features extracted from workload datasets such as the status of VMs (e.g resource utilization), features of workloads (e.g resource requirement). We will construct the functional set by using the general operators. The original genetic programming will be used as the search mechanism.

	\item Conduct feature extraction on the predictable workloads and unpredictable workloads. \\
	\bx{The goal of this sub-objective is to construct a GP primitive set by applying feature extraction on various types of application workload.} In previous sub-objective, we develop a baseline GP-HH on static workload. In order to develop a general GP-HH that can handle all kinds of workloads, we will extract features from predictable workloads such as linear continuous changing workloads, periodic workloads, and from unpredictable workloads: once-in-a-lifetime workloads. We will consider the using properties of workloads as features such as non-stationarity, burstiness, and self-similarity \cite{Feitelson:2002kn}. We will test the extracted features by applying classification on the training and test set. The final features will be used in the primitive set.

	% types including static, continuously changing, and periodic workloads, and two other workload types: once-in-a-life-time, unpredictable workloads. Based on these features, 
	% \textcolor{Blue}{More...}
	% \item Investigate the possible functional operators. 

	% \bx{The goal of this sub-objective is to construct a GP functional set.}
	% \textcolor{blue}{More will come.}

	\item Develop a Cooperative GP-HH approach to evolve dispatching rules for placing container and VMs. \\
	\bx{The goal of this sub-objective is to develop a cooperative GP approach to evolve dispatching rules.} In the baseline approach, we develop a GP-HH approach for single-level of placement with static workload. However, there is a case that no current VM is suitable for a container to place in; a new VM is needed to place at this moment. This case incurs a second level of placement. Therefore, to construct a complete placement dispatching rule, we will develop a cooperative GP-HH approach to solve the two-level of placement problem. We may reuse the single-level GP-HH in both level or develop a new GP-HH in the VMs to PMs level.  

	% This sub-objective explores suitable representations for GP to construct useful dispatching rules. It also proposes new genetic operators as well as search mechanisms. \textcolor{blue}{More will come.}

	\end{enumerate}

% \subsection{Objective Four (Optional) Large-scale Static Consolidation Problem}
% Propose a preprocessing method to eliminate redundant variables 
% Current static consolidation takes all servers into consider which will lead to a scalability problem. In this objective, we will investigate two branches of methods, first one categorizes a number of containers into fewer groups so that the granularity decreases \cite{Piraghaj:2015uf}. Second method categorizes PMs so that only a small number of PMs are considered. This approach will dramatically reduce the search space. The potential approaches that can be applied in this task are various clustering methods.

	% The 
	% initial placement can be considered as a two-level of multi-dimensional bin-packing problem with multi-objectives. 
	% \item First, from the perspective of \emph{Cloud resource allocation model}, 
	% 	traditional Infrastructure as a Service (IaaS) resource allocation model 
	% 	considers service allocation and VM placement as separated responsibility.
	% 	Cloud users or brokers need to concern about the resource mapping and VM selection and 
	% 	Cloud providers take care of the VM placement. 
	% 	However, as Cloud users tend to over-provision in order to satisfy the QoS, they often book more
	% 	resources than their need. This is the major reason for the low utilization in 
	% 	Cloud computing \cite{Vogels:2008bg}. This problem cannot be 
	% 	resolved solely from the Cloud providers' perspective but to change the current resource allocation model. In the new resource allocation model, the responsibility of service allocation and VM placement are in the same hand of the Cloud provider.
	% 	Thereafter, Cloud providers have the full control of resource management. 
	% 	However, this process makes the VM initial placement a more complicated problem. Currently, only few researchers \cite{} have noticed this problem and propose initial work to address this problem.
	% \item Dynamic server consolidation is the process that the resource management continuously detects the server runtime status and if one of the server is overloaded. Then,
	% one of the VM or container running inside the server will be migrated to other machine 
	% so that the applications do not suffer from a performance degradation. In a container-based
	% environment, there are three questions to be answered. \emph{When to migrate ?} refers to determine the time point that a physical server is overloaded. \emph{Which container to migrate?} refers to determine
	% which container need to be migrated so that it optimize the global energy consumption.
	% \emph{Where to migrate?} refers to determine which VM and host that a container is migrated to.  
	% Specifically, in the second question, the main idea in the literature is still simple heuristics and random selection. Therefore, we are going to investigate using a genetic programming technique to learn to choose the best. In the 
	% third question, literature also rely on simple bin-packing heuristics which do not consider the impact of environment. Therefore, we are going to propose an idea which uses the features of workload, to decide which VM
	% is the best choice.

	% \item Static server consolidation is the process that a batch of VM and container joint is
	% consolidated in order to achieve an low energy consumption status. This stage is often
	% applied when the overall energy consumption is reached a predefine threshold. The static
	% server consolidation can globally optimize the energy consumption of the data center.
	% The process is similar with the initialization stage but with different objectives and constraints.

	% \item 

	% \item Second, container as a service has become an important trend in the 		Cloud computing industry and being support by many Cloud providers 	such as Amazon, Azure and many 
	% 	open-source projects. Both Cloud users and providers 
	% 	are beneficial to its lightweight. It provides a finer granularity resource management
	% 	for Cloud providers. From Ref \cite{}, we observed that VM-level consolidation could further improve the utilization of resource as well as the footprint from traditional hypervisors. 
	% 	However, there is not much container consolidation methods were discussed in the literature. 
	% 	New models and consolidation methods need to be proposed to solve the problem. Moreover, similar to
	% % 	VM placement, container placement is also an multi-objective which need to be addressed.

	% \item Third, the joint VM and container poses another level of consolidation problem. Ref \cite{} states, 
	% 	one of the reasons that container consolidation has high SLA violations is because the 
	% 	higher migration rate of containers. Therefore, designing algorithms that dynamically select between VM and container migration based on application SLA requirement as well as the impact on energy consumption is the major concern of the joint allocation. This can be treated as a dynamic task. 

	% \item Fourth, a common problem that faced by both traditional VM-based and the recent container-based data 	center is the affinity aware resource allocation problem. 
	% 	Modern Cloud-native applications normally have more than one copy of its implementation called replica in order to resolve the stateless as well as load balancing problem. Hence, they must be allocated into different servers to maintain its reliability. Similarly, the backup of databases has the same issue. In the CaaS scenario, more constraints appear such as operating system aware allocation which means, 
	% 	certain container can only be allocated in a specific operating system \cite{}. 
	% 	The affinity-aware allocation has been discussed in the literature, 
	% 	however, they can only be applied in the VM-based data center.   
	% \item Fifth, a resource-utilization aware co-location scheme can be helpful in order to resolve the 
	% 	resource competition problem. The study is about the behavior of the applications deployed
	% 	in the same physical machine. The previous research assumes that the applications' behavior
	% 	is a priori \cite{}, however, applications' behavior can be changed over time. It is important to allocate
	% 	the compatible applications in the same physical machine so that the physical machine reaches a 
	% 	stable status.

	% \item Sixth, large scale of server consolidation has always been a challenge in a Cloud data center. 
	% 	Especially, typical number of servers in a data center is at the million-level. Many approaches 
	% 	have been proposed in the literature to resolve the problem. 
	% 	There are mainly two ways, both rely on distributed methods, 
	% 	hierarchical-based \cite{} and agent-based management systems \cite{}.
	% 	The major problem in agent-based systems is that agents rely on heavy communication to maintain a high-level utilization. Therefore, it causes heavy load in the networking. Hierarchical-based approaches are the predominate methods. Hierarchical-based methods, in essence, are centralized systems where all the states of machines are collected and analyzed. One of way to improving the effectiveness of centralized system is to reduce the size of variables without losing too much of the consolidation performance. The main idea is to eliminating the high-utilized servers so that it reduces the dimensionality. 
% \end{enumerate}