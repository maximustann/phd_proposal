\section{Research Goals}
% In this thesis, we aims at providing a series of approaches to continuously optimize the a joint allocation of VMs and containers that considers three consolidation scenarios: Initialization, global consolidation, Dynamic consolidation. In addition, the static allocation normally involves with large amount of variables which is particular difficult to optimize. We are also going to propose a method to solve this problem.  These approaches combine element of AI planning, to ensure the objectives and constraint fulfillment, and of Evolutionary Computation, to evolve a population of near-optimal solutions. The research aims at determining a flexible way in creation of solutions to solve server consolidation problems. As discussed in the previous section, the research goal can be achieved in the following objectives and sub-objectives.

\subsection{Objective One: Develop EC-based approaches for the single objective joint allocation of containers and VMs}

Currently, most research focus on VM-based server consolidation technique. They often modeled this problem as a vector bin-packing problem \cite{Zhang:2016cx}. Container adds an extra layer of abstraction on top of VM. The placement problem has become a two-step procedure, in the first step, containers are packed into VMs and then VMs are consolidated into physical machines. These two steps are inter-related to each other. Previous research \cite{Piraghaj:2015uf} solve this problem in separated steps where the first step allocate containers to VMs and the second step allocate VMs to PMs with simple bin-packing heuristics. Therefore, this is the first research that trying to solve the problem.

\begin{enumerate}
	\item First, our first sub objective is to propose a descriptive single objective model for the bilevel optimization problem of joint allocation of container and VM. The reason to establish this model is because current server consolidation models are mostly VM-based, they cannot be directly applied on bilevel problems. Therefore, variables, constraints and objective functions need to be clarified before applying any optimization algorithm.
	Each level of the problem will be formulated to a multi-dimensional vector bin packing problem. It is still unclear that which objective function is the best to capture the relationship between container and VM so that the overall energy is low. We will investigate several resource wastage models \cite{Ferdaus:2014ep, Xu:2010df, Gao:2013gg} and select a suitable one. In addition, several models have to be considered, including energy model \cite{Dayarathna:2016ua}, price model \cite{AlRoomi:2013te}, and workload model \cite{Magalhaes:2015ep}.
		
	\item Second, we will first develop a baseline approach that solve the problem using nested Evolutionary algorithms \cite{Sinha:2017et}. We will start from the simplest form: one dimensional bin-packing in each level to more complex multi-dimensional bin-packing.

	Nested methods have been used in solving bilevel problem for years, they are reported as effective approaches. We will investigate several approaches such as Nested Particle Swarm Optimization \cite{Li:2006br}, Differential evolution (DE) based approach \cite{Angelo:2013ee, Zhu:2006in} and Co-evolutioanry approach \cite{Legillon:2012dd}. In order to adapt our problem to these existing approaches, we will develop suitable representations and genetic operators.

	\item Third, although nested approaches have been reported effective, they are often very time consuming. Therefore, our third sub-objective will focus on developing more efficient algorithms. There are several possible directions to be explored such as metamodeling-based methods \cite{Wang:2007em} and single-level reduction. 
		% \emph{New operators and searching mechanisms}\\
		% In order to utilize Evolutionary Computation (EC) to solve this problem, we are going to develop searching mechanisms according to the nature of problem as well as the selected representation. In order to achieve this goal, we will design several new operators. In order to evaluate the quality of these components, we will perform analytical analysis on the result.
\end{enumerate}
\subsection{Objective Two: Develop EC-based approaches for the multi-objective joint allocation problem}

As previous section (see \ref{sec:motivation}) mentioned, the task is multi-objective since the number of VM migration has to be minimized while keep the overall energy low. In addition, periodic optimization is a time-dependent problem which means the optimal consolidation in previous operation might lead to more migrations in the current consolidation. The robustness of a data center is particularly important. The robustness measures the stableness of result of consolidation.

	% In order to measure the degree of robustness, we need to design a robustness measure. The second sub-objective is to design static consolidation algorithm with considering its previous immediate result. The third objective extends the second objective to a more general case, considering both previous immediate and next allocation. The evaluation of algorithm is based on analytical analysis of fitness functions and robustness measure. 

\begin{enumerate}
	\item First, we will develop EC-based approaches to solve the multi-objective joint allocation problem. In this problem, multiple objectives may involve at both of the levels. We will start from a simple case considering multi-objective in lower level: Minimizing VM migration and energy consumption.  Currently,  there are few the studies using EC methods \cite{Yin:2000ut,Deb:2010wh} for multiobjective bilevel optimization. We will investigate which one is more suitable for this binary problem. Furthermore, like the case in single objective problem, we need to develop new representations, genetic operators to apply the algorithms to solve the problem.


	\item Second, we will design a robustness measure. Previous studies only use simple measurement which counts the migration number between two static consolidation. This measurement aims at minimizing the number of migration between two  static placement processes. It may cause more migration in the next consolidation. Therefore, it needs a time-aware measure of the robustness of system. Therefore, in this objective, the first sub-problem we are going to solve is to propose a robustness measure. Currently, only a few research propose robustness aware server consolidation techniques \cite{Takouna:2014ua, Grimes:2016vv} have been proposed. They are either static threshold or probability-based threshold to measure the robustness of PMs. We will investigate an adaptive measure based on the historical data and current status.
	
	\item Third, we will design a proactive server consolidation approach. Based on a prediction of future server consolidation and the robustness measure,we will first design an approach which maximize the robustness and also minimize the current energy consumption. Proactive consolidation \cite{Farahnakian:2015vj,Tan:2011vy} has been studied extensively. Their experience in analyzing the workload pattern can be useful in designing the new algorithm.
	
	% \item \emph{Design a }\\

		% We will generalize the previous sub-objective to a more general one: design a time-aware allocation method which takes previous and next allocation into consider.
	\end{enumerate}

\subsection{Objective Three: Develop a hyper-heuristic Genetic Programming (GP) approach for automatically generating dispatching rules for dynamic consolidation}

Previously, dynamic consolidation methods,including both VM-based and container-based, are mostly based on bin-packing algorithm such as First Fit Descending and human designed heuristics. As Mann's research \cite{Mann:2015ua} showed, server consolidation is more harder than bin-packing problem because of multi-dimensional of resources and many constraints. Therefore, general bin-packing algorithms do not perform well with many constraints and specific designed heuristics only perform well in very narrow scope. Genetic programming has been used in automatically generating dispatching rules in many areas such as job shop scheduling \cite{Nguyen:2014eu}. GP also has been successfully applied in bin-packing problems \cite{Burke:2006ei}. Therefore, we will investigate GP approaches for solving the dynamic consolidation problem. We will start from considering one-level of problem: migrate one VM each time to a PM. 

% Therefore, in this objective, we will use GP to automatically generate heuristics or dispatching rules.

\begin{enumerate}
	\item First, we will investigate which features and attributes are important when dealing with energy efficiency problem. As the basic component of a dispatching rule, primitive set contains the states of environment including: status of VMs (e.g. utilization, wastage), features of workloads (e.g. resource consumption). Although there is no research has investigate how to use them to construct dispatching rules, there are extensive statistical analysis on workload \cite{Verma:2009wi}. The effectiveness of functional set and primitive set will be tested by applying the constructed dispatching rules on dynamic consolidation problem.

	\item Develop GP-based methods for evolving Dispatching rules \\
		This sub-objective explores suitable representations for GP to construct useful dispatching rules. It also proposes new genetic operators as well as search mechanisms.
	\end{enumerate}

\subsection{Objective Four (Optional) Large-scale Static Consolidation Problem}
Propose a preprocessing method to eliminate redundant variables 
Current static consolidation takes all servers into consider which will lead to a scalability problem. In this objective, we will propose a method that categorizes servers so that only a small number of servers are considered. This approach will dramatically reduce the search space. The potential approaches that can be applied in this task are various clustering methods.


	% The 
	% initial placement can be considered as a two-level of multi-dimensional bin-packing problem with multi-objectives. 
	% \item First, from the perspective of \emph{Cloud resource allocation model}, 
	% 	traditional Infrastructure as a Service (IaaS) resource allocation model 
	% 	considers service allocation and VM placement as separated responsibility.
	% 	Cloud users or brokers need to concern about the resource mapping and VM selection and 
	% 	Cloud providers take care of the VM placement. 
	% 	However, as Cloud users tend to over-provision in order to satisfy the QoS, they often book more
	% 	resources than their need. This is the major reason for the low utilization in 
	% 	Cloud computing \cite{Vogels:2008bg}. This problem cannot be 
	% 	resolved solely from the Cloud providers' perspective but to change the current resource allocation model. In the new resource allocation model, the responsibility of service allocation and VM placement are in the same hand of the Cloud provider.
	% 	Thereafter, Cloud providers have the full control of resource management. 
	% 	However, this process makes the VM initial placement a more complicated problem. Currently, only few researchers \cite{} have noticed this problem and propose initial work to address this problem.
	% \item Dynamic server consolidation is the process that the resource management continuously detects the server runtime status and if one of the server is overloaded. Then,
	% one of the VM or container running inside the server will be migrated to other machine 
	% so that the applications do not suffer from a performance degradation. In a container-based
	% environment, there are three questions to be answered. \emph{When to migrate ?} refers to determine the time point that a physical server is overloaded. \emph{Which container to migrate?} refers to determine
	% which container need to be migrated so that it optimize the global energy consumption.
	% \emph{Where to migrate?} refers to determine which VM and host that a container is migrated to.  
	% Specifically, in the second question, the main idea in the literature is still simple heuristics and random selection. Therefore, we are going to investigate using a genetic programming technique to learn to choose the best. In the 
	% third question, literature also rely on simple bin-packing heuristics which do not consider the impact of environment. Therefore, we are going to propose an idea which uses the features of workload, to decide which VM
	% is the best choice.

	% \item Static server consolidation is the process that a batch of VM and container joint is
	% consolidated in order to achieve an low energy consumption status. This stage is often
	% applied when the overall energy consumption is reached a predefine threshold. The static
	% server consolidation can globally optimize the energy consumption of the data center.
	% The process is similar with the initialization stage but with different objectives and constraints.

	% \item 

	% \item Second, container as a service has become an important trend in the 		Cloud computing industry and being support by many Cloud providers 	such as Amazon, Azure and many 
	% 	open-source projects. Both Cloud users and providers 
	% 	are beneficial to its lightweight. It provides a finer granularity resource management
	% 	for Cloud providers. From Ref \cite{}, we observed that VM-level consolidation could further improve the utilization of resource as well as the footprint from traditional hypervisors. 
	% 	However, there is not much container consolidation methods were discussed in the literature. 
	% 	New models and consolidation methods need to be proposed to solve the problem. Moreover, similar to
	% % 	VM placement, container placement is also an multi-objective which need to be addressed.

	% \item Third, the joint VM and container poses another level of consolidation problem. Ref \cite{} states, 
	% 	one of the reasons that container consolidation has high SLA violations is because the 
	% 	higher migration rate of containers. Therefore, designing algorithms that dynamically select between VM and container migration based on application SLA requirement as well as the impact on energy consumption is the major concern of the joint allocation. This can be treated as a dynamic task. 

	% \item Fourth, a common problem that faced by both traditional VM-based and the recent container-based data 	center is the affinity aware resource allocation problem. 
	% 	Modern Cloud-native applications normally have more than one copy of its implementation called replica in order to resolve the stateless as well as load balancing problem. Hence, they must be allocated into different servers to maintain its reliability. Similarly, the backup of databases has the same issue. In the CaaS scenario, more constraints appear such as operating system aware allocation which means, 
	% 	certain container can only be allocated in a specific operating system \cite{}. 
	% 	The affinity-aware allocation has been discussed in the literature, 
	% 	however, they can only be applied in the VM-based data center.   
	% \item Fifth, a resource-utilization aware co-location scheme can be helpful in order to resolve the 
	% 	resource competition problem. The study is about the behavior of the applications deployed
	% 	in the same physical machine. The previous research assumes that the applications' behavior
	% 	is a priori \cite{}, however, applications' behavior can be changed over time. It is important to allocate
	% 	the compatible applications in the same physical machine so that the physical machine reaches a 
	% 	stable status.

	% \item Sixth, large scale of server consolidation has always been a challenge in a Cloud data center. 
	% 	Especially, typical number of servers in a data center is at the million-level. Many approaches 
	% 	have been proposed in the literature to resolve the problem. 
	% 	There are mainly two ways, both rely on distributed methods, 
	% 	hierarchical-based \cite{} and agent-based management systems \cite{}.
	% 	The major problem in agent-based systems is that agents rely on heavy communication to maintain a high-level utilization. Therefore, it causes heavy load in the networking. Hierarchical-based approaches are the predominate methods. Hierarchical-based methods, in essence, are centralized systems where all the states of machines are collected and analyzed. One of way to improving the effectiveness of centralized system is to reduce the size of variables without losing too much of the consolidation performance. The main idea is to eliminating the high-utilized servers so that it reduces the dimensionality. 
% \end{enumerate}