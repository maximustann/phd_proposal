%% $RCSfile: using.tex,v $
%% $Revision: 1.1 $
%% $Date: 2010/04/23 01:57:05 $
%% $Author: kevin $
%%
\chapter{Literature Review}\label{C:background}
This chapter begins by providing a fundamental background to the field of Cloud computing in Section \ref{sec:background}.
Section \ref{} discusses the resource allocation in PaaS as well as the bi-level optimization including both biologically and non-biologically
inspired approaches; Section \ref{} delves into approaches that optimize the static placement. In addition, time-series aware problems and their approaches will be discussed; Section \ref{} discusses the issue of dynamic server consolidation and genetic programming; Section \ref{} covers approaches of scalability; Finally, Section \ref{} presents a summary of the important points identified in this review, alongside a discussion of the limitations of existing approaches.


\section{Background}

% This chapter begins by providing an overall understanding of Cloud computing and its related research field.
% Then, it narrows down to the Energy-aware resource management problem in Section \ref{C:}.


\subsection{An Overview of Cloud Computing}

Cloud computing is a paradigm that allows Cloud users and End users to acess Cloud services and applications based on their requirements regardless of where the services are. The term ``Cloud'' refers to no matter where the businesses and users are, they are able to access services. The advent of Cloud computing matches the trend of fast growing applications where these applications are deploying in a third-party data center and application users can use them without installing on their local computers.

Cloud computing has five chacteristics that makes it popular:

\begin{enumerate}
 \item On-demand self-service, it means a Cloud user can require computing resources (e.g CPU time, storage, software use) without the interaction with Cloud provider.
 \item Broad network access, Computing resources are connected and delivered over the network.
 \item Resource pool, a Cloud provider has a ``pool'' of resources which are normally virtualized servers. In IaaS, it provides predefined sizes of VMs. In PaaS, the resources are `invisible' to Cloud users who have no knowledge or ability to control. 
 \item Rapid elasticity, from the perspective of Cloud users, computing resources are assigned and released in real time. In addition, the resources assign to their software is ``infinite''. Therefore, Cloud users do not need to worried about the scalability of their applications.
 \item Measured Service provides an accurate measure of the usage of computing resources. It is fundamental to the pay-as-you-go policy.
\end{enumerate}

Cloud computing is derived from a few computing paradigms including Grid computing, service computing and etc. Grid computing allow their users to run a large-scale computational intensive tasks on a geographically distributed computing resources. Cloud computing is similar to grid computing of utilizing a group distributed
comtpuing resources. Unlike grid takes each computer as an individual resource, Cloud computing virtualizes the Physical resources into VMs and provides a centralized resource management. Service computing, including Service Oriented Architecture (SOA) and Web services, focus on constructing large scale Web-based application using the Internet. In SOA, web services are the building blocks of softwares and they can be distributed developed, deployed and composited through the Internet. The technologies, such as SOAP and REST, enables Web serivces.

% Service computing such as Service Oriented Architecture and Web Services, focus on the seamless business processes. 

% The purpose of Cloud computing is to reduce the cost of ``in-house'' 


% Cloud computing has made one critical change in software industry, it separates the role of traditional service provider into service provider and infrastructure provider. As Wei \cite{Wei:2010fn} states, ``one provides the computing of services, and the other provides the services of computing''. Therefore, this separation add one more layer between service provider and users, as: Cloud providers, Cloud users (service providers), and End users. 

% The separate responsibility between Cloud providers and Cloud users has completely reformed the software industry \cite{Buyya:2009ix} by providing three major benefits to Cloud users.
% First, Cloud users do not need upfront investment in hardwares (e.g PMs and networking devices) and pay for hardwares' maintenance. Therefore, it eliminates the risk of initial investment.
% Second, Cloud users do not need to worry about the limited resources which can obstruct the performance of their services when unexpected high demand occurs. Cloud providers off an elastic nature of Cloud which can dynamic allocates and releases resources for a software.  Cloud users only need to pay the resources that they have used under a \emph{pay-as-you-go} policy. Third, Cloud users can publish and update their applications at any location as long as there is an Internet connection. These advantages allow anyone or organization to deploy their softwares on Cloud in a reasonable price. 

% As previous section mentioned, our goal in this thesis is to help Cloud providers to increase their profit from data centers. Specifically, we focus on how to save money by cutting expenses of Cloud data centers. 
% Cloud providers have two ways of achieving this goals, one is to provide better Quality of Service (QoS) service such as high throughput, low latency, and high availability, to attract more Cloud users to use Cloud services. The other way is to save money 
% Each stakeholder has their objectives.  End users consume application. They require a guarantee quality of softwares including functional requirements which are the functionalities defined by Cloud users, and non-functional requirements such as availability, security, and network latency. Cloud users develop and deploy softwares on Cloud. They provide functional correct softwares and they also desire the non-functional requirements of softwares can be satisfied by Cloud resources. Cloud providers offer low-level resources such as computational power, storage and network bandwidth. Cloud providers want to increase their profit by attracting more Cloud users to use Cloud services and reducing the expense caused
% \begin{figure}[H]
% 	\centering
% 	\includegraphics[width=0.5\textwidth]{pics/energyConsumption.png}
% 	\caption{Energy consumption distribution of data centers \cite{Rong:2016js}}
% 	\label{fig:consumption}
% \end{figure} 

% Apart from upfront investment, energy consumption \cite{Kaplan:up01fR-k} is the major expense of data centers. Therefore, it is also the top concern of Cloud providers. Energy consumption is derived from several parts as illustrated in Figure \ref{fig:consumption}. Cooling system and servers or PMs account for a majority of the consumption. A recent survey \cite{Cho:2016kz} shows that the recent development of cooling techniques have reduced its energy consumption and now 
% server consumption has become the dominate energy consumption component. 

% \begin{figure}
% 	\centering
% 	\includegraphics[width=0.5\textwidth]{pics/util.png}
% 	\caption{Disproportionate between utilization and energy consumption \cite{Barroso:2007jt}}
% 	\label{fig:unproportional}
% \end{figure} 
% According to Hameed et al \cite{Hameed:2016cma}, PMs are far from energy-efficient. 
% The main reason for the wastage is that the energy consumption of PMs remains high even when the utilization are low (see Figure \ref{fig:unproportional}). Therefore, a concept of \emph{energy proportional computing} \cite{Barroso:2007jt} raised to address the disproportionate between utilization and energy consumption. 


% Virtualization \cite{Uhlig:2005do} is the fundamental technology that enables Cloud computing. It partitions a physical machine's resources (e.g. CPU, memory and disk) into several isolated units called virtual machines (VMs) where each VM allows an operating system running on it. This technology rooted back in the 1960s' and was originally invented to enable isolated software testing, because VMs can provide good isolation which means applications running in co-located VMs within the same PM without interfering each other \cite{Somani:2009ho}. Soon, people realized that it can be a way to improve the utilization of hardware resources: With each application deployed in a VM, a PM can run multiple applications. Later, a dynamic migration of VM was invented, which compresses and transfers a VM from one PM to another. This technique allows resource management in real time which inspires the strategy of server consolidation. 


% \subsection{Resource Management in IaaS}
% The resource management in IaaS can be roughly separated into three \cite{Svard:2015ic, Mishra:2012kx} which are applied in different scenarios: Application initialization, Prediction and Global consolidation, and Dynamic resource management (see Figure \ref{fig:management}). 

% \begin{figure}
% 	\centering
% 	\begin{subfigure}[b]{0.9\textwidth}
% 		\includegraphics[width=\textwidth]{pics/initialization.png}
% 		\caption{Initialization}
% 	\end{subfigure}
% 	\begin{subfigure}[b]{0.9\textwidth}
% 		\includegraphics[width=\textwidth]{pics/dynamic_resource.png}
% 	\caption{Dynamic resource management}
% 	\end{subfigure}
% 	\begin{subfigure}[b]{0.9\textwidth}
% 		\includegraphics[width=\textwidth]{pics/predict_consolidate.png}
% 	\caption{Prediction and Consolidation}
% 	\end{subfigure}
% 	\caption{Three stages of resource management in IaaS}
% 	\label{fig:management}
% \end{figure}


% Server consolidation is the core functionality involving in all Cloud resource management operations. These operations

% Data center constantly receives new requests for applications initialization. Once the new applications have been allocated, the utilization begins to drop. This is because, initially, applications are compactly allocated on PMs. As old applications instance are released because of canceling, the compact structure become loose. Dynamic resource management is a process which can slow the utilization from decreasing. It consolidates by re-allocating one application at a time. Finally, global consolidation is conducted periodically to dramatically improve the resource utilization.
% \begin{enumerate}
% 	\item \emph{Application initialization} is applied when new applications or new VMs arrive and the problem is to allocate them into a minimum number of PMs.

% 	In this problem, a set of applications or VMs are waiting in a queue. The resource capacity of the PM and usage by applications are characterized by a vector of resource utilizations including CPU, memory and etc. Then, the allocation system must select a minimum number of PMs to accommodate them so that after the allocation, the resource utilizations remain high. The problem is to consider the different combinations of applications so that the overall resource utilization is high. This problem is naturally modeled as a static bin-packing problem \cite{CoffmanJr:1996ui} which is a NP-hard problem meaning it is unlikely to find an optimal solution of a large problem. 

% 	\item \emph{Prediction and Global consolidation} is conducted periodically to adjust the current allocation of applications so that the overall utilization is improved.

% 	In this problem, time is discrete and it can be split into basic time frames, for example: ten seconds. A periodical operation is conducted in every $N$ time frames.
% 	A cloud data center has a highly dynamic environment with continuous arriving and releasing of applications. Releasing applications cause hollow in PMs; new arrivals cannot change the structure of current allocation. Therefore, after the initial allocation, the overall energy efficiency is likely to drop along with time elapsing. 

% 	In prediction, an optimization system takes  the current applications' utilization records as the input. Make a prediction of their utilization in the next period of time. 
% 	In Global consolidation, based on the predict utilization and the current allocation - including a list of applications/VMs and a list of PMs, the system adjusts the allocation so that the global resource utilization is improved.

% 	In comparison with initialization, instead of new arrivals, the global consolidation considers the previous allocation. Another major difference is that global consolidation needs to minimize the differences of allocation before and after the optimization. This is because the adjustment of allocation relies on a technique called live migration \cite{Clark:2005uda}, and it is a very expensive operation because it occupies the resources in both the host and the target. Therefore, global optimization must be considered as a time-dependent activity which makes the optimization even difficult.

% In comparison with dynamic consolidation, global consolidation takes a set of VMs as input instead of one. Therefore, it is time consuming and often treated as a static problem.
% 	\item \emph{Dynamic resource management} 
%  	Dynamic resource management is applied in three scenarios. \textbf{First},  it is applied when a PM is overloading. In order to prevent the QoS from dropping, an application is migrated to another PM. This is called hot-spot mitigation \cite{Mishra:2012kx}. \textbf{Second}, it is applied when a PM is under-loading. Under-loading is when a PM is in a low utilization state normally defined by a threshold. At this moment, all the applications in the under-loading PM are migrated to other active PMs, so the PM becomes empty and can be turned off. This is called dynamic consolidation. \textbf{Third}, it is applied when a PM having very high level of utilization while others having low. An adjustment is to migrate one or more application from high utilized PMs to low ones. This is called load balancing.

% 	No matter which scenario it is, a dynamic resource management always involves three steps . 
% 	\begin{itemize}
% 		\item \emph{When to migrate?} refers to determine the time point that a PM is overloaded or underloaded. It is often decide by a threshold of utilization.
% 		\item \emph{Which application to migrate?} refers to determine which application need to be migrated so that it optimize the global energy consumption.
% 		\item \emph{Where to migrate?} refers to determine which host that an application is migrated to. This step is called dynamic placement which is directly related to the consolidation, therefore, it is decisive in improving energy-efficiency. 
% 	\end{itemize}

% 	Among three operations, dynamic placement is a dynamic and on-line problem.
% 	The term ``dynamic'' means the request comes at an arbitrary time point. An on-line problem is a problem which has on-line input and requires on-line output \cite{Borodin:uQcy_H6C}. It is applied when a control system does not have the complete knowledge of future events.

% 	There are two difficulties in this operation, firstly, dynamic placement requires a fast decision while the search space is very large (e.g hundreds of thousands of PMs). Secondly, migrate one application at a time is hard to reach a global optimized state.

% \end{enumerate}



% Finally, a consolidation plan includes four major items:
% 			\begin{enumerate}
% 				\item A list of existing PMs after consolidation
% 				\item A list of new virtual machines created after consolidation
% 				\item A list of old PMs to be turned off after consolidation
% 				\item The exact placement of applications and services
% 			\end{enumerate}

% By the nature of Cloud resource management, server consolidation techniques can also be categories into static and dynamic methods \cite{Xiao:2015ik, Verma:2009wi}. Static method is a time consuming process which is often conducted off-line in a periodical fashion; initialization and global consolidation belong to this category. It provides a global optimization to the data center. Dynamic method adjusts PMs in real time. It often allocates one application at a time. Therefore, it can be executed quickly and often provides a local optimization to the data center.


















% They are bonded by the Quality of service and computing resource. Quality of service and the computing resources are the two sides of a coin. 



% Cloud users deploy their software on Clouds. They want to increase the profit by increasing income and decreasing expense. In order to accomplish this goal, they can attract more End suers by improving the functionality of softwares and the non-functionality features by guaranteeing Quality of Service (QoS). To improve the non-functionality features, Cloud users need to reserve enough resources as well as minimizing the resource so that the cost is low.

% service capacity planning is the core process. The capacity planning has two conflicting objectives, on one hand, it must meet End users' QoS requirement by using enough resources.  On the other hand, the cost must be minimized. In pre-Cloud era, the capacity planning determines the upfront investment in infrastructure, therefore, capacity, reliability, and scalability are all need to be carefully considered and balanced. In Cloud environment, the burden of capacity planning is largely released by elastic resource management and the pay-as-you-go policy.

% Cloud users identify a list of critical QoS parameters called Service Level Agreement (SLA) which specifies the non-functional requirements such as throughput, latency, and availability. These QoS parameters are mapped to resources (e.g. CPU, memory, network bandwidth) which can satisfy these requirements. Violation of SLA will lead to penalty and decreasing in number of users. Therefore, in essence, the key to attract more users is an effective resource management system which can rapidly react to the fluctuating resource demand. 

% Beside increase the income, reduce the expense is another way to improve profit. As previous section mentioned, energy consumption is the main source of expense. In energy consumption, server energy consumption is the core that needs to be improved. 

% \subsection{Energy-aware Resource Management}
% \subsection{An Overview of Evolutionary Computation}


% In order to understand Cloud computing, firstly we will illustrate the five essential elements of Cloud computing and their advantages.

% Cloud computing has five essential elements:


% \textcolor{Blue}{What is your purpose to describe the following content?}
% \textcolor{Red}{
% 	I would like to discuss the differences, advantages of disadvantage of the resource management in different service models. Therefore, after illustrate how they are work. The point is to compare the resource management. And then, lead to a new service model. And the advantage of new service model should be obvious. 
% }\\
% Traditional Cloud computing has three service model as illustrated in Figure \ref{}.
% Infrastructure as a Service (IaaS), Platform as a Service (PaaS), and Software as a Service. 
% \subsection{Resource Management}
% Scope of Cloud computing resource management.
% \begin{enumerate}
%  \item Actors
%  \item Management Objectives
%  \item Resource Types
%  \item Enabling Technologies
% \end{enumerate}







% \section{VM-based Static Consolidation Methods}

So far in the industry, most Cloud data center is based on virtual machine technology. Therefore, VM-based resource management is the mainstream in both industry and academia. 

As previous mentioned, server consolidation is one of the technique to reduce the power consumption. Various techniques have been proposed in this field, these techniques can be roughly grouped into static and dynamic approaches.

Static approaches adjust the allocation of VMs in a periodical fashion (e.g. weekly or monthly). 

Dynamic appproaches are conducted by a runtime placement manager to migrate VMs 
automatically in response to workload variations.




% Static initialization, is also frequently referred to initial placement problem \cite{Jennings:2015ht}. Whenever a request for provisioining of applications by one or more Cloud users. The resource management system schedules the applications into a set of PMs. Currently, most state-of-the-art research focus on VM-based placement, in this case, applications are installed in VMs. Therefore, ``application placement'' and ``VM placement'' are used interchangable in the literature. 

% In energy-aware resource management, the initialization has the objective of minimizing the used PMs. In literature, the static initialization problem is often modeled as the vector bin packing problem. Each application represents an item and PMs represents bins.

% A d-dimensional Vector Bin Packing Problem ($VBP_d$), give a set of items $I^1, I^2, \dots, I^n$ where each item has $d$ dimension of resources represented in real number $I^i \in R^d$. A valid solution is packing $I$ into bins $B^1, B^2, \dots, B^k$. For each bin and each dimension, the sum of resources can not exceed the capacity of bin. The goal of Vector Bin Packing problem is to find a valid solution with minimum number of bins. $VBP_d$ is an NP-hard problem.

% Because of its NP-hard nature, several researchers propose well-known bin-packing heuristics
% First Fit (FF), First Fit Decreasing (FFD), Best Fit (BF) and etc. These algorithms have constant-factor approximation to bin-packing problems. However, VM placement is more complex than bin-packing problem \cite{Mann:2015ua}, 

% Panigraph et al \cite{Panigrahy:2011wk} study variants of First Fit Decreasing (FFD) algorithms and inspired by bad isntances for FFD-type algorithms, they propose a geometric heuristics which outperform FFD-based heuristics in most of cases.

\section{Container initilization}
Initialization is one of the major step in resource management. It can be considered as a static problem \cite{Jennings:2015ht} or dynamic problem \cite{Beloglazov:2012bw}. 
As we discussed in the previous section, a dynamic allocation normally cannot gives a global optimized solution of a batch of tasks. Therefore, in the context of maximizing the energy efficiency of a data center, we category initialization into a static optimization approach.

% Previous research mainly use heuristic to solve this problem. 
Mesos \cite{Hindman:2011ux} is a platform for sharing commodity clusters between cluster computing frameworks such as Hadoop and MPI. It has a two-level of resource allocation architecture where a master node and several slave node. A master node only decides how many resources to offer to each framework (slave) based on fair sharing policy \cite{Ghodsi:2011vm}. Each slave node belongs to a cluster framework and it makes the decision of which resources to use. Each framework has to define its allocation policy. Mesos focus on sharing resources across multiple frameworks and has a better scalability since it deligates the application placement to decentralised slave nodes. The main problem for Mesos is that it does not consider energy consumption for Cloud providers with user-defined allocation policy.

Piraghaj et al \cite{Piraghaj:2016bw} propose an architecture for container-based resource management. Their allocation approach is policy-based, where it allocates each VM with containers until its estimated utilization above 90\%. This heuristic is similar to First Fit, it is fast but does not guarantee global optimal.

% \input{dynamic_consolidation}
% \section{Evolutionary Computation Approaches on Server Consolidation Problem}






% \subsection{An Overview of server consolidation}







% The reasons for energy wastage can be derived from several components of a data center, including 
% cooling systems, network equipments, and server consumption. 
% A well-accepted measurement: PUE (Power Usage Effectiveness) \cite{Belady:IMLoaM62}
% a standard measurement for data center energy efficiency which compares the 
% total power with the power used to power IT equipment (e.g. server, network equipments). 
% A recent survey \cite{Cho:2016kz} shows that the recent development of cooling techniques 
% have reduced its energy consumption and now 
% server consumption has become the dominate energy consumption component.
% Despite improvements in hardwares, various software techniques have been proposed 
% to reduce the energy consumption of servers 
% such as: Server Consolidation and Dynamic Voltage and Frequency Scaling (DVFS) \cite{}.

% Virtualization \cite{Uhlig:2005ub} is the core technology that not only enables 
% the elastic management of Cloud resource but also can be used to improve the utilization and reduce 
% energy consumption.
% It maps a physical machine's system resource - including processors, memory, and 
% other devices - into isolated units called \emph{Virtual Machines (VMs)} which allows 
% multiple operating system to run on. 
% In essence, virtualization add an extra layer of software called 
% \emph{Virtual Machine Monitor (VMM)} or \emph{hypervisors} that can deploy, 
% release and migrate VMs at runtime. 
% Numerous VMMs have been designed for x86 commodity machines such as 
% Xen \cite{Barham:2003vu}, KVM \cite{Kivity:2007wu}, and VMware ESX server \cite{Barham:2003vu}.
 
% \textcolor{Maroon}{A brief introduction of server consolidation}

% It aims at improving the income by guaranteeing \emph{Quality of Service (QoS)}
% \cite{Calheiros:2011ul} of the maximum number of applications that a datacenter can accommodate.
% Server consolidation \cite{Zhang:2010vo} is one of the widely used strategies 
% for resource management \cite{marinescu2013cloud}.
% It reduces the server energy consumption by gathering virtual machines (VMs) into a fewer 
% number of physical servers so that idle servers can be turned off. 
% The server consolidation techniques on the server-level
% have been extensively studied in the past decade \cite{}. 
% However, the recent development of container technology enables a VM-level of consolidation, which 
% has not driven much attention. 
% Container is a lightweight virtualization
% technology which allows an application running in a single container. 
% Multiple containers can be packed in a single virtual machine. 
% Two main advantages make the container popular. 
% First, containers do not need a Virtual Machine Monitor (VMM) but relies on the operating system; 
% it reduces the overhead used on managing the virtual system. 
% Second, the communication \cite{} between containers are much 
% easier (e.g. inter-process communication) than an inter-VM communication. This feature is 
% particularly useful for micro-service-based Web applications where their processes are packed
% into separated containers.
% This new technology has brought new challenges to server consolidation. 
% Traditional algorithms can not be directly applied since there is an extra level
% of virtualization. Affinity and communication aware allocation play an much important role 
% in container-based environment. Therefore, new techniques and algorithms are need to be proposed. 

% Currently, few literatures address the 

% Therefore, this thesis will focus on providing solutions to 
% container-based server consolidation.

% Mainly, there are two types of method: static and dynamic.
% Static methods are often treated as off-line approaches and applied in a periodical manner 
% where a batch of VMs are allocated to a set of servers. 
% They are conducted at a given point of time when
% the overall utilization in a data-center is degraded into a certain level: 
% e.g, a predefined CPU utilization threshold. Because static methods often consider partial or all VMs
% in a datacenter, it is often treated as a global optimization task \cite{}.
% The static method often models the problem as a off-line bin-packing problem and 
% solved with deterministic or heuristic algorithms. The goal is often to find a global optimal solution
% in terms of server utilization and other criteria.
% Dynamic method is an on-line approach. It assumes a scenario when a single server is 
% overloading with multiple VMs, migrate one of the internal VMs out from 
% the host will release the overloading. Dynamic method is used in between 
% two static consolidation processes to ease the overloaded server as well as consolidation.
% As it only moves one VM at a time, it often applies greedy-based heuristic, therefore, hard to 
% reach a global optimization.

% \textcolor{Maroon}{Difficulty of server consolidation}
% Server consolidation is often considered as 
% a global optimization problem where its goal is to minimize the energy consumption.
% Challenges are posed at different stages of consolidation process. 
% Static problem is often modeled as a bin-packing problem  \cite{Mann:2015ua} 
% which is known as NP-hard meaning it is unlikely to find an optimal solution 
% of a large problem. 
% Furthermore, server consolidation often has 
% much complicated assumptions and constraints - including multi-dimension resources, 
% migration cost, and heterogeneous bins \cite{Mann:2015ua}.
% Because of its NP-hard nature, deterministic methods such as 
% Integer Linear Programming \cite{Speitkamp:2010vp} and Mixed
% Integer Programming \cite{} are unsuitable for a large scale problem 
% because of the long computation time. 
% Heuristic methods such as First Fit Decreasing (FFD) \cite{Panigrahy:2011wk}, 
% Best Fit Decreasing (BFD) \cite{Xu:2010vh}, 
% and other bin packing algorithms are often applied to approximate the optimal solution.
% Moreover, manually designed heuristics are designed to tackle the special requirements such 
% as \cite{}. 
% Although these greedy-based heuristics can quickly solve the consolidation problem, 
% As \cite{Mann:2015ua} shown, server consolidation is a lot more harder than bin-packing problem,
% therefore, these greedy-based heuristics can not reach a good approximation but easy to 
% be stuck at a local optima.

% In addition to traditional VM-based server consolidation, container-based server consolidation
% has an extra level of virtualization which leads to an even difficult problem.
% Traditional server consolidation algorithms cannot be directly applied to 
% the problem because of the different structure and complexity. 

% This thesis, therefore, aims at
% providing an end-to-end solution to the container-based server consolidation problem.

% First, aggressive consolidation causes overloading physical resources. 
% It leads to performance degradation since the application cannot obtain enough resources
% the VM promised. It is hard to determine the maximum level of utilization of a physical machine.


% Resource allocation and scheduling is the core of resource management in Cloud computing.
% The main purpose is to satisfy both Cloud users' and Cloud providers' requirements by 
% allocating sufficient resources to incoming tasks as well as keep a high utilization of the resources.
% In order to accomplish this goal, 
% resource allocation and scheduling tasks are often treated as optimization tasks.

% An abstract model of resource allocation is shown in Figure \ref{}.





% \subsection*{An Overview of Server Consolidation}

% A Service consolidation is the process of packing virtual machines in a number of physical 
% machines in order to reach a high utilization of resource as well as using a minimum number of 
% physical machines. The key aspect of server consolidation is that, in order to achieve the 
% desired result, the permutation of virtual machines must be considered. It is important to list the 
% difference between a static and dynamic server consolidation approaches \cite{}. 
% Static 
% In static approaches, . In dynamic approaches, bla.... A typical system model for a data center
% resource management system can be seen in Figure \ref{}. 
% \begin{figure}
% 	\centering
% 	\includegraphics[width=1.0\textwidth]{pics/dataCenter-1.png}
% 	\caption{A datacenter management model \cite{Varasteh:2015fu}}
% 	\label{fig:arch}
% \end{figure}

% The taxonomy of server consolidation has not reach an agreement. In Verma's work \cite{Verma:2009wi}, they categorize it into three groups: static, semi-static and dynamic consolidation. In static consolidation, applications are placed on PM without any further movement. Semi-static refers to periodical adjustment. Dynamic consolidation is still applied on a set of VMs in response to their workload variations.









% A dynamic server consolidation approach
% can usually be decomposed into a series of steps, 
% reflecting the process required to produce a solution \cite{}. 
% These steps are shown in Figure \ref{} and discussed below:
% \begin{enumerate}
% 	\item When to migrate. Dynamic migration occurs on two scenarios: migrating VMs from overloaded server; and item migrating VMs from underloaded server.
% 	\item Which VMs to migrate.  After deciding to migrate a VM from a server, the next step is 
% 	to make a decision of which VM to migrate.
% 	\item Where to migrate the VMs. The key step is to determine where to allocate a VM which leads to global optimization.
% \end{enumerate}


% \subsection*{A Comparison between CaaS and IaaS based Cloud model}
% From a computing system design point of view, we believe service allocation and VM Placement
% are closely related and should be considered as a single allocation task.
% \subsection*{Service Allocation}
% Service allocation refers to the process of mapping a Web service on a certain type of VM.
% It is conducted by Cloud customers (e.g Web service providers) or Cloud brokers deligated by a 
% Cloud customer.
% The resource mapping involve with two steps, 
% resource demand profiling \cite{} and VM selection \cite{}. 
% Resource demand profiling is an estimation of the workload of a service. 
% Because the web application has dynamic workload over time \cite{}, 
% service providers or cloud brokers normally would like to estimate the future workload so that they 
% can choose how much resources to rent in order to guarantee the 
% Service Level Agreements (SLAs)  \cite{} to end customers. In this step, historical statistics
% are often used and based on its peak workload estimation, service providers often
% rent resources more than they need. Since the peak workload only accounts for a small portion
% of its total operation time, intelligent strategies are applied to tackle the over-provisioning and 
% under-provisioning problems.

% Public Cloud providers often provide various configurations of VM, 
% often refered to as VM types or instance types \cite{Li:2011ti}. 
% An instance type is defined as its resources such as memory size, number of processors and 
% CPU frequency. 

% Previous research focus on how to rent an appropriate amount of resources so that it minimize the 
% service providers' costs.

% Ref \cite{Candeia:2010wt} considers e-Science applications with bag-of-tasks (BoT) model. It
% aims at executing a bag of independent tasks with the least amount of Cloud resources before
% a deadline.
% Unlike a service allocation problem, where service is permanantly deployed in a reserved VM, 
% they consider an on-demand VM allocation. 
% That means, when a bag of tasks comes, the system
% dynamically assign a set of VM to execute these tasks. 
% It evaluates four heuristic algorithms and 
% concludes that a greedy-based approach achieves the best result.

% Li et al \cite{Li:2011ti} consider a dynamic cloud scheduling problem 
% from Cloud brokers' perspective. It considers scenarios such as Cloud provider changing its offer 
% (e.g changing of pricing schemes or VM types) and service performance changing, 
% a cloud broker needs to adjust the VMs allocation across multiple Cloud providers. 
% This work proposes a model which maximize a Cloud consumers' profit by adjusting VMs across
% multiple Clouds. Their model does not consider a Cloud provider's profit.

% Wang and Xia \cite{Wang:2016ui} propose a MIP formulation for energy-aware VM placement in Cloud.
% The major difference between their work and previous work is that they use a non-linear energy model \cite{Gandhi:2009wp}.  Based on this model, they consider two resources CPU and memory. In order to solve the non-linear problem, they propose a linearization method which uses piecewise linear function to approximate the non-linear objective function. In the end, they uses a relaxation method to relax the integer linear programming problem into continuous and apply a rounding function to obtain a near-optimal solution. In summary, the energy model is the key objective in consolidation problem. With different models, the applied algorithm can be very different. However, EC approaches can deal with both linear and non-linear problem without any changes. That is an obvious advantage. 

% Virtualization technology was first developed in IBM System/360 in 1960s 
% targeting a finer granularity resource management. 
% It partitions a physical machine into separated resources 
% called virtual machine (VM) which can be allocated 
% and moved from one server to another. 
% This flexibility not only allows resources to be managed in a dynamic manner, 
% but also enables server consolidation.

% In a Cloud datacenter, server consolidation is used as technique to combat \emph{server sprawl}.
% Server sprawl refers to the low utilization of physical servers. The main cause for server sprawl is
% the requirement of running applications in isolation \cite{Vogels:2008bg}. 
% That is, an application is deployed in one or more servers which 
% offer much more resources than it needs. 
% With full virtualization \cite{}, a server's physical resources including CPUs, memory, and I/O devices
% are divided into finer granularity level of resources.
% Virtual machines offer different sizes of resources that 
% can be choosed to satisfy different demands from applications. 

% \subsection*{An Overview of Evolutionary Computation}


% \subsection*{Initial Placement}

% \section*{Container-based VM Multiplexing}
% Container has been introduced back in the 1980s' \cite{}. The recent development of container
% allows only one process running in a container; this is revolutionary invention is called application
% container. It plays an important role in Cloud computing since it is lightweighted, easier to configure
% and enable finer adjustment than the VM-based resource management.


% \begin{figure}
% 	\centering
% 	\includegraphics[width=0.6\textwidth]{pics/container.png}
% 	\caption{A Container as a Service Deployment Model}
% 	\label{fig:container}
% \end{figure}

% % % % % \section*{Dynamic VM Placement}
% % % % The traditional dynamic VM placement approaches can be categorized into three groups: Heuristic
% % % based approaches, deterministic approaches and dyanmic techniques with load prediction.

% % However, we believe the dynamic VM placement problem is highly dependent on an overall workload

% \section*{Large scale VM Placement}
% If you are writing an MSc or PhD thesis you should \emph{not} be using this style. Instead use \verb=vuwthesis=, which is based on the book style, and conforms to the VUW thesis rules. The thesis style is rather different from the project report style. 

% This document is formatted using a local (to ECS and MSOR at VUW) style file. When you write your project report you should be very careful when changing the beginning. The document class settings should read:

% \begin{verbatim}
% \documentclass[11pt
%               , a4paper
%               , twoside
%               , openright
%               ]{report}
% \end{verbatim}
% The options to the document class specify that:
% \begin{itemize}
% \item 11pt font is to be used for the main body text,
% \item  we will print on A4 paper, 
% \item we will use duplex (two-sided) printing,
% \item we want chapters to start on a right-hand page. 
% \end{itemize}

% The opitons you supply to the  \texttt{vuwproject} style will depend upon
% what you are using the style for.

% \subsection{Specifying the details}
% The \texttt{vuwproject} style sets up the front page properly, and provides various commands allowing you to specify the author, title, supervisor or supervisors, the school from which the report is being submitted and the degree that the report is being submitted for. The style has deliberately been designed to do as little as possible. This means that your document can easily be re-formatted as a technical report, or for submission to a conference or journal by using the appropriate style.

% It is also possible to use the style to easily produce documents on a
% stand-alone computer where your \LaTeX installtion might not have all
% of the  files and fonts available to machines within ECS or MSOR.

% Most of the options to the \texttt{vuwproject} style are currently a simple
% choice and there's a default that will make it obvious if you do not make
% a choice.

% Use one of the following options to use fonts available on ECS/MSOR machines
% or to use images that imitate them (assumes you have copies of the images)
% \begin{itemize}
% \item \verb+font+
% \item \verb+image+
% \end{itemize}

% Use one of the following options to set the school,
% \begin{itemize}
% \item \verb+ecs+
% \item \verb+msor+
% \end{itemize}

% Use one of the following options to choose a pre-defined degree,
% \begin{itemize}
% \item \verb+bschonscomp+
% \item \verb+mcompsci+
% \end{itemize}

% or use this command to use an explicit degree or diploma name
% \begin{itemize}
% \item \verb+\otherdegree{DEGREE OR DIPLOMA NAME}+
% \end{itemize}

% So, for example, to submit a report for the Master of Comp Sci degree, which
% the style knows about, from within ECS, using the images, you'ld ensure the
%  \texttt{vuwproject} line options looked like:

% \begin{verbatim}
% \usepackage[image,ecs,mcompsci]{vuwproject}
% \end{verbatim}

% whereas for a degree from within MSOR, when creating the final version on
% an ECS or MSOR machine where you have access to the fonts, you would use
% these options

% \begin{verbatim}
% \usepackage[font,msor]{vuwproject}
% \end{verbatim}


% and add the other degree's name using this command 

% \begin{verbatim}
% \otherdegree{DEGREE OR DIPLOMA NAME}
% \end{verbatim}

% To specify the supervisor or supervisors use either of the following commands in the preamble.
% \begin{itemize}
% \item \verb+\supervisor{The Supervisor}+
% \item \verb+\supervisors{Super 1 and Super 2}+
% \end{itemize}

% If you fail to set any degree or supervisor, or the school, then the front page will report this.

% The \texttt{vuwproject} style also sets the default font to be Palatino, using the \texttt{mathpazo} package. Palatino is one of VUW's `offical' fonts, and is the font used for the heading on the front page. The \texttt{mathpazo} package also typesets maths in a style which suits Palatino. 

% \section{Copying the style}
% If you want to write your project report away from VUW you will need to make your own copy of the \texttt{vuwproject} style.

% You can find out where the original lives by reading the messages that \LaTeX\ prints when it is run.

% Alternatively, you can down load a copy of the  \texttt{vuwproject} style from
% the ECS webpages.

% Any changes made to your own copy of the \texttt{vuwproject} style will not be reflected in the original, and \textit{vice versa}. Hence it makes sense to leave this as it is, and use a local style file for your own definitions.   
