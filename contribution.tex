\chapter{Proposed Contributions and Project Plan}\label{C:con}

This thesis will contribute to the field of Cloud Computing by proposing novel solutions for the joint allocation of container and VM, and to the field of Evolutionary Computation by proposing new representations and genetic operators in evolutionary algorithms. The proposed contributions of this project are listed below:

\begin{enumerate}
	\item Propose a new model for the joint allocation of container and VM problem.
	New representations will be proposed to problem.
	The first EC-approach to solve the problem. 
	Reduce the energy consumption of data centers.
	\item Propose a new robustness measure for time-dependent server consolidation problem. New optimization algorithms not only consider the two-level of packing problem but also take the previous and next consolidation into considered.
	\item Develp a GP-based hyper-heuristic approach to automatically generate dispatching rules for dynamic server consolidation problem. New functional and primitive set for constructing useful dispatching rules will be studied. New genetic operations and representations will be proposed.
	\item Develop a preprocessing algorithm to reduce the number of variables in a large scale static consolidation problem without scacrificing too much performance. Useful features of server consolidation problem will be studied.
\end{enumerate}

\section{Overview of Project Plan}
Six overall phases have been defined in the initial research plan for this PhD project, as
shown in Figure \ref{fig:}. The first phase, which comprises reviewing the relevant literature, investigating both VM-based and container-based server consolidation algorithms, and producing the proposal, has
been completed. The second phase, which corresponds to the first objective of the thesis, is
currently in progress and is expected to be finished on time, thus allowing the remaining
phases to also be carried out as planned.

\section{Project Timeline}
The phases included in the plan above are estimated to be completed following the timeline
shown in Figure \ref{}, which will serve as a guide throughout this project. Note that part of the first phase has already been done, therefore the timeline only shows the estimated remaining time for its full completion.

\section{Thesis Outline}
The completed thesis will be organised into the following chapters:
\begin{itemize}
	\item \textit{Chapter 1: Introduction} \\
	This chapter will introduce the thesis, providing a problem statement and motivations, defining research goals and contributions, and outlining the structure of this dissertation.
	\item \textit{Chapter 2: Literature Review} \\
	The literature review will examine in the existing work on VM-based and container-based server consolidation, discussing the fundamental concepts in this field in order to provide the reader with the necessary background. Multiple sections will then follow, 
	considering issues such as static consolidation, dynamic consolidation, and large-scale of consolidation problem. The focus of this review is on investigating server consolidation techniques that are based on Evolutionary Computation.
	\item \textit{Chapter 3: An EC Approach to the joint Allocation of Container and Virtual Machine} \\
	This chapter will establish a new model for the joint allocation of container and virtual machine problem. Furthermore, this chapter will introduce a new bi-level approach to solve this problem. One of the critical aspects of this approach is the representation of solution. Therefore, multiple representations will be proposed, analysed and compared.
	\item \textit{Chapter 4: An EC Approach to the Time-dependent Global Server Consolidation} \\
	In this chapter, a robustness measure of time-dependent server consolidation will be proposed. Furthermore, a time-dependent optimization techniques will be employed to solve this problem. In the first, it will only consider the previous allocation results to minimize both cost of migration as well as the overall energy consumption of data center. Then, this approach will be generalized to consider next consolidation.
	\item \textit{Chpater 5: A Genetic Programming-based Hybrid-heuristic Approach to Dynamic Consolidation} \\
	This chapter focuses on providing a Genetic Programming-based hybrid heuristic approach to automatic generate
	dispatching rules to dynamic consolidation problem.
	\item \textit{Chapter 6: A Preprocessing Algorithm for Large-scale Static Consolidation}\\
	This chapter proposes a preprocessing algorithm for large scale static consolidation to reduce the number of variables so that the static consolidation can be more scalable. 
	\item \textit{Chapter 7: Conclusions and Future Work}
	In this chapter, conclusions will be drawn from the analysis and experiments conducted in the different phases of this research, and the main findings for each one of them will be summarised. Additionally, future research directions will be discussed.

\end{itemize}


\section{Resources Required}
\subsection{Computing Resources}
An experimental approach will be adopted in this research, entailing the execution of exper-
iments that are likely to be computationally expensive. The ECS Grid computing facilities
can be used to complete these experiments within reasonable time frames, thus meeting this requirement.
\subsection{Library Resources}
The majority of the material relevant to this research can be found online, using the univer-
sity’s electronic resources. Other works may either be acquired at the university’s library, or
by soliciting assistance from the Subject Librarian for the fields of engineering and computer science.
\subsection{Conference Travel Grants}
Publications to relevant venues in this field are expected throughout this project, therefore
travel grants from the university are required for key conferences.