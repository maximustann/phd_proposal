\section{Background}

This chapter begins by providing an overall understanding of Cloud computing and its related research field.
Then, it narrows down to the Energy-aware resource management problem in Section \ref{C:}.


\subsection{An Overview of Cloud Computing}

%<<<<<<< HEAD
%\textcolor{Blue}{Try to clearfy the relationship among three stakeholders}

%Cloud computing allows their users to access Cloud resources from anywhere in the world. Software developers deploy their softwares in the Cloud in a form of service, hence, their customers can use them without installing on their local computers. Cloud computing has made one critical change in software industry, it separates the role of traditional service provider into service provider and infrastructure provider. As Wei \cite{Wei:2010fn} states, ``one provides the computing of services, and the other provides the services of computing''. Therefore, this separation add one more layer between service provider and users, as: Cloud provider, Cloud users (service provider), and End users. 
%=======
%Cloud computing allows their users to access Cloud resources from anywhere in the world. Software developers deploy their softwares in the Cloud in a form of service, hence, their customers can use them without installing on their local computers. Cloud computing has made one critical change in software industry, it separates the role of traditional service provider into service provider and infrastructure provider.  Therefore, this separation add one more layer between service provider and users, as: Cloud provider, Cloud users (service provider), and End users. 
%>>>>>>> 2011cb91584b90ffd83cada4af7a86f40fe3836a

Cloud computing has completely reformed the software industry \cite{Buyya:2009ix} by providing three major benefits to Cloud users.
First, Cloud users do not need upfront investment in hardwares (e.g PMs and networking devices) and pay for hardwares' maintenance. 
Second, Cloud users do not need to worry about the limited resources which can obstruct the performance of their services when unexpected high demand occurs. The elastic nature of cloud can dynamic allocate and release resources for a service. In addition, Cloud user can pay as much as the resource under a \emph{pay-as-you-go} policy.
Third, Cloud users can publish and update their applications at any location 
as long as there is an Internet connection. 
These advantages allow anyone or organization to deploy their softwares on Cloud in
a reasonable price. 



Each of the Cloud stakeholders has their goal. End users consume application. They want applications with better quality and lower price. Cloud users develop and deploy softwares on Cloud. They also desire a high Quality of service (QoS) by specifying a Service Level Agreement (SLA) with the Cloud provider. 

They are bonded by the Quality of service and computing resource. Quality of service and the computing resources are the two sides of a coin. 


For End users, they consume applications deployed on Cloud. 




End users consume the application deployed on Cloud. They require a guarantee quality of softwares including functional requirements which are the functionalities defined by Cloud users, and non-functional requirements such as availability, security, and network latency.

Cloud users deploy their software on Clouds. They want to increase the profit by increasing income and decreasing expense. In order to accomplish this goal, they can attract more End suers by improving the functionality of softwares and the non-functionality features by guaranteeing Quality of Service (QoS). To improve the non-functionality features, Cloud users need to reserve enough resources as well as minimizing the resource so that the cost is low.

% service capacity planning is the core process. The capacity planning has two conflicting objectives, on one hand, it must meet End users' QoS requirement by using enough resources.  On the other hand, the cost must be minimized. In pre-Cloud era, the capacity planning determines the upfront investment in infrastructure, therefore, capacity, reliability, and scalability are all need to be carefully considered and balanced. In Cloud environment, the burden of capacity planning is largely released by elastic resource management and the pay-as-you-go policy.

Cloud users identify a list of critical QoS parameters called Service Level Agreement (SLA) which specifies the non-functional requirements such as throughput, latency, and availability. These QoS parameters are mapped to resources (e.g. CPU, memory, network bandwidth) which can satisfy these requirements. Violation of SLA will lead to penalty and decreasing in number of users. Therefore, in essence, the key to attract more users is an effective resource management system which can rapidly react to the fluctuating resource demand. 

Beside increase the income, reduce the expense is another way to improve profit. As previous section mentioned, energy consumption is the main source of expense. In energy consumption, server energy consumption is the core that needs to be improved. 

\subsection{Energy-aware Resource Management}
\subsection{An Overview of Evolutionary Computation}


In order to understand Cloud computing, firstly we will illustrate the five essential elements of Cloud computing and their advantages.

Cloud computing has five essential elements:
\begin{enumerate}
 \item On-demand self-service, it means a Cloud user can require computing resources (e.g CPU time, storage, software use) without the interaction with Cloud provider.
 \item Broad network access, Computing resources are connected and delivered over the network.
 \item Resource pool, a Cloud provide has a ``pool'' of resources which are normally virtualized servers. In IaaS, it provides predefined sizes of VMs. In PaaS, the resources are `invisible' to Cloud users who have no knowledge or ability to control. 
 \item Rapid elasticity, from the perspective of Cloud users, computing resources are assigned and released in
real time. In addition, the resources assign to their software is ``infinite''. Therefore, Cloud users do not need to worried about the scalability of their applications.
 \item Measured Service provides an accurate measure of the usage of computing resources. It is fundamental to the pay-as-you-go policy.
\end{enumerate}

\textcolor{Blue}{What is your purpose to describe the following content?}
\textcolor{Red}{
	I would like to discuss the differences, advantages of disadvantage of the resource management in different service models. Therefore, after illustrate how they are work. The point is to compare the resource management. And then, lead to a new service model. And the advantage of new service model should be obvious. 
}\\
Traditional Cloud computing has three service model as illustrated in Figure \ref{}.
Infrastructure as a Service (IaaS), Platform as a Service (PaaS), and Software as a Service. 
\subsection{Resource Management}
Scope of Cloud computing resource management.
\begin{enumerate}
 \item Actors
 \item Management Objectives
 \item Resource Types
 \item Enabling Technologies
\end{enumerate}






