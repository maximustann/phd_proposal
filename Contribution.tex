\chapter{Proposed Contributions and Project Plan}\label{C:con}

This thesis will contribute to the field of Cloud Computing by proposing novel solutions for improving energy efficiency in container-based clouds. It will also contribute to the field of Evolutionary Computation (EC) by proposing new representations and genetic operators for solving bilevel optimization problems. The proposed contributions of this project are listed below:
 
\begin{enumerate}
	\item Two new bilevel models for predicting energy consumption in container-based clouds: \textbf{first}, a new bilevel energy model for initial placement of application; \textbf{second}, a new bilevel energy and migration model for periodic placement of application with the consideration of three types of workload. The above two models will address the relationship between energy consumption and five factors such as locations of container, tyeps of VM, and locations of VM.  
	These two bilevel models can be used in optimizing energy consumption and minimizing migration cost in container-based clouds.

	% A new bilevel model for the joint placement of container and VM problem.  The bilevel model will address the relationship between containers, VMs, PMs, and energy consumption. The bilevel model can be used in optimizing the energy consumption in a container-based cloud data center.
	\item A new EC-based bilevel single-objective optimization algorithm for solving the initial placement of application problem based on previously proposed bilevel energy model. The new algorithm contributes to both Cloud computing and EC. From the perspective of cloud computing, the algorithm produces a resource allocation with better energy efficiency in container-based clouds. From the perspective of EC, the algorithm develops new representations, search mechanisms for bilevel optimization problems. In addition, the work also develops a hybrid approach that combines EC with clustering techniques.

	\item A new EC-based bilevel multi-objective algorithm with Pareto front approach for solving the periodic placement of application. This algorithm will consider three types of predictable workload: static, periodic, and linear continuously changing workloads. From the perspective of cloud computing, the algorithm is expected to achieve better energy efficiency and migration cost than existing VM-based approaches with consideration of three types of workload. From the perspective of EC, the algorithm proposes specific representations and genetic operators for three types of workload. 

	\item A new genetic programming hyper-heuristic (GP-HH) approach for solving single-objective dynamic placement of application with three types of predictable workload and two types of unpredictable workload in \textbf{VM-based clouds}. The proposed GP-HH focuses on solving the single-level of placement: VM to PM. From the perspective of cloud computing,
	the GP-HH can be used in VM-based clouds and it is expected to fast allocate VMs to PMs and achieve a near-optimal solution in energy consumption.
	From perspective of EC,  the GP-HH is the first to solve resource allocation problem in container-based clouds. The algorithm will propose new primitive set, search mechanisms.

	\item A new cooperative GP-HH approach for solving single-objective dynamic placement of application with five types of workload in \textbf{container-based clouds}. This work is based on the previously proposed GP-HH approach.
	From perspective of cloud computing, the cooperative GP-HH can quickly allocate containers and VMs to achieve near optimal solutions in terms of energy consumption.
	From perspective of EC, the new cooperative GP-HH method will solve bilevel dynamic problem.
\end{enumerate}

\section{Overview of Project Plan}
Six overall phases have been defined in the initial research plan for this PhD project, as
shown in Table \ref{tab:plan}. The first phase, which comprises reviewing the relevant literature, investigating both VM-based and container-based server consolidation algorithms, and producing the proposal, has
been completed. The second phase, which corresponds to the first objective of the thesis, is
currently in progress and is expected to be finished on time, thus allowing the remaining
phases to also be carried out as planned.


\begin{table}[]
\centering
\caption{Phases of project plan}
\label{tab:plan}
\scalebox{0.85}{
\begin{tabular}{|c|l|l|}
\hline
\multicolumn{1}{|l|}{Phase} & Task                                                                                                                                                                                 & Duration (Months) \\ \hline
1                           & \begin{tabular}[c]{@{}l@{}}Reviewing literature, overall design, selection of datasets\\ and writing the proposal\end{tabular}                                                       & 12 (Complete)     \\
2                           & \begin{tabular}[c]{@{}l@{}}Develop a single-objective EC-based approach for the joint\\ allocation of containers and VMs\end{tabular}                                                & 7                 \\
3                           & \begin{tabular}[c]{@{}l@{}}Develop multi-objective EC-based approaches for container-based \\ cloud in periodic placement of application with considering various types of workload\end{tabular} & 7                 \\
4                           & \begin{tabular}[c]{@{}l@{}}Develop a cooperative Genetic programming based \\ hyper-heuristic approach for dynamic placement.\end{tabular}                                         & 7                 \\
5                           & Writing the thesis                                                                                                                                                                   & 6                 \\ \hline
\end{tabular}
}
\end{table}



\section{Project Timeline}
The phases included in the plan above are estimated to be completed following the timeline
shown in Table \ref{table-timeline}. The timeline will serve as a guide throughout this project. Note that part of the first phase has already been done. Therefore the timeline only shows the estimated remaining time for full completion.


\begin{table}
\protect\caption{Time Line}
\label{table-timeline}
\footnotesize
\begin{center}
\scalebox{0.8}{
\begin{tabular}{|l||cccc|cccc|cccc|}

\hline
  \textbf{Task}      & \multicolumn{4}{c}{ }&  \multicolumn{4}{c}{Months}&\multicolumn{4}{c|}{ }  \\ \hline
&2&4&6&8&10&12&13&16&18&20&22&24 \\ \hline
Literature Review and Updating &x&x&x&x&x&x&x&x&x&x&x&x \\ [1mm]

\cellcolor{LightCyan}Develop a new bilevel energy model&x&&&&&&&&&&& \\ 

Propose a new EC based bilevel optimization approach to solve the &x&x&&&&&&&&&& \\ 
initial placement of application & &&&& & & & & & & &  \\

\cellcolor{LightCyan} Improve the scalability of the proposed & & x&x & & & & & & & & & \\
\cellcolor{LightCyan}  EC-based approach up to one thousand applications & & & & & & & & & & & &   \\ [1mm]
Propose a multi-objective bilevel model for periodic placement  & & & &x&& & & & & & & \\ 
with consideration of static workload. & & & &&& & & & & & & \\ [1mm]

\cellcolor{LightCyan} Propose a multi-objective bilevel EC-based algorithm for & & & & & x&x&& & & & & \\ 
\cellcolor{LightCyan}  periodic placement of application with Pareto front approach.  & & & & & & & & & & & &   \\ [1mm]

Extend the bilevel multi-objective model for adapting  &&&&&&x&&&&&&\\ [1mm]
three types of predictable workload.  &&&&&&&&&&&&\\ [1mm]

\cellcolor{LightCyan}Extend the previous EC-based multi-objective algorithm &&&&&&&x&&&&&\\ [1mm]
\cellcolor{LightCyan}to adapt to three types of workload. &&&&&&&&&&&&\\ [1mm]

Develop a GP-based hyper-heuristic (GP-HH) algorithm for  &&&&&&&&x&x&&& \\ [1mm]
automatically generating dispatching rules for the single-level placement. & & & & & &&& & & & &  \\[1mm]

\cellcolor{LightCyan} Conduct feature extraction on the three predictable workloads and two  &&&&&&&&&x&&&\\ 
\cellcolor{LightCyan} unpredictable workloads to construct a new primitive set. &&&&&&&&&&&& \\ [1mm]

Develop a cooperative GP-HH approach for automatically   &&&&&&&&&x&x&& \\ [1mm]
generating dispatching rules for placing both containers and VMs. &&&&&&&&&&&&\\ [1mm]
\cellcolor{LightCyan}  Writing the first draft of the thesis &&&&&&&&&&x&x&  \\ \hline
 Editing the final draft   &&&&&&&&&&x&x&x \\ \hline

\end{tabular}
}
\end{center}
\end{table}


\section{Thesis Outline}
The completed thesis will be organized into the following chapters:
\begin{itemize}
	\item \textit{Chapter 1: Introduction} \\
	This chapter will introduce the thesis, providing a problem statement and motivations, defining research goals and objectives, and outlining the structure of the final thesis.
	\item \textit{Chapter 2: Literature Review} \\
	The literature review will illustrate the fundamental background of Cloud computing, resource management, and server consolidation. It will examine the existing work on VM-based and container-based server consolidation and discuss concepts in this field in order to provide readers with the necessary background. Multiple sections will consider issues such as initial placement of application, periodic placement, and dynamic placement of application. The focus of this review is on investigating server consolidation techniques.
	\item \textit{Chapter 3: Develop EC-based approaches for the single objective joint placement of containers and VMs for initial placement of application.} \\
	This chapter will establish a new bilevel energy model for the joint placement of container and VM. Based on this model, this chapter will introduce a new EC-based bilevel algorithm combined with clustering technique and heuristics to solve the initial placement of application. 
	\item \textit{Chapter 4: Develop multi-objective EC-based approaches  for periodic placement of application} \\
	This chapter proposes a new bilevel energy and migration model based on a previously proposed energy model with three types of workload.  This chapter will also propose new EC-based approaches for bilevel multi-objective periodic placement, considering three types of workload. It is then followed by algorithm performance evaluation that contains an experiment design, setting, results and analysis.
	\item \textit{Chpater 5: Develop a  single-objective cooperative Genetic Programming hyper-heuristic (GP-HH) approach for automatically generating dispatching rules for dynamic placement of application} \\
	This chapter focuses on providing a Genetic Programming-based hybrid heuristic approach to automatically generate dispatching rules to a dynamic consolidation problem. This chapter will propose two algorithms -- a GP-HH for single-level of placement: VM-PM and a cooperative GP-HH for bilevel placement: container-VM and VM-PM.
	\item \textit{Chapter 7: Conclusions and Future Work}
	In this chapter, conclusions will be drawn from the analysis and experiments conducted in the different phases of this research, and the main findings for each phase of them will be summarized. Additionally, future research directions will be discussed.

\end{itemize}


\section{Resources Required}
\subsection{Computing Resources}
An experimental approach will be adopted in this research, entailing the execution of experiments that are likely to be computationally expensive. The ECS Grid computing facilities
can be used to complete these experiments within reasonable time frames, thus meeting this requirement.
\subsection{Library Resources}
The majority of the material relevant to this research can be found on-line, using the university electronic resources. Other works may either be acquired at the university library, or
by soliciting assistance from the Subject Librarian for the fields of engineering and computer science.
\subsection{Conference Travel Grants}
Publications to relevant venues in this field are expected throughout this project. Therefore travel grants from the university are required for key conferences.