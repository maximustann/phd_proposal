\chapter{Proposed Contributions and Project Plan}\label{C:con}

This thesis will contribute to the field of Cloud Computing by proposing novel solutions for bilevel optimization of the joint allocation of container and VM, and to the field of Evolutionary Computation by proposing new representations and genetic operators in evolutionary algorithms. The proposed contributions of this project are listed below:
 
\begin{enumerate}
	\item A new bilevel model for the joint placement of container and VM problem.  The bilevel model will address the relationship between containers, VMs, PMs, and energy consumption. The bilevel model can be used in optimizing the energy consumption in a container-based cloud data center.
	\item An EC-based bilevel optimization algorithm for the application initial placement based on previous proposed bilevel model. This algorithm is expected to achieve better performance than existing VM-based approaches in terms of energy consumption. 
	\item An EC-based bilevel optimization algorithm with a clustering-based pre-processing approach to improve the scalability of previous proposed algorithm. This work can be used to reduce the complexity of the bilevel problem by combining containers into larger groups.
	\item EC-based bilevel multi-objective approaches for periodic optimization with considering various types of workload.
	This work will modify previous model to adapt to the multi-objective problem. It will also propose several multi-objective approaches with aggregation and Pareto front methods. In addition, a uniform representation for various workloads will be proposed to make the algorithm more resilient.
	\item A new genetic programming hyper-heuristic (GP-HH) approach for single-objective dynamic placement with various types of workload. The proposed GP-HH is expected to learn good placement patterns from previous placement solutions output from previous proposed algorithms. After the learning process, GP-HH can automatically generate heuristics. These generated heuristics can achieve fast dispatching containers to suitable PMs. In addition, the containers and VMs can achieve a near-optimal solution in energy consumption.
\end{enumerate}

\section{Overview of Project Plan}
Six overall phases have been defined in the initial research plan for this PhD project, as
shown in Table \ref{tab:plan}. The first phase, which comprises reviewing the relevant literature, investigating both VM-based and container-based server consolidation algorithms, and producing the proposal, has
been completed. The second phase, which corresponds to the first objective of the thesis, is
currently in progress and is expected to be finished on time, thus allowing the remaining
phases to also be carried out as planned.


\begin{table}[]
\centering
\caption{Phases of project plan}
\label{tab:plan}
\scalebox{0.85}{
\begin{tabular}{|c|l|l|}
\hline
\multicolumn{1}{|l|}{Phase} & Task                                                                                                                                                                                 & Duration (Months) \\ \hline
1                           & \begin{tabular}[c]{@{}l@{}}Reviewing literature, overall design, selection of datasets\\ and writing the proposal\end{tabular}                                                       & 12 (Complete)     \\
2                           & \begin{tabular}[c]{@{}l@{}}Develop a single-objective EC-based approach for the joint\\ allocation of containers and VMs\end{tabular}                                                & 7                 \\
3                           & \begin{tabular}[c]{@{}l@{}}Develop multi-objective EC-based approaches for container-based \\ cloud in periodic optimization with considering various types of workload\end{tabular} & 7                 \\
4                           & \begin{tabular}[c]{@{}l@{}}Develop a cooperative Genetic programming based \\ hyper-heuristic approach for dynamical placement.\end{tabular}                                         & 7                 \\
5                           & Writing the thesis                                                                                                                                                                   & 6                 \\ \hline
\end{tabular}
}
\end{table}



\section{Project Timeline}
The phases included in the plan above are estimated to be completed following the timeline
shown in Table \ref{table-timeline}, which will serve as a guide throughout this project. Note that part of the first phase has already been done, therefore the timeline only shows the estimated remaining time for its full completion.


\begin{table}
\protect\caption{Time Line}
\label{table-timeline}
\footnotesize
\begin{center}
\scalebox{0.88}{
\begin{tabular}{|l||cccc|cccc|cccc|}

\hline
  \textbf{Task}      & \multicolumn{4}{c}{ }&  \multicolumn{4}{c}{Months}&\multicolumn{4}{c|}{ }  \\ \hline
&2&4&6&8&10&12&13&16&18&20&22&24 \\ \hline
Literature Review and Updating &x&x&x&x&x&x&x&x&x&x&x&x \\ [1mm]
Develop a new model for the joint placement &x&&&&&&&&&&& \\ 
% \rowcolor{LightCyan}
of VMs and containers & & && & & &  & & & & &  \\ [1mm]

Develop an EC-based bilevel optimization approach & &x&&& & & & & & & &\\ 
for the joint placement of VMs and containers & &&&& & & & & & & &  \\
Improve the scalability of the EC-based approach with & & &x & & & & & & & & & \\
 a pre-processing method & & & & & & & & & & & &   \\ [1mm]
Modify the model and develop a baseline aggregation & & & &x&x& & & & & & & \\ approach for the multi-objective problem & & & &&& & & & & & & \\ [1mm]
Develop an EC-based approach to solve the multi-objective & & & & & &x&x& & & & & \\ joint allocation problem
with Pareto front approach & & & & & &&& & & & &  \\[1mm]
Propose an EC-based multi-objective algorithm for periodic &&&&&&&&x&&&&\\
optimization considering various types of predictable workload &&&&&&&&&&&& \\ [1mm]
Develop a baseline GP-based hyper-heuristic approach &&&&&&&&&x&x&& \\ [1mm]
Construct a GP primitive set by applying feature extraction &&&&&&&&&x&x&&\\ on various types of application workload &&&&&&&&&&&& \\ [1mm]
Develop a Cooperative GP-HH approach to evolve dispatching &&&&&&&&&x&x&&\\ rules for placing container and VMs &&&&&&&&&&&&\\ [1mm]
Writing the first draft of the thesis &&&&&&&&&&x&x&  \\ \hline
 Editing the final draft   &&&&&&&&&&x&x&x \\ \hline

\end{tabular}
}
\end{center}
\end{table}


\section{Thesis Outline}
The completed thesis will be organized into the following chapters:
\begin{itemize}
	\item \textit{Chapter 1: Introduction} \\
	This chapter will introduce the thesis, providing a problem statement and motivations, defining research goals and objectives, and outlining the structure of the final thesis.
	\item \textit{Chapter 2: Literature Review} \\
	The literature review will illustrate the fundamental background of Cloud computing, resource management, and server consolidation. It will examine the existing work on VM-based and container-based server consolidation, discuss concepts in this field in order to provide readers with the necessary background. Multiple sections will consider the issues such as application initial placement, periodic placement, and dynamic placement. The focus of this review is on investigating server consolidation techniques.
	\item \textit{Chapter 3: Develop EC-based approaches for the single objective joint placement of containers and VMs for application initial placement.} \\
	This chapter will establish a new bilevel model for the joint placement of container and virtual machine problem. Based on this model, this chapter will introduce a new EC-based bi-level approach to solve the application initial placement problem. 
	\item \textit{Chapter 4: Develop EC-based approaches for the multi-objective joint allocation problem for periodic optimization} \\
	This chapter will first modify the previous proposed bilevel model to adapt to multi-objective problem with various workload.  It will also propose new EC-based approaches for the bilevel multi-objective joint placement of containers and VMs with considering various types of workload. It is then followed by algorithm performance evaluation that contains an experiment design, setting, results and analysis.
	\item \textit{Chpater 5: Develop a hyper-heuristic single-objective Genetic Programming (GP) approach for automatically generating dispatching rules for dynamic placement} \\
	This chapter focuses on providing a Genetic Programming-based hybrid heuristic approach to automatic generate dispatching rules to dynamic consolidation problem. A cooperative GP-HH will be proposed in this chapter for generating dispatching rules for two-level of placement.
	\item \textit{Chapter 7: Conclusions and Future Work}
	In this chapter, conclusions will be drawn from the analysis and experiments conducted in the different phases of this research, and the main findings for each one of them will be summarized. Additionally, future research directions will be discussed.

\end{itemize}


\section{Resources Required}
\subsection{Computing Resources}
An experimental approach will be adopted in this research, entailing the execution of experiments that are likely to be computationally expensive. The ECS Grid computing facilities
can be used to complete these experiments within reasonable time frames, thus meeting this requirement.
\subsection{Library Resources}
The majority of the material relevant to this research can be found on-line, using the university electronic resources. Other works may either be acquired at the university library, or
by soliciting assistance from the Subject Librarian for the fields of engineering and computer science.
\subsection{Conference Travel Grants}
Publications to relevant venues in this field are expected throughout this project, therefore
travel grants from the university are required for key conferences.